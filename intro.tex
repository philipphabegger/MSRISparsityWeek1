\chapter{Introduction}

These note reflect the contents of  one week  at the MSRI
Summer Graduate School title ``Sparsity of Algebraic Points'' which
was held (virtually) from June 7, 2021 until June 18, 2021.

Readers beware, the usual disclaimers apply! The notes are in flux and
subject to change at every moment. They are not meant to give a
complete overview of this field.

One of the most classical areas in mathematics, dating back to ancient
times, is the study of solutions of polynomials equations in integer
or rational unknowns. These \textit{diophantine equations} have
provided a rich source of new mathematics in the last 2000 years.
Number fields, \textit{i.e.}, finite extensions of the field of
rational numbers, entered the picture in the 19th century. They are
tools in studying diophantine equations. But it also makes sense to
investigate solutions of diophantine equations with values in number
fields or the ring of integers.

%One theme that we will explore in this week is the observation that
More recently, we began studying solutions of polynomial equations in
a further class of numbers, so-called \textit{special points}. Special
points is not a precise term but rather refers to complex numbers, or
more general points on certain varieties, that have rich arithmetic
and analytic properties.

In this one week course we will be interested in three classes of
special points:

\begin{itemize}
\item Roots of unity, \textit{i.e.}, complex numbers of the form
  $e^{2\pi i q}$ with $q$ a rational number. These are precisely the
  points of finite order of the multiplicative group $\IC^\times$. 

\item Consider an elliptic curve $E$ defined over a number field
  $K\subset\IC$. We regarding points in $E(\IC)$ of finite order as
  special points. More generally, torsion points on abelian varieties
  can be understood as special.

\item Finally, to any elliptic curve $E$ defined over $\IC$ we can
  attach its ring of endomorphisms
  $$\mathrm{End}(E) = \{\psi \colon E\rightarrow E : \psi\text{
    is a morphism of varieties and }\psi(0)=0\},$$
  here $0\in E(\IC)$ is the neutral element.\footnote{Addition on
    $\mathrm{End}(E)$ is pointwise addition and multiplication is
    composition. It is a theorem that $\mathrm{End}(E)$ is a
    commutative ring for any elliptic curve defined over a field of
    characteristic $0$.}
  The  structure of $E$ as an algebraic group gives provides
  multiplication-by-$N$ endomorphisms $[N] \in \mathrm{End}(E)$ for
  all $N\in\IZ$. So $\mathrm{End}(E) \supset \IZ$.
  We say that $E$ has complex multiplication if
  $\mathrm{End}(E)\not=\IZ$.
  Such an elliptic curve curve corresponds to a \textit{special
    point}, also called \textit{CM points},
  on the coarse moduli space of all elliptic curves.   
\end{itemize}

As we will see, these three cases have an important similarity: special
points are images of certain algebraic points under an analytic map.
We will discuss all three cases in more detail during the next week. 

In the first chapter we investigate the Ihara--Serre--Tate Theorem on
tuples of roots of unity on algebraic curves and the
Andr\'e--Edixhoven Theorem on CM points on plane algebraic curves.

A little over a decade ago, Pila and Zannier introduced o-minimal
geometry to this sort of problem. We will introduce o-minimal geometry
and see some of its applications. The second chapter is an overview of
o-minimal geometry. We will also handle the Pila--Wilkie Theorem.

Chapter 3 is devoted to function-theoretic version of the classical
Lindemann--Weiestrass Theorem and the (unproven) Schanuel Conjecture.
These results are sometimes called Ax--Schanuel or
Ax--Lindemann--Weierstrass Theorems and play an important role in the
Pila--Zannier approach.

Later on in Chapter 4 we will prove the Manin--Mumford Conjecture for
abelian varieties, the special points are torsion points. We will also
sketch a proof of the Andr\'e--Oort Conjecture in the plane.

The final chapter is devoted to selected topics in unlikely
intersections. Here special points are replaced by \textit{special
curves}. We will indicate how to adjust the Pila--Zannier approach. We
will sketch a proof of a result of Masser and Zannier on a family of
elliptic curves.

%%% Local Variables:
%%% TeX-master: "main"
%%% End:
