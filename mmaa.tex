\chapter{The Conjectures of Manin--Mumford and Andr\'e--Oort}

\section{Overview}

Let $A$ be an abelian variety defined over $\IC$.

\begin{definition}
  The torsion group $A_{\mathrm{tors}}$ of $A$ is the subgroup of points
  of finite order of the abelian group $A(\IC)$.
\end{definition}

In this chapter we will sketch a proof of the following theorem,
originally due to Raynaud (who proved it without assume that the
everything in sight is defined over a number field).

\begin{theorem}[Manin--Mumford Conjecture]
  Suppose $K\subset\IC$ is a number field and $A$ is an abelian
  variety defined over $K$. 
  Let $V\subset A$ be an irreducible closed subvariety, also defined over a
  number field $K$. Then
  \begin{equation*}
    V(\IC) \cap A_{\mathrm{tors}} \text{ lies Zariski dense in
      $V$}\quad\Leftrightarrow \quad
    \text{$V$ is a torsion coset of $A$.}
  \end{equation*}
\end{theorem}

Recall that a torsion coset of $A$ is the translate of an abelian
subvariety of $A$ by a point in $A_{\mathrm{tors}}$. 

This is the generalization of the translation to abelian varieties of
the Ihara--Serre--Tate Theorem, Theorem~\ref{thm:ist}. In view of this
analogy it is reasonable to declare torsion cosets of $A$ as the
special subvarieties of $A$.

There are many proofs of Theorem~\ref{thm:mm}. We will follow the
o-minimal route and use the Pila--Wilkie Theorem (a second time). The
idea to use the Pila--Wilkie counting theorem in this context goes
back to Zannier. We will follow the proof by Pila and
Zannier~\cite{PilaZannier}. The basic idea is to reinterpret torsion
points of $A$ as points in $\mathrm{T}_0(\IC)$ that are rational with
respect to a fixed period lattice basis.

This powerful approach has many applications. For example, it led to
the first condition proof of the Andr\'e--Oort Conjecture for
subvarieties of $Y(1)^m$ by Pila~\ref{Pila:AO}. Later on, and after work of
many people including Daw, Klingler, Pila, Orr, Ullmo, Yafaev the
approach was extended to the moduli space of abelian varieties by
Tsimerman. Here important work of
Andreatta--Goren--Howard--Madapusi-Pera and Yuan--Zhang on the average
Colmez conjecture and Peterzil--Starchenko on definability was used.

We will stick to the case of subvarieties of $Y(1)^m$.
Recall that we defined special points in
Definition~\ref{def:specialY1m}. We will define special subvarieties
of $Y(1)^m$ here. The modular transformation polynomials $\Phi_N$
were introduced in Definition-Lemma~\ref{deflem:modtranspoly}. 

\begin{definition}
  An irreducible subvariety $V$ of $Y(1)^m$ is called special if it is
  an irreducible component of the zero locus of
  \begin{equation*}
    \Phi_{N_j}(X_{i_j},X_{i'_j}) \quad (j\in \{1,\ldots,r\})
  \end{equation*}
  for some  $N_1\ldots,N_r\in\IN$ and $i_1,i'_1,\ldots,i_r,i'_r\in
  \{1,\ldots,r\}$. 
\end{definition}

\begin{example}
  For $m=3$ there is an irreducible component of the vanishing locus of
  $\Phi_2(X_1,X_1)$ and $\Phi_3(X_2,X_3)$ of the form
  \begin{alignat*}1
    \{(1728,x_2,x_3) : \,\,&\text{there is a degree $3$ isogeny between
      the elliptic curves over $\IC$}\\ &\text{with $j$-invariants $x_2$ and $x_3$}\}
  \end{alignat*}
  as the elliptic curve over $\IC$  of $j$-invariant 1728
  admits an endomorphisms of degree $2$. This endomorphism arises as
  $2 = (1+i)(1-i)$ in $\IZ[i]$. The other possible $X_1$ 
  coordinates are $8000$ (endomorphism ring $\IZ[\sqrt{-2}]$) and
  $-3375$ (endomorphism ring $\IZ[\sqrt{-7}/2+1/2]$).

  So allowing $i_j=i'_j$ in the definition of special subvariety of
  $Y(1)^m$ leads to constant coordinates. 
\end{example}

\begin{theorem}[Andr\'e--Oort Conjecture for $Y(1)^m$, Pila's Theorem~\cite{Pila:AO}]
  Let $V\subset Y(1)^m$ be an irreducible closed subvariety. Then
  \begin{equation*}
    V(\IC) \cap Y(1)^m_{\mathrm{special}} \text{ lies Zariski dense in
      $V$}\quad\Leftrightarrow \quad
    \text{$V$ is special.}
  \end{equation*}    
\end{theorem}

\section{Arithmetic of Torsion Points}

Here we collect some fundamental facts about the torsion points of an
abelian variety $A$ defined over a number field $K\subset \IC$ of dimension $g$. We will
assume that $A$ is presented as a subvariety of $\IP^n$. 

As a complex Lie group $A(\IC)$ is isomorphic to
$\mathrm{T}_0(A)/\Omega$ where $\Omega$ is the period lattice, a
discrete subgroup of $\mathrm{T}_0(A)$ of rank $2g$. So the group
$A(\IC)$ is isomorphic to $(\IR/\IZ)^{2g}$.

Say $N\in\IN$ and let $A[N] =  \ker[N] = \{x\in A(\IC) : [N](x) = 0\}$, here
$[N]$ is the multiplication-by-$N$ morphism.
As a group we have $A[N] \cong (\IZ/N\IZ)^{2g}$ and in particular
$\#A[N] = N^{2g}$.

Let $\overline K$ be the algebraic closure of $K$ in $\IC$.
Then $A[N] = \{x\in A(\overline K) : [N](x)=0}$. In fact, all torsion
points of $A$ are $K$-valued points.

%Now suppose that $A$ is defined over a number field $K\subset\IC$.
Let $\absgalk$ denote the absolute Galois group of $K$. Let
$[x_0:\cdots:x_n]\in \IP^n(\overline{K})$ with $(x_0,\ldots,x_n) \in
\overline{K}^{n+1}\ssm\{0\}$. Say $\sigma\in \absgalk$, then
\begin{equation*}
  \sigma([x_0:\cdots:x_n]) = [\sigma(x_0):\cdots:\sigma(x_m)]
\end{equation*}
defines a well-defined action of $\absgalk$ on $\IP^n(\overline K)$.

\begin{exercise}\leavevmode
  \begin{enumerate}
  \item [(i)]
    Assume $x\in \IP^n(\overline K)$. Show that $x\in \IP^n(K)$ if and
    only if  $\sigma(x)=x$ for all
    $x\in \absgalk$.
  \item[(ii)] Let $W\subset\IP^n$ be an irreducible subvariety defined
    over $\overline K$. Find a sensible definition for $\sigma(W)$,
    for $\sigma\in \absgalk$.  Show that $W$ is the zero set of
    polynomials with coefficients in $K$ if and only if $\sigma(W)=W$
    for all $\sigma\in \absgalk$.
  \end{enumerate}
\end{exercise}


As $A$ is cut out in $\IP^n$ by homogeneous polynomials with
coefficients in $K$ we find that the action of $\absgalk$ restricts to
an action on
$A(\overline K)$ that fixes $A(K)$.

The $[N]\colon A\rightarrow A$ is, at least Zariski-locally, presented
by a collection of homogeneous polynomials. That is, in a Zariski
neighborhood of any point $P_0\in A(\overline K)$ there exist
$f_{N,0},\ldots,f_{N,n}\in K[X_0,\ldots,X_n]$, all homogeneous of the
same degree, such that 
$[N](P) = [f_{N0}(P):\cdots:f_{Nn}(P)]$ for all $\overline K$-points  $P$ in the said
neighborhood.
Patching together neighborhoods yields
We have
$$\sigma([N](P)) = [f_{N0}(\sigma(P)):\cdots:f_{Nn}(\sigma(P))] =
[N](\sigma(P))$$
holds for \textit{all} $P\in A(\overline K)$.


\begin{lemma}\leavevmode
  \begin{enumerate}
  \item [(i)]
    The
    action of $\absgalk$ on $A(\overline K)$ restricts to an action on
    $A[N]$ for all $N\in\IN$.
    Let us denote it by
      $\rho_N\colon \absgalk\rightarrow \mathrm{Aut}(A[N])$.
  \item[(ii)]
    The extension $K(A[N])/K$ is Galois
    of degree at most $N^{4g^2}$.
  \item[(iii)] Suppose $P\in A[N]$, then $K(P)/K$ has degree at most
    $N^2$. 
  \end{enumerate}
\end{lemma}
\begin{proof}
  Part (i) has proved before the lemma.  For part (ii) observe that
  $A[N]\cong
  (\IZ/N\IZ)^{2g}$ implies
  $\#\mathrm{Aut}(A[N])\le \mat{2g}(\IZ/N\IZ)$.
  So  $\rho_N$ has finite image with order at most $N^{4g^2}$. The kernel
  of $\rho_N$ is the Galois group $\mathrm{Gal}(\overline K/K(A[N]))$.
  So $K(A[N])/K$ is normal of degree at most $N^{4g^2}$.
  Part (iii) follows from (i) and as $\#A[N]=N^2$  
\end{proof}

This is just the analog of the action of the Galois group $\mathrm{Gal}(\IQbar/\IQ)$
on roots of unity in $\IC^\times$ that we investigate in
Section~\ref{sec:rootsof1}.

However, there are some notable important. First, the in contrast to
the multiplicative setting, there is no reason to believe that
$K(A[N])/K$ is an abelian extension and in general it is not.
Moreover, in general there is no simple formula for the degree of the
extension $K(A(P))/K$ if $P$ has order $N$. In general, $K(A(P))/K$ is
not Galois.

For the sake of completeness let us state here a theorem of Serre for
elliptic curves.

\begin{theorem}[Serre]
  Suppose $E$ is an elliptic curve. There exists $c(E)>0$ such that
  if $p\ge c(E)$ is prime number, then 
\end{theorem}
Now suppose $V$ is an  subvariety of $A$ defined over $K$. We can
think of $V$ as the zero set on $A(\overline K)$ of homogeneous
polynomials with coefficients in $K$. 

It is important to distinguish between $A[N]$ and the subgroup
$A(K)[N] = \{x\in A(K) : [N](x)=0\}$. Indeed, the subgroup $A(K)[N]$ may be
smaller than $A(K)$.


\section{Proof of the Manin--Mumford Conjecture}

Let $A$ be an abelian variety defined over $\IC$. We return again to
the notation introduced in Example~\ref{ex:thetafunc}.
\begin{itemize}
\item We have an immersion $A\rightarrow\IP^n$
\item We have a uniformizing map $u\colon T_0(A)\rightarrow
  A(\IC)$  with the tangent space
  $\mathrm{T}_0(A)$. The map $u$ is a holomorphic, surjective group
  homomorphism.
\item The kernel $\ker u$ is a discrete subgroup $\Omega\subset\IC^g$
  of rank $2g$, the period lattice. We fix a basis $(\omega_1,\ldots,\omega_{2g})$ of
  $\Omega$ and obtain a fundamental domain.
\end{itemize}

In this setup it has already proved useful to work in 2 distinct
coordinate systems in $T_0(A)$.

\bigskip
\noindent The \textbf{complex coordinates} arise via an isomorphism
$\mathrm{T}_0(A)\rightarrow\IC^g$ of $\IC$-vector spaces. 

\bigskip
\noindent The \textbf{period coordinates} arise via an isomorphism
$\mathrm{T}_0(A)\rightarrow \IR^{2g}$ of $\IR$-vector spaces
induced by the basis $(\omega_1,\ldots,\omega_{2g})$. 


\section{Further Reading and Open Problems}



%%% Local Variables:
%%% TeX-master: "main"
%%% End:
