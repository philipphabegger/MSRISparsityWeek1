\chapter{The Conjectures of Manin--Mumford and Andr\'e--Oort}

\section{Overview}

Let $A$ be an abelian variety defined over $\IC$.

\begin{definition}
  The torsion group $A_{\mathrm{tors}}$ of $A$ is the subgroup of points
  of finite order of the abelian group $A(\IC)$.
\end{definition}

In this chapter we will sketch a proof of the following theorem,
originally due to Raynaud.

\begin{theorem}[Manin--Mumford Conjecture]
  Let $V\subset A$ be an irreducible closed subvariety. Then
  \begin{equation*}
    V(\IC) \cap A_{\mathrm{tors}} \text{ lies Zariski dense in
      $V$}\quad\Leftrightarrow \quad
    \text{$V$ is a torsion coset of $A$.}
  \end{equation*}
\end{theorem}

Recall that a torsion coset of $A$ is the translate of an abelian
subvariety of $A$ by a point in $A_{\mathrm{tors}}$. 

This is the generalization of the translation to abelian varieties of
the Ihara--Serre--Tate Theorem, Theorem~\ref{thm:ist}. In view of this
analogy it is reasonable to declare torsion cosets of $A$ as the
special subvarieties of $A$.

There are many proofs of Theorem~\ref{thm:mm}. We will follow the
o-minimal route and use the Pila--Wilkie Theorem (a second time). The
idea to use the Pila--Wilkie counting theorem in this context goes
back to Zannier. We will follow the proof by Pila and
Zannier~\ref{pilazannier}. The basic idea is to reinterpret torsion
points of $A$ as points in $\mathrm{T}_0(\IC)$ that are rational with
respect to a fixed period lattice basis.

This powerful approach has many applications. For example, it led to
the first condition proof of the Andr\'e--Oort Conjecture for
subvarieties of $Y(1)^m$ by Pila~\ref{Pila:AO}. Later on, and after work of
many people including Daw, Klingler, Pila, Orr, Ullmo, Yafaev the
approach was extended to the moduli space of abelian varieties by
Tsimerman. Here important work of
Andreatta--Goren--Howard--Madapusi-Pera and Yuan--Zhang on the average
Colmez conjecture and Peterzil--Starchenko on definability was used.

We will stick to the case of subvarieties of $Y(1)^m$.
Recall that we defined special points in
Definition~\ref{def:specialY1m}. We will define special subvarieties
of $Y(1)^m$ here. The modular transformation polynomials $\Phi_N$
were introduced in Definition-Lemma~\ref{deflem:modtranspoly}. 

\begin{definition}
  An irreducible subvariety $V$ of $Y(1)^m$ is called special if it is
  an irreducible component of the zero locus of
  \begin{equation*}
    \Phi_{N_j}(X_{i_j},X_{i'_j}) \quad (j\in \{1,\ldots,r\})
  \end{equation*}
  for some  $N_1\ldots,N_r\in\IN$ and $i_1,i'_1,\ldots,i_r,i'_r\in
  \{1,\ldots,r\}$. 
\end{definition}

\begin{example}
  For $m=3$ there is an irreducible component of the vanishing locus of
  $\Phi_2(X_1,X_1)$ and $\Phi_3(X_2,X_3)$ of the form
  \begin{alignat*}1
    \{(1728,x_2,x_3) : \,\,&\text{there is a degree $3$ isogeny between
      the elliptic curves over $\IC$}\\ &\text{with $j$-invariants $x_2$ and $x_3$}\}
  \end{alignat*}
  as the elliptic curve over $\IC$  of $j$-invariant 1728
  admits an endomorphisms of degree $2$. This endomorphism arises as
  $2 = (1+i)(1-i)$ in $\IZ[i]$. The other possible $X_1$ 
  coordinates are $8000$ (endomorphism ring $\IZ[\sqrt{-2}]$) and
  $-3375$ (endomorphism ring $\IZ[\sqrt{-7}/2+1/2]$).

  So allowing $i_j=i'_j$ in the definition of special subvariety of
  $Y(1)^m$ leads to constant coordinates. 
\end{example}

\begin{theorem}[Andr\'e--Oort Conjecture for $Y(1)^m$, Pila's Theorem~\cite{Pila:AO}]
  Let $V\subset Y(1)^m$ be an irreducible closed subvariety. Then
  \begin{equation*}
    V(\IC) \cap Y(1)^m_{\mathrm{special}} \text{ lies Zariski dense in
      $V$}\quad\Leftrightarrow \quad
    \text{$V$ is special.}
  \end{equation*}    
\end{theorem}

\section{Arithmetic of Torsion Points}

\section{The Manin--Mumford Conjecture}

Let $A$ be an abelian variety defined over $\IC$. We return again to
the notation introduced in Example~\ref{ex:thetafunc}.
\begin{itemize}
\item We have an immersion $A\rightarrow\IP^n$
\item We have a uniformizing map $u\colon T_0(A)\rightarrow
  A(\IC)$  with the tangent space
  $\mathrm{T}_0(A)$. The map $u$ is a holomorphic, surjective group
  homomorphism.
\item The kernel $\ker u$ is a discrete subgroup $\Omega\subset\IC^g$
  of rank $2g$, the period lattice. We fix a basis $(\omega_1,\ldots,\omega_{2g})$ of
  $\Omega$ and obtain a fundamental domain.
\end{itemize}

In this setup it has already proved useful to work in 2 distinct
coordinate systems in $T_0(A)$.

\begin{itemize}
\item \textbf{Complex coordinates:} We fix an isomorphism
  $\mathrm{T}_0(A)\rightarrow\IC^g$ of $\IC$-vector spaces. 
\item \textbf{Period coordinates:} We fix the isomorphism
  $\mathrm{T}_0(A)\rightarrow \IR^{2g}$ of $\IR$-vector spaces
  induced by the basis $(\omega_1,\ldots,\omega_{2g})$. 
\end{itemize}
\section{Further Reading and Open Problems}



%%% Local Variables:
%%% TeX-master: "main"
%%% End:
