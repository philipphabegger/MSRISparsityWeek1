\chapter{Functional Transcendence}

\section{Overview}

In this chapter we first state the Pila--Wilkie Theorem. This
important tool gives us information on the distribution of rational
points of bounded height on a definable set. We will see two
applications of this theorem. The first is towards some cases of the
Ax--Schanuel Theorem which is used to systematically treat situations
that arose in Section~\ref{subsub:functrans}. The second application
will come in a further chapter. 

\section{The Pila--Wilkie Theorem}


\begin{definition}
  Let $x\in \IQ$. There exist coprime integers $p,q$ with
  $q\ge 1$ and $x=p/q$. The (exponential) height of $x$ is $H(x)=\max\{|p|,q\}$.
  Let $(x_1,\ldots,x_m)\in\IQ^m$. % There exist coprime integers
  % $p_1,\ldots,p_m,q$ with $q\ge 1$ and $x_i = p_i/q$ for all $i\in
  % \{1,\ldots,m\}$.
  The exponential height of $(x_1,\ldots,x_m)$ is $H(x_1,\ldots,x_m)
  =\max \{H(x_1),\ldots,H(x_m)\}$. 
%  = \max_{1\le i\le m} \{|p_1|,\ldots,|p_m|,q\}$. 
\end{definition}

In Hector's lecture we will encouter heights that are normalized by
taking $\log H(\cdots)$ in the case $m=1$. In the context of the
Pila--Wilkie Theorem the exponential height is slightly more
practical.


\begin{example}\label{ex:heightcount1}\leavevmode
  \begin{enumerate}
  \item [(i)]
  One fundamental feature of the height is the Northcott property: for
  all $T\ge 1$ the set
  \begin{equation*}
    \{ x\in \IQ : H(x)\le T\}
  \end{equation*}
  is finite. In this statement we may replace $x\in\IQ$ by $x\in
  \IQ^m$.

  Let use define $a(T) = \#\{x\in\IQ : H(x)\le T\}$ which is
  non-decreasing in $T$. How quickly does $a(T)$ grow in $T$?
  If $T\ge 1$ is an integer, then $a(T) \ge 2T+1$ by considering
  integers
  $-T,-T+1,\ldots,0,\ldots,T-1,T$. But we have all rational numbers at
  our disposal and $H(x)=H(-x)=H(1/x)=H(-1/x)$ for all
  $x\in\IQ^\times$. So it suffices to count rational numbers in
  $(0,1)$ of height at most $T$, quadruple this number, and add 3
  (for $0,\pm 1$)
  ,\textit{i.e.}, 
  \begin{equation*}
    a(T) = 3 + 4 \sum_{q=1}^T \# \{ p : 1\le p \le q-1 \text{ and
    }\gcd(p,q)=1\} = -1+4\sum_{q=1}^T \varphi(q)
  \end{equation*}
  with $\varphi$ as in Section~\ref{sec:rootsof1}. By Lemma~\ref{lem:lgo}
  $a(T)$ grows at least as $T^{3/2}$. In fact, it is known that
  $a(T)\sim \frac{12}{\pi^2} T^2$ as $T\rightarrow\infty$.


\item[(ii)] Let $d\ge 1$ be an integer and consider
  $X = \{(x,x^d) : x\in \IR\}$. If $x\in\IQ$ then $H(x^d) = H(x)^d$.
  So
  \begin{equation*}
    \#\{ (x,y) \in X\cap \IQ^2 : H(x,y)\le T\} \sim
    \frac{12}{\pi^2}T^{2/d} 
  \end{equation*}
  as $T\rightarrow\infty$. 
  
  \item[(iii)] It follows from the Lindemann--Weierstrass Theorem that
    if $x\in\IQ^\times$, then $e^x \not\in\IQ$. So the graph
    $\{(x,e^x) : x\in\IR\}$ contains precisely $1$ rational point
    $(0,1)$.

  \item[(iv)] A transcendental graph can contain infinitely many
    rational points. Consider the graph of $x\mapsto 2^x$ on $\IR$, so 
    $X=\{(x,2^x) : x \in\IR\}$. This set is definable in $\IRexp$.
    If $t\ge 1$ is an integer and $x\in \{1,\ldots,t\}$, then
    $(x,2^x)$ is integral of height at most $T=2^t$. Therefore,
    \begin{equation*}
      \# \{(x,2^x)\in\IQ^2 : H(x,2^x)\le T\} \ge t= (\log T)/\log 2.
    \end{equation*}
    It is not difficult to prove an upper bound that is linear in
    $\log T$. 
\end{enumerate}
\end{example}

\begin{definition}
  Let $X\subset\IR^m$ be any subset and $T\ge 1$. The \emph{counting
    function} is 
  \begin{equation*}
    N(X,T) = \#\{ x\in X\cap \IQ^m : H(x)\le T\}. 
  \end{equation*}
\end{definition}

In our applications we will usually take $X$ to be a  set definable in
an o-minimal structure.

\begin{example}
  We revisit Example~\ref{ex:heightcount1} using the counting
  function.
  \begin{enumerate}
  \item [(i)] We have $N(\IR,T)\sim \frac{12}{\pi^2}T^2$ as
    $T\rightarrow\infty$.
  \item [(ii)] Let $d\ge 1$ be an integer and $X= \{(x,x^d):
    x\in\IR\}$. Then $N(X,T)\sim \frac{12}{\pi^2}T^{2/d}$ as
    $T\rightarrow\infty$.
  \item[(iii)] We have $N(\text{graph of }\exp,T) = 1$ for all $T\ge
    1$.
  \item [(iv)] We have $\log T \ll
    N(\text{graph of }x\mapsto 2^x,T)\ll \log T$ for $T\rightarrow\infty$.\footnote{$\ll$
      signifies that the left-hand side is bounded by a positive
      constant times the right-hand side.}
  \end{enumerate}
\end{example}


\section{Further Reading and Open Problems}

%%% Local Variables:
%%% TeX-master: "main"
%%% End:
