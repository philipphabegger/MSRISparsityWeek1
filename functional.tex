\chapter{The Pila--Wilkie Theorem and Functional Transcendence}

\section{Overview}

In this chapter we first state the Pila--Wilkie Theorem. This
important tool gives us information on the distribution of rational
points of bounded height on a definable set. We will see two
applications of this theorem. The first is towards some cases of the
Ax--Schanuel Theorem which is used to systematically treat situations
that arose in Section~\ref{subsub:functrans}. The second application
will come in a further chapter. 

\section{The Pila--Wilkie Theorem}

\begin{definition}
  \label{def:height1}
  \begin{enumerate}
  \item [(i)]
  Let $x\in \IQ$. There exist coprime integers $p,q$ with
  $q\ge 1$ and $x=p/q$. The \emph{exponential height}, or just
  \emph{height},  of $x$ is $H(x)=\max\{|p|,q\}$.
  \item[(ii)] Let $(x_1,\ldots,x_m)\in\IQ^m$. There exist coprime integers
  $p_1,\ldots,p_m,q$ with $q\ge 1$ and $x_i = p_i/q$ for all $i\in
  \{1,\ldots,m\}$.
  The \emph{exponential height}, or just \emph{height},
  of $(x_1,\ldots,x_m)$ is $H(x_1,\ldots,x_m)
  = \max_{1\le i\le m} \{|p_1|,\ldots,|p_m|,q\}$.
  \end{enumerate}
\end{definition}

In Hector's lecture we will encouter heights that are normalized by
taking $\log H(\cdots)$. In the context of the
Pila--Wilkie Theorem the exponential height is slightly more
practical.

\begin{exercise}
  Show that $H(xy)\le H(x)H(y)$ and $H(x+y)\le 2H(x)H(y)$ for all
  $x,y\in\IQ$. 
\end{exercise}

\begin{example}\label{ex:heightcount1}
  \begin{enumerate}
  \item [(i)]
  One fundamental feature of the height is the Northcott property: for
  all $T\ge 1$ the set
  \begin{equation*}
    \{ x\in \IQ : H(x)\le T\}
  \end{equation*}
  is finite. In this statement we may replace $x\in\IQ$ by $x\in
  \IQ^m$.

  Let use define $N(T) = \#\{x\in\IQ : H(x)\le T\}$ which is
  non-decreasing in $T$. How quickly does $N(T)$ grow in $T$?
  If $T\ge 1$ is an integer, then $N(T) \ge 2T+1$ by considering
  integers
  $-T,-T+1,\ldots,0,\ldots,T-1,T$. But we have all rational numbers at
  our disposal and $H(x)=H(-x)=H(1/x)=H(-1/x)$ for all
  $x\in\IQ^\times$. So it suffices to count rational numbers in
  $(0,1)$ of height at most $T$, quadruple this number, and add 3
  (for $0,\pm 1$)
  ,\textit{i.e.}, 
  \begin{equation*}
    N(T) = 3 + 4 \sum_{q=1}^T \# \{ p : 1\le p \le q-1 \text{ and
    }\gcd(p,q)=1\} = -1+4\sum_{q=1}^T \varphi(q)
  \end{equation*}
  with $\varphi$ as in Section~\ref{sec:rootsof1}. By Lemma~\ref{lem:lgo}
  $N(T)$ grows at least as $T^{3/2}$. In fact, it is known that
  $N(T)\sim \frac{12}{\pi^2} T^2$ as $T\rightarrow\infty$.


\item[(ii)] Let $d\ge 1$ be an integer and consider
  $X = \{(x,x^d) : x\in \IR\}$. If $x\in\IQ$ then $H(x,x^d) = H(x)^d$.
  So
  \begin{equation*}
    \#\{ (x,y) \in X\cap \IQ^2 : H(x,y)\le T\} \sim
    \frac{12}{\pi^2}T^{2/d} 
  \end{equation*}
  as $T\rightarrow\infty$. 
  
  \item[(iii)] It follows from the Lindemann--Weierstrass Theorem that
    if $x\in\IQ^\times$, then $\exp(x) \not\in\IQ$. So the graph
    $\{(x,\exp(x)) : x\in\IR\}$ contains precisely $1$ rational point
    $(0,1)$.

  \item[(iv)] A transcendental graph can contain infinitely many
    rational points. Consider the graph of $x\mapsto 2^x$ on $\IR$, so 
    $X=\{(x,2^x) : x \in\IR\}$. This set is definable in $\IRexp$.
    If $t\ge 1$ is an integer and $x\in \{1,\ldots,t\}$, then
    $(x,2^x)$ is integral of height at most $T=2^t$. Therefore,
    \begin{equation*}
      \# \{(x,2^x)\in\IQ^2 : H(x,2^x)\le T\} \ge t= (\log T)/\log 2.
    \end{equation*}
    It is not difficult to prove an upper bound that is linear in
    $\log T$. 
\end{enumerate}
\end{example}

\begin{definition}
  \label{def:countingfunc}
  Let $X\subset\IR^m$ be any subset and $T\ge 1$. The \emph{counting
    function} is 
  \begin{equation*}
    N(X,T) = \#\{ x\in X\cap \IQ^m : H(x)\le T\}. 
  \end{equation*}
\end{definition}

In our applications we will usually take $X$ to be a  set definable in
an o-minimal structure.

\begin{example}
  \label{ex:heightcount2}
  We revisit Example~\ref{ex:heightcount1} using the counting
  function.
  \begin{enumerate}
  \item [(i)] We have $N(\IR,T)\sim \frac{12}{\pi^2}T^2$ as
    $T\rightarrow\infty$.
  \item [(ii)] Let $d\ge 1$ be an integer and $X= \{(x,x^d):
    x\in\IR\}$. Then $N(X,T)\sim \frac{12}{\pi^2}T^{2/d}$ as
    $T\rightarrow\infty$.
  \item[(iii)] We have $N(\text{graph of }\exp,T) = 1$ for all $T\ge
    1$.
  \item [(iv)] We have $\log T \ll
    N(\text{graph of }x\mapsto 2^x,T)\ll \log T$ for $T\rightarrow\infty$.\footnote{$\ll$
      signifies that the left-hand side is bounded by a positive
      constant times the right-hand side.}
  \end{enumerate}
\end{example}

\begin{exercise}
  Let $C\subset [0,1]$ be the Cantor set. What can we say about the
  growth of $N(C,T)$?\footnote{I don't know.}
\end{exercise}

The Pila--Wilkie Theorem states that on a definable set, an abundance of
rational points implies the existence of a real-semi algebraic curve.

\begin{definition}
  Let $X\subset\IR^m$ be any subset. We define
  \begin{equation*}
    X^{\mathrm{alg}} = \bigcup_{\substack{\text{connected, real
          semi-algebraic}\\ \text{curve }C\subset X}} C.
  \end{equation*}
\end{definition}

\begin{example}
  We revisit the four sets from Examples~\ref{ex:heightcount1} and
  \ref{ex:heightcount2}.
  \begin{enumerate}
  \item [(i)] If $X=\IR$ then $X^{\mathrm{alg}}=\IR$.
  \item[(ii)] Let $d\ge 1$ be an integer. If
    $X\subset\IR^2$ is the graph of $x\mapsto x^d$, then
    $X^{\mathrm{alg}}=X$.
  \item[(iii)] If $X$ is the graph of $\exp\colon\IR\rightarrow\IR$, we
    claim that $X^{\mathrm{alg}}=\emptyset$. Indeed, if
    $X^{\mathrm{alg}}$ were non-empty, it would contain a connected
    real semi-algebraic curve. Thus the components of $(x,\exp(x))$
    would be algebraically dependent for $x$ in some non-empty open
    interval and thus on all of $\IR$. But then there would exist
    $A\in \IR[X,Y]\ssm\{0\}$ with $A(x,\exp(x))=0$ for all $x\in\IR$.
    This can easily be ruled out as $\exp$
    grows quicker than any polynomial. 
    
  \item[(iv)] If $X$ is the graph of $x\mapsto 2^x$, then
    $X^{\mathrm{alg}}=\emptyset$.    
  \end{enumerate}
  Here are some further examples:
  \begin{enumerate}
  \item [(v)] Let $X = \{(y,x,y^x) : y\in (0,\infty), x\in\IR\}$. Then
    $X$ is definable in $\IRanexp$ and $\dim X = 2$.
    Observe that $X$ contains many real semi-algebraic curves,
    \textit{e.g.}, $\{(y,x,y^x) : y\in (0,\infty)\}$ for all
    $x\in\IQ$.
    One can show that $X^{\mathrm{alg}}$ is not definable in
    $\IRanexp$.

  \item[(vi)] Let $C = \{(x,y) : 0<x<1 \text{ and } 0<y<\exp(x) \}$.
    This is $(1,1)$-cell in the o-minimal structure $\IRan$ (and
    $\IRanexp$). It is not definable in $\IRexp$ as the top boundary
    is a portion of the graph of the exponential function. However,
    $C^{\mathrm{alg}}=C$ since $C$ is open subset of $\IR^2$. 
  \end{enumerate}
\end{example}

% \begin{exercise}
%   Find a subset $X\subset\IR^2$ that
%   is \emph{not} real semi-algebraic with $X^{\mathrm{alg}}=X$.   
% \end{exercise}

\begin{theorem}[Pila--Wilkie~\cite{PilaWilkie}]
  \label{thm:pilawilkie}
  Let $\cS$ be an o-minimal structure,
  let $\epsilon>0$, and let $X\subset\IR^m$ be
  definable in $\cS$.
  There exists  $c(\epsilon,X)>0$ such that
  \begin{equation*}
    N(X\ssm X^{\mathrm{alg}},T)\le c(\epsilon,X) T^\epsilon \quad\text{for
      all}\quad T\ge 1.
  \end{equation*}
\end{theorem}

So rational points can at most accumulate ``excessively'' on the
algebraic locus of a definable set. In Hector's lecture we will see a
conceptually related ``exceptional'' subset of a projective variety in
connection with Lang's Conjecture.

The proof of the Pila--Wilkie Theorem automatically produces the
following version that is uniform in families.

\begin{theorem}[Pila--Wilkie~\cite{PilaWilkie}]
  Let $\cS$ be an o-minimal structure,  let $X\subset\IR^n\times \IR^m$ be
  definable in $\cS$, and let $\epsilon >0$.
  There exists  $c(X,\epsilon)>0$ such that
  \begin{equation*}
    N(X_y\ssm X_y^{\mathrm{alg}},T)\le c(X,\epsilon) T^\epsilon \quad\text{for
      all}\quad T\ge 1\text{ and all }y\in\IR^n. 
  \end{equation*}
\end{theorem}

Let us finish this section with a historical remark on the
\textit{Determinant Method}. This is the ``arithmetic core'' of the
proof. The method was introduced by Bombieri and
Pila~\cite{BombieriPila} at a time when the connection between
o-minimal geometry and diophantine approximation had not been made.
So their results were confined to low dimension (essentially $\le 2$).

\begin{theorem}[Bombieri--Pila, Theorem~1~\cite{BombieriPila}]
  Let $I = [a,b]$ with real numbers $a<b$ an d suppose $f\colon
  I\rightarrow \IR$ is the restriction of a real analytic function to
  $I$. Suppose that $f$ is no algebraic and let $\Gamma(f)\subset
  I\times \IR$ denote the graph of $f$. For all $\epsilon > 0$ there
  exists $c(f,\epsilon)> 0$ with
  \begin{equation*}
    \# (\Gamma \cap T^{-1}\IZ^2) \le c(f,\epsilon)T^\epsilon
    \quad\text{for all}\quad T\ge 1. 
  \end{equation*}
\end{theorem}

This result can after earlier work of Jarn\'ik and Swinnerton-Dyer,
\cite{PilaWilkie} contains a brief historical account. 

\begin{exercise}
  \begin{enumerate}
  \item [(i)]  Deduce the Bombieri--Pila Theorem from the Pila--Wilkie Theorem.
    What o-minimal structure do you need? 
  \item[(ii)] Read the proof of the Bombieri--Pila Theorem.
  \item[(iii)] Show that in general, the growth rate
    $c(X,\epsilon)T^\epsilon$ in Theorem~\ref{thm:pilawilkie}
    cannot be improved in the following sense.
    Let $\epsilon \colon [1,\infty)\rightarrow (0,\infty)$ be strictly
    decreasing with $\lim_{T\rightarrow\infty}\epsilon(T)=0$. 
    Find a function $f\colon [0,1]\rightarrow \IR$ that is the
    restriction of a real analytic function defined on an open
    neighborhood of $[0,1]$ such that
    $$
    N(\Gamma,T) \ge T^{\epsilon(T)}
    $$
    for infinitely many $T\in\IN$; here $\Gamma$ is the graph of $f$.
    (Hint: See Example 7.5~\cite{Pila:Subanalytic} or \cite{Surroca:IMRN}). 
  \end{enumerate}
\end{exercise}

In dimension 2, Pila~\cite{Pila:Subanalytic}, was able prove
results towards the Pila--Wilkie Theorem before the introduction of
o-minimal geometry. 

\section{Schanuel's Conjecture and Ax's Theorem}

Schanuel's Conjecture is major open problem in transcendence theory.

\begin{conjecture}[Schanuel]
  Let $z_1,\ldots,z_m\in\IC$ be $\IQ$-linearly independent. Then
  \begin{equation*}
    \mathrm{trdeg}(\IQ(z_1,\ldots,z_m,\exp(z_1),\ldots,\exp(z_m)/\IQ) \ge
    m. 
  \end{equation*}
\end{conjecture}

\begin{example} What is known towards Schanuel's Conjecture?
  \begin{enumerate}
  \item [(i)]
  If $m=1$ then the condition on $z=z_1$ is $z\not=0$ and the
  conjecture predicts $\mathrm{trdeg}(z,\exp(z))\ge 1$.
  For $m=1$ Schanuel's Conjecture is equivalent to the statement that
  \begin{equation*}
    \text{if}\quad z\in\IQbar^\times \quad\text{then}\quad \exp(z)\in \IC\ssm \IQbar.
  \end{equation*}
  This case is known, it is Lindemann's Theorem.
  \item[(ii)] Say $m\ge 1$ and suppose $z_1,\ldots,z_m\in\IQbar$ are
    $\IQ$-linearly independent. Then the Lindemann--Weierstrass
    Theorem states that $\exp(z_1),\ldots,\exp(z_m)$ are algebraically
    independent. So Schanuel's Conjecture is known for algebraic $z_1,\ldots,z_m$.
  \item[(iii)] Say $m\ge 1$ and suppose $\exp(z_1),\ldots,\exp(z_m)$
    are algebraic and that $z_1,\ldots,z_m$ are $\IQ$-linearly
    independent. Then Baker's Theorem states that $z_1,\ldots,z_m$ are
    $\IQbar$-linearly dependent. 
\end{enumerate}
\end{example}

\begin{exercise}
  \begin{enumerate}
  \item [(i)]  Show that Schanuel's Conjecture implies that the iterated
    exponential $\exp^N(1)$ is
    transcendental for all $N\ge 1$.
  \item[(ii)] Show that Schanuel's Conjecture implies that
    $\pi^{\pi^{\pi^\pi}}$ is transcendental.
  \end{enumerate}
\end{exercise}

Schanuel's Conjecture seems well out of reach at the moment.
But Ax showed that the functional version is true. 

\begin{theorem}[Ax,~Theorem~3~\cite{AxSchanuel}]
  \label{thm:Axmult}
  Let $\Delta\subset\IC$ be a domain and suppose
  $f_1,\ldots,f_m \colon\Delta \rightarrow\IC$ are holomorphic
  functions that are $\IQ$-linearly independent \textbf{modulo
    constants}. Then
  \begin{equation*}
    \mathrm{trdeg}(\IC(f_1,\ldots,f_m,\exp(f_1),\ldots,\exp(f_m))/\IC) \ge
    m+1. 
  \end{equation*}  
\end{theorem}

The following corollary will be of importance to us. We will prove it
assuming Theorem~\ref{thm:Axmult}. 

\begin{corollary}[Ax--Lindemann--Weierstrass]
  \label{cor:alw}
  Let $C\subset\IC^m$ be a real semi-algebraic curve.
  After replacing $C$ by a non-empty open subset of itself
 $\exp(C)=\{\exp(z) :z\in
  C\}$\footnote{Here $\exp(z) = (\exp(z_1),\ldots,\exp(z_m))$ for
    $z=(z_1,\ldots,z_m)\in\IC^m$.} is either Zariski dense in
  $(\IC^\times)^m$ or contained in
  the
  translate of a proper algebraic subgroup of $\IG_{\mathrm{m}}^m$.   
\end{corollary}
\begin{proof}  
  Locally at some point of $C$ we can parametrize using a
  semi-algebraic function $f=(f_1,\ldots,f_m)\colon (-1,1)\rightarrow C$. We may assume
  that $f$ is real
  analytic. Considering the Taylor series we can extend $f$ to a
  non-constant  holomorphic map $\Delta\rightarrow\IC^m$. The
  transcendence degree of
  $\IR({f_1}|_{(-1,1)},\ldots,{f_m}|_{(-1,1)})/\IR$ is $1$ and this
  remains true for $\IC(f_1,\ldots,f_m)/\IC$ by the Identity Theorem.
  Let $V$ denote the Zariski closure of $\exp(C)$ in
  $\IG_{\mathrm{m}}^m$. It is irreducible. 
  Then $(\exp(f_1),\ldots,\exp(f_m))$ takes values in $V(\IC)$ by
  the Identity Theorem.
  Therefore,
  \begin{alignat*}1
    &\mathrm{trdeg}\,\IC(f_1,\ldots,f_m,\exp(f_1),\ldots,\exp(f_m))/\IC
    \\
    &\quad \le
    \mathrm{trdeg}\,\IC(f_1,\ldots,f_m)/\IC+\mathrm{trdeg}\,\IC(\exp(f_1),\ldots,\exp(f_m))/\IC\\
    &\quad \le
    1+\dim V.
  \end{alignat*}

  If $\dim V=m$ then there is nothing to show. Otherwise we have $\dim
  V \le m-1$ and 
  Ax's Theorem implies that there exist
  $(a_1,\ldots,a_m)\in\IZ^m\ssm\{0\}$ and $c\in\IC$ such that
  $a_1f_1+\cdots+a_mf_m=c$ identically. It follows that the
  multiplicative relation
  $X_1^{a_1}\cdots X_m^{a_m} = \exp(c)$ holds identically on
  $\exp(C)$. So it also holds on the Zariski closure $V$ of $\exp(C)$. 
\end{proof}

\begin{exercise}
  Study the classification of algebraic subgroups of $\IG_{\mathrm{m}}^m$
  and prove the following strengthening of Corollary~\ref{cor:alw2}.
  See Chapters 3.1 and 3.2~\cite{BG} for background on algebraic
  subgroups of $\IG_{\mathrm{m}}^m$. 
\end{exercise}


The following corollary can be proved using Corollary~\ref{cor:alw}
in combination with an induction argument. 

\begin{corollary}
  \label{cor:alw2}
  Let $C$ be as in Corollary~\ref{cor:alw}. After replacing $C$ by a
  non-empty open subset of itself, the Zariski closure of $\exp(C)$ is
  the translate of a proper algebraic subgroup of $\IG_{\mathrm{m}}^m$. 
\end{corollary}

\section{The Ax--Lindemann--Weierstrass Theorem for Abelian Varieties}
\label{sec:alwav}

Ax's Theorem~\ref{thm:Axmult} has an analog for abelian varieties,
also proved by Ax.

In this section we will see a proof of the Ax--Lindemann--Weierstrass
Theorem, Corollaries~\ref{cor:alw} and \ref{cor:alw2}, using the
Pila--Wilkie Theorem. The Pila--Wilkie Theorem was first used in this
area in the influential paper~\cite{PilaZannier} by Pila and Zannier.
There, the authors gave a new proof of the Manin--Mumford Conjecture.
Our approach towards the Ax--Lindemann--Weierstrass theorem here is
influenced by work of Orr, Pila, Ullmo, and Yafaev. 
% While Ax's original proof predates the work of Pila and Zannier, the
% latter paved the way to further developments that had many
% applications.

We will use the setup introduced in Example~\ref{ex:thetafunc}. So $A$
is a $g$-dimensional abelian variety defined over $\IC$ and embedded
in projective space $\IP^n$. Moreover, we have a uniformizing map
$u\colon \IC^g\rightarrow A(\IC)$ defined theta functions
$\vartheta_0,\ldots,\vartheta_n$. Recall that $u$ is a surjective
complex analytic group homomorphism with kernel $\Omega$. Let
$(\omega_1,\ldots,\omega_{2g})$ denote a fixed basis of $\Omega$ and
let $\cF$ denote the fundamental domain (\ref{eq:fdav}).
We recall Convention~\ref{conv:identCCRR2}. 
Recall that $\cF$ is real semi-algebraic and that the pre-image under
$u|_{\cF}$ of an algebraic subvariety of $A$ is definable in $\IRan$.


\begin{definition}
  Let $A$ be an abelian variety. A \emph{coset} of $A$ is the translate of an
  abelian subvariety of $A$. A \emph{torsion coset} of $A$ is a coset
  that contains a point of finite order. Equivalently a torsion coset
  of $A$ is the translate of an abelian subvariety by a point of
  finite order. 
\end{definition}


\begin{theorem}[Ax--Lindemann--Weierstrass for abelian varieties]
  \label{thm:alwav}
  Let $C\subset\IC^g$ be a real semi-algebraic curve. After replacing
  $C$ by a non-empty open subset, the Zariski
  closure of $u(C)$ in $A$ is a coset of $A$.
\end{theorem}

\subsubsection{Step 1: Complexifying a suitable open subset of $C$.}

Let $z_1,\ldots,z_g$ be the complex coordinates of $\IC^g$ with real
parts $x_1,\ldots,x_g$ and imaginary parts $y_1,\ldots,y_g$. By the
Cell Decomposition Theorem applied to $C$, which is definable in
$\IRalg$, there is an open subset $U$ of $C$ that is homeomorphic to
$(0,1)$. One can arrange this homeomorphism to be real semi-algebraic
and real analytic. All this essentially amounts to finding a smooth
point of $C$. We consider $x_1|_{U},\ldots,{x_g}|_U,{y_1}|_U,\ldots
y_g|_{U}$ as restricted coordinate functions; the $\IR$-algebra they
generate is an integral domain by the Identity Theorem for real
analytic functions. Then
$\mathrm{trdeg}\,\IR({x_1}|_U,\ldots,{y_g}|_U)/\IR=1$ and so
$\mathrm{trdeg}\,\IC({z_1}|_U,\ldots,{z_g}|_U)/\IC=1$. Therefore, $U$
lies in a complex algebraic curve $C_{\IC}\subset\IC^g$. We can
arrange that $C_\IC$ is irreducible.

Let $V$ denote the Zariski closure of $u(C_\IC)$ in $A$. Let $Z$
denote a maximal irreducible algebraic subset of $\IC^g$ that contains $C_{\IC}$
and such that $Z(\IC)\subset u^{-1}(V(\IC))$. To
simplify the notation we identify $Z$ with $Z(\IC)$ and $V$ with
$V(\IC)$.


\subsubsection{Step 2: Applying the Pila--Wilkie Theorem}

\begin{definition}
  \label{def:stabZ}
  The \emph{stabilizer} of $Z$ is the subgroup
  \begin{equation*}
    \mathrm{Stab}(Z) = \{v\in \IC^g : v+Z=Z\}
  \end{equation*}
  of $\IC^g$. 
\end{definition}

The structure of the subgroup $\mathrm{Stab}(Z)$ is well-behaved by
the following exercise. We will not need it this result in our proof
of Theorem~\ref{thm:alwav}.

\begin{exercise}
  Show that the group $\mathrm{Stab}(Z)$ is a vector subspace of
  $\IC^g$. 
\end{exercise}

\begin{lemma}
  The stabilizer $\mathrm{Stab}(Z)$ is infinite. 
\end{lemma}

At this stage we also do an $\IR$-linear change of coordinates and
replace $\IC^g$ with $\IR^{2g}$ with the help of our period lattice
basis $(\omega_1,\ldots,\omega_{2g})$.
The period lattice $\Omega$ becomes $\IZ^{2g}$ and $\cF$ becomes
$[0,1]^{2g}$. To avoid double counting it is useful to work with
$[0,1)^{2g}$ which is also real semi-linear. 

We now write down a set that is definable in $\IRan$. Recall that
$u|_{[0,1]^{2g}}^{-1}(V)$ is definable in $\IRan$. We set
\begin{alignat*}1  
  X =  \{x \in\IR^{2g} : & Z\cap(x+[0,1)^{2g})\text{ contains a neighborhood
    of $Z$}\\
  &\text{and } (Z-x)\cap [0,1)^{2g} \subset u|_{[0,1]^{2g}}^{-1}(V)\}. 
\end{alignat*}

Let $\omega \in\IZ^{2g}$ such that $\dim Z\cap (\omega+[0,1)^{2g})$
contains a non-empty open subset of $Z$. We claim that $\omega\in X$.
Indeed, say $y\in Z$ with $y\in \omega+[0,1)^{2g}$. Then $y\in
u^{-1}(V)$, using $u(Z)\subset V$, and hence $y \in \omega+u^{-1}(V)$
as $u^{-1}(V)$ is periodic under the period lattice. So $y \in
\omega+u|_{[0,1]^{2g}}^{-1}(V)$ for all $y$ with $y\in Z\cap
(\omega+[0,1)^{2g})$.

We have showed
$$\{\omega\in\IZ^{2g} :    Z\cap (\omega+[0,1)^{2g})\text{ contains a non-empty open subset of $Z$}\} \subset X.$$ The important observation is that
$Z$ meets sufficiently many translates of $[0,1)^{2g}$ in top dimension. To see
this 
we briefly  return to the original complex
coordinates and remember that $Z$ is a complex
algebraic curve in $\IC^g$. The fundamental fact is that a complex
algebraic variety in $\IC^g$ is either finite or not compact.

\begin{exercise}
  Let $Z$ be an infinite algebraic subset of $\IC^g$. Show that $Z$ is
  not compact. 
\end{exercise}

As $Z$ is irreducible, its regular locus $Z^{\mathrm{reg}}$ is
path-connected. 
So there exists  a continuous function $\gamma : [0,\infty)\rightarrow
Z^{\mathrm{reg}}$ such that $\lim_{t\rightarrow\infty} \|\gamma(t)\| = \infty$
where $\|\cdot\|$ is the hermitian norm on $\IC^g$. 
For all $t\in [0,\infty)$ the point $\gamma(t)$ is contained in
$\omega(t)+[0,1)^{2g}$ for a unique period $\omega(t)\in\IZ^{2g}$.
In the unlucky event
that $(\omega(t)+[0,1)^{2g}) \cap Z$ has empty interior
in $Z$, then $\gamma(t)$ must be on the boundary of $[0,1)^{2g}$ and
$(\omega(t)+\omega')\cap Z$ has non-empty interior in $Z$
for some $\omega' \in \{-1,0\}^{2g}$. In this case we replace
$\omega(t)$ by $\omega(t)+\omega'$. 
By
what we showed above $\omega(t)\in X\cap\IZ^{2g}$.
Recall that $H(\omega(t))$ is
just the maximal entry of $\omega(t)$ in absolute value.
Each time
$\gamma$ enters a new translate of $[0,1)^{2g}$ the height
$H(\omega(t))$ increases by a bounded amount.
We see that $N(X,T)$ grows proportionally to the number of translates of
$[0,1)^{2g}$ that $\gamma$ passes through. The image of $t\mapsto \omega(t)$ must be infinite as $\gamma$ is unbounded. 
So we conclude that
\begin{equation*}
  N(X,T) \ge \# \{\omega(t) : H(\omega(t)) \le T\}  \gg T\quad\text{as}\quad T\rightarrow\infty.
\end{equation*}

By the Pila--Wilkie Theorem, with $\epsilon=1/2$ say, we conclude
$$
N(X\ssm X^{\mathrm{alg}})  \le c(X) T^{1/2}.
$$
So $X$ contains a connected, real semi-algebraic curve $R$ that
contains some $\omega(t)\in\IZ^{2g}$.

% In the Pila--Wilkie Theorem we can arrange for $R$ to contain some
% $\omega\in \IZ^{2g}$.

Then $Z\subset Z-R+\omega$. If $x\in R$ then $Z-x\subset u^{-1}(V)$ as
both sides are complex analytic and since $u^{-1}(V)$ contains a
non-empty open subset of $Z-x$. Since $u^{-1}(V)$ is periodic under
the period lattice we find $Z-x+\omega\subset u^{-1}(V)$ for all $x\in
R$. Therefore,
\begin{equation*} Z\subset Z-R+\omega \subset u^{-1}(V).
\end{equation*}

We can complexify $R$ just as above to obtain an irreducible
complex algebraic $R_\IC$ variety containing finitely many points of
$R$ and with $Z-R_{\IC} \subset u^{-1}(V)$.
But recall that $Z$ was a maximal irreducible algebraic subset of
$u^{-1}(V)$. We conclude $Z = Z - R_{\IC}+\omega$.


\subsubsection{Completion of the Proof of Theorem~\ref{thm:alwav}}

Under the hypothesis of Theorem~\ref{thm:alwav} we have showed that
there exists an irreducible subvariety $Z\subset\IC^g$ that contains a
non-empty open subset of $C$ such that the group $\mathrm{Stab}(Z)$ is infinite.

% \begin{lemma}
%   The group $\mathrm{Stab}(Z)$ is a vector subspace of $\IC^g$. 
% \end{lemma}
% \begin{proof}
%   The stabilizer can be rewritten as
%   $$G=\mathrm{Stab}(Z) = \bigcap_{z\in Z}(Z-z).$$

%   As $Z$ is Zariski closed in $\IC^g$ we see that $G$
%   is Zariski closed.

%   Suppose $f\in\IC[X_1,\ldots,X_n]$ lies in the ideal of
%   $G$. Let us decompose $f=\sum_{j=0}^d f_j$ where
%   $f_j\in\IC[X_1,\ldots,X_n]$ is homogenous of degree $j$.
%   Say $v\in G$. For all $l\in\IZ$ we have $lv \in G$ and so
%   $0=f(lv) = \sum_{j=0}^d l^j f_j(v)$ for all $l\in\IZ$.
%   It follows that $f_0(v)=\cdots=f_d(v)=0$.
  
%   So the ideal of $G$ is generated by homogeneous polynomials. In
%   particular, $v\in G$ and $\lambda\in \IC$ imply $\lambda v\in G$.   
% \end{proof}

Recall that $u(Z)$ lies Zariski dense in the irreducible $V$. We must
verify that $V$ is a coset of $A$. %the translate of an abelian subvariety of $A$.

We mimic here Definition~\ref{def:stabZ} for $A$. 
\begin{definition}
  The stabilizer of $V\subset A$ is
  \begin{equation*}
    \mathrm{Stab}(V) = \{P\in A(\IC) : P+V=V\}. 
  \end{equation*}
  It is a subgroup of $A(\IC)$. We let $\mathrm{Stab}^\circ(V)$ denote
  the connected component of $\mathrm{Stab}(V)$ containing $0$. 
\end{definition}


\begin{lemma}
  The stabilizer $\mathrm{Stab}(V)$ is an algebraic subgroup of $A$ of
  positive dimension.
\end{lemma}
\begin{proof}
  The stabilizer equals
  \begin{equation*}
    \mathrm{Stab}(V) = \bigcup_{P \in V} (V-P).
  \end{equation*}
  In this presentation we see that it is Zariski closed in $A$.
  
  Let $z\in Z$ and $v\in G$ be arbitrary. Then $z+v\subset Z$,
  so $u(z) \subset V-u(v)$. We find $V\subset V-u(v)$ and hence
  $u(v)+V=V$ by considering the dimension.
  It follows that 
    $u(G) \subset\mathrm{Stab}(V)$.  
\end{proof}

\begin{proof}[End of proof of Theorem~\ref{thm:alwav}]
  The connected component $B = \mathrm{Stab}^\circ(V)$ is an abelian
  subvariety of $A$ of positive dimension.
  If $A=B$ then $V=A$ and we are done. 
  Consider the quotient homomorphism $q\colon A\rightarrow A/B$.

  The fibers of $q|_V$ have dimension $\dim B$. So $\dim q(V) = \dim
  V-\dim B$. The $q$ induces a $\IC$-linear map between the tangent
  spaces $\mathrm{T}_0(A)\rightarrow\mathrm{T}_0(B)$. The image of $C$
  under this map is a definable and semi-algebraic set in
  $\mathrm{T}_0(B)$.

  If the dimension of the image of $U\subset C$ is $1$, then by induction on
  $\dim A$ the image $q(V)$ is a coset
  of $A/B$. As fibers of $q|_V$ are  cosets
  of $A$ we conclude that $V$ is a coset of $A$. This completes the proof in the current
  case.

  Second, assume that the image of $U\subset C$ is finite. Then it is
  a point. After translating we may arrange that $C$ is contained
  $\mathrm{T}_0(B)$. But we are already assuming $\dim B \le \dim
  A-1$. 
  So we proceed by induction on $\dim A$ and conclude that $V$ is a
  coset of $B$ and thus also of $A$. This
  conclude the proof in the second case.
\end{proof}

\begin{exercise}
  Formulate an Ax--Lindemann--Weierstrass Theorem for
  $\IG_{\mathrm{m}}^m$. Prove this using the method in Theorem~\ref{thm:alwav}.  
\end{exercise}

\section{Some consequences}
\label{sec:conclusionft}

We keep the notation of Example~\ref{ex:thetafunc}. 
Let $A$ be an abelian variety defined over $\IC$ of dimension $g$. Suppose it is
immersed in $\IP^n$ and let $u\colon \IC^{g}\rightarrow A(\IC)$ be a
uniformizing map. Finally, fix a basis of $\Omega = \ker u$ and a
fundamental domain $\cF \subset\IC^g$.

We can now determine the structure of the algebraic locus of the
preimage of a subvariety under the restricted uniformizing map. 

\begin{definition}
  \label{def:ueno}
  In the setup above we define
  \begin{equation*}
    V^{\circ} = \bigcup_{\substack{\text{coset $K$ of $A$} \\\text{with $\dim
          K\ge 1$ and $K\subset
          V$}}} K(\IC).
  \end{equation*}
  Then $V^{\circ}$ is called the \emph{Ueno locus} of $V$. 
\end{definition}

\begin{theorem}
  \label{thm:imagealg}
  Let $V\subset A$ be an irreducible subvariety and set
  $X = u|_{\cF}^{-1}(V(\IC))$. Then $u(X^{\mathrm{alg}}) = V^{\circ}$.
\end{theorem}
\begin{proof}
  Each point of $X^{\mathrm{alg}}$ is contained in a connected real
  semi-algebraic curve contained completely in
  $u|_{\cF}^{-1}(V(\IC))$. So $u(X^{\mathrm{alg}}) \subset V^{\circ}$
  follows from the
  Ax--Lindemann--Weierstrass Theorem for $A$, Theorem~\ref{thm:alwav}.

  Conversely, if $P\in V^{\circ}$, then there exists an abelian
  subvariety $B\subset A$ with $P+B \subset V$ and $\dim B \ge 1$. The
  tangent space $\mathrm{T}_0(B)$ is a vector subspace of
  $\mathrm{T}_0(A)$ of positive dimension. So it is contains a real
  semi-algebraic curve passing through the origin. This yields
  $P\in u(X^{\mathrm{alg}})$. 
\end{proof}


\begin{definition}
  Let $V$ be an irreducible subvariety of $A$. A coset $K$ of $A$ is
  called a \emph{maximal coset} of $V$ if $K\subset V$ and if it is
  a maximal coset of $A$ with respect to inclusion that is contained
  in $V$. 
\end{definition}

Let $K_1,\ldots,K_r$ be cosets of $A$ contained in $V$ with
$K_1\subsetneq \cdots\subsetneq K_r$. Then $r-1\le \dim K_r\le V$. So
any coset of $A$ that is contained in $V$ is contained in a maximal
coset of $V$.

\begin{theorem}
  \label{thm:ueno}
  Let $V$ be an irreducible subvariety of $A$.
  \begin{enumerate}
  \item [(i)] There is a finite set $\cB$ of abelian subvarieties of
    $A$ such that any maximal coset of $V$ is the translate of a
    member of  $\cB$.
  \item[(ii)] The Ueno locus $V^{\circ}$ is Zariski closed in $V$.
  \item[(iii)] We have $V^{\circ} = V\Longleftrightarrow \dim\mathrm{Stab}(V)\ge
    1$. 
  \end{enumerate}
\end{theorem}
\begin{proof}
  We sketch the proof for (i) by following the argument  of Lemma
  2~\cite{BombieriZannier:algptsav}. 

  Let $P\in V(\IC)$ and let $B$ be an abelian subvariety of $A$ such
  that $P+B$ is a maximal coset of $V$.

  Let $V_0=V$
  We inductively construct $r\in\IN_0$ and a nested sequence of irreducible subvarieties
  $V=V_0\supsetneq  V_1\supsetneq \cdots \supsetneq V_r$ such that
  \begin{itemize}
  \item [(i)] we have $P+B \subset V_i$ for all $i$ and
  \item[(ii)] for each $i\in \{0,\ldots,r-1\}$ there exists $b=b_i\in B(\IC)$
    such that $V_{i+1}$ is an irreducible component of $V_{i}\cap (V_{i}-b)$.
  \end{itemize}
  Moreover, $B = \mathrm{Stab}(V_r)$. 

  Clearly, $V=V_0$ already satisfies (i) and (ii). Suppose $i\ge 0$
  and 
  $V_0,\ldots,V_{i}$ have already been constructed.

  First, assume  $b+V_{i} = V_{i}$ for all $b\in B(\IC)$. Then $B\subset
  \mathrm{Stab}(V_{i})$. In particular,
  $P+B\subset P+\mathrm{Stab}(V_{i}) \subset V_{i}\subset V$ as $P\in
  V_{i}(\IC)$. Since $P+B$ is a maximal coset of $V$ we conclude
  $B=\mathrm{Stab}(V_i)$. We set $i=r$ and terminate the construction.
  
  Second, assume $b+V_i\not=V_i$ for some $b\in B(\IC)$.
  Then $P+B \subset V_i\cap (b+V_i)$ and so there exists an
  irreducible component $V_{i+1}$ of $V_i\cap (b+V_i)$ containing
  $P+B$. We note that $V_{i+1}\subsetneq V_i$.

  As the dimension of the $V_i$ is strictly decreasing the
  construction must terminate after finitely many steps.

  We  require $\deg(b+V_i)=\deg V_i$; the degree is with respect to the immersion
  $A\rightarrow\IP^n$. 
  B\'ezout's Theorem implies $\deg V_{i+1} \le (\deg
  V_i)^2$. So $\deg V_r= \deg B$ is bounded from above solely in terms
  of $r\le \dim V$ and $\deg V_0=\deg V$. So $\deg B$ is bounded
  from above solely in terms of $A$ and $V$. 

  Now it is known that $A$ contains at most finitely many abelian
  subvarieties of bounded degree. This implies (i).

  By (i) we only need to consider finitely many different $B$ in the
  definition of the Ueno locus. Given an abelian subvariety $B\subset
  A$ we have $P+B\subset V$ if and only if $\dim_P
  \varphi|_V^{-1}(\varphi(P)) \ge \dim B$. By dimension theory in the
  set
  $\{ P \in V(\IC) : \dim_P
  \varphi|_V^{-1}(\varphi(P)) \ge \dim B\}$ is Zariski closed in $V$.
  As $V^{\circ}$ is a finite union of such sets it must also be
  Zariski closed. This is (ii).

  Finally, suppose $V=V^{\circ}$.  Then by (ii) and since $V$ is
  irreducible, a single culprit is responsible. There exists an
  abelian subvariety $B\subset A$ with $\dim B\ge 1$ with $P+B\subset
  V$ for all $P\in V(\IC)$. But then $B\subset \mathrm{Stab}(V)$ and
  so ``$\Longrightarrow$'' follows. The converse direction is
  immediate from the definition. 
\end{proof}


%\begin{exercise}
%   Show that $\mathrm{Stab}(Z) = \bigcap_{v\in Z}(Z-v)$ and conclude
%   that $\mathrm{Stab}(Z)$ is an algebraic subgroup of $\IC^g$. 
% \end{exercise}






% \begin{example}
%   Let us investiage the connection with Section~\ref{subsub:functrans} in the
%   proof of the Ihara--Serre--Tate Theorem. Let $P\in
%   \IC[X,Y]$ be irreducible and let $\ell$ be a prime number with
%   $P\mid P(X^\ell,Y^\ell)$. So $P\mid P(X^{\ell^r},Y^{\ell^e})$ for
%   all $r\in\IN$.

%   Let $V=\{(z_1,z_2)\in(\IC^\times)^2 : P(z_1,z_2)=0\}$. The pre-image
%   $X = \{(w_1,w_2)\in\IC^2 : P(\exp(w_1),\exp(w_2))=0\}$ is a complex analytic
%   subset of $\IC^2$ of (complex) dimension $1$.
%   If $(w_1,w_2)\in X$, then $(\ell^rw_1,\ell^rw_2)\in X$ for all
%   $r\in\IN$. 
  
%   Then $X$ has a smooth point, and so we can find a local
%   parametrization
%   $f=(f_1,f_2)\colon \Delta\rightarrow X$ where $\Delta = \{z\in\IC :
%   |z|<1\}$.   
% \end{example}

\section{Further Reading and Open Problems}

%We keep the notation from Section \ref{sec:conclusionft}.

Abramovich's paper~\cite{abramovich} contains some history on the
contributions of Kawamata, Ochiai, and Ueno to Theorem~\ref{thm:ueno}
as well as a generalization of this result.

% We have the following theorem. 
% \begin{theorem}[Abramovich, Kawamata, Ueno]
%   \begin{enumerate}
%   \item [(i)]   The Ueno locus $V^{\circ}$ is Zariski closed in $V$.
%   \item[(ii)] We have
%     $V^{\circ}=V$ if and only if  $\dim \mathrm{Stab}(V)\ge 1$.
%   \item[(iii)]
%     Moreover, there
%     exists a finite set $\cB$ of abelian subvarieties of $A$ with the
%     following property. If $C$ is a coset of $A$ with $C\subset V$, then
%     $C$ is contained in the translate of a member of $\cB$. 
%   \end{enumerate}
% \end{theorem}
% \begin{proof}
%   See Theorems 1 and 2 for (i) and (ii). 
% \end{proof}

% \begin{theorem}
%   Let $\varphi\colon \IC\rightarrow A(\IC)$ be a non-constant holomorphic map with
%   image in $V(\IC)$.
%   Then $\varphi(\IC)  \subset V^{\circ}(\IC)$. 
% \end{theorem}

In Hector's lectures we will get to know the notion of hyperbolicity
and smooth varieties of general type.

\begin{definition}
  Let $V$ be a smooth, irreducible projective variety defined over $\IC$.
  \begin{enumerate}
  \item [(i)] We call $V$ \emph{hyperbolic} if
    any holomorphic map $\IC \rightarrow A(\IC)$ with values in
    $V(\IC)$ is
    constant.
  \item[(ii)] We say $V$ is of \emph{general type} if its
    Kodaira dimension equals $\dim V$. 
  \end{enumerate}
\end{definition}

\begin{theorem} Let $V$ be an smooth  irreducible subvariety of $A$. 
  \begin{enumerate}
  \item [(i)]  Then $V$ is hyperbolic if and only if
    $V^{\circ}=\emptyset$. 
  \item[(ii)] If $V$ is smooth, then
    \begin{equation*}
      V \text{ is of general type}\quad\Leftrightarrow \quad\dim
      \mathrm{Stab}(V)=0 \quad\Leftrightarrow\quad
      \text{$V^{\circ}$ is
        not Zariski dense in $V$.}
    \end{equation*}
  \end{enumerate}
\end{theorem}

The following problem is related to Schanuel's Conjecture.\footnote{It is
  open as far as I know.}

\begin{exercise}
  Prove $\pi^{\pi^{\pi^\pi}}\not\in\IZ$.    
\end{exercise}

The following conjecture due to Wilkie remains open. It predicts that
in $\IRexp$, one can obtain a poly-logarithmic upper bound for the
number of rational points of bounded height. 

\begin{conjecture}[Wilkie]
  Let $X$ be definable in $\IRexp$. There exist $c_1(X)>0$ and $c_2(X)>0$
  such that
  \begin{equation*}
    N(X\ssm X^{\mathrm{alg}},T) \le c_1(X) (\log T)^{c_2(X)}
  \end{equation*}
  for all $T\ge e$.
\end{conjecture}
%\begin{proof}
%   For (i) we require that the image of any non-constant holomorphic
%   map $\IC\rightarrow A(\IC)$ with image in $V(\IC)$ has values in
%   $V^{\circ}$. This is a theorem, !!!REF. The converse direction is easier, indeed, any coset of
%   an abelian variety is the target of a non-constant holomorphic map
%   on $\IC$.  
% \end{proof}

%%% Local Variables:
%%% TeX-master: "main"
%%% End:
