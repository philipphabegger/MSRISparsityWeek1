\chapter{O-minimal Geometry}
\section{Overview}

O-minimal geometry was developed in the 1980s by various authors
including Pillai, Steinhorn, and van den Dries. Some call it an
incarnation of Grothendieck's vision of a ``tame topology'' described
in his ``Esquisse d'un Programme''. It strives to generalize real
semi-algebraic geometry to include also real (sub)-analytic sets.
This makes it a powerful and robuts tool to study various problems
including conjectures in diophantine geometry.

O-minimal Geometry has strong ties mathematical logic and model
theory. 

The basic motivation is to carve out a class ``interesting'' subsets
of $\IR^n$ that is stable under various basic observations. These sets
will be called ``definable'' (in a given o-minimal structure). 

For example, consider an algebraic curve $C\subset (\IC^\times)$ and
the map $\be \colon\IR^2\rightarrow (\IC^\times)^2$ defined by
\begin{equation*}
  \be(x,y) = (e^{2\pi i x},e^{2\pi i y}). 
\end{equation*}
Consider the set $X = \be^{-1}(C) \cap [0,1)^2$. Then $C$ contains
infinitely many torsion points of $(\IC^\times)^2$ if and only if $X$
contains infinitely many rational points.
We will see that $X$ is definable in a certain o-minimal structure.
Note that $X$ is not necessarily real analytic (but it is real
semi-analytic). 

\section{O-minimal Structures}


\begin{definition}
  \label{def:structure}
%  For each $m\in \IN_0$ consider the set of all subsets of $\IR^n$.
  A \emph{structure} $\cS$ is a sequence $(S_m)_{m\in\IN_0}$
  where each $S_m$ is a set of subsets of $\IR^m$ with the following
  properties for all $m\in \IN_0$:
  \begin{enumerate}
  \item [(i)] Each $S_m$ is closed under boolean operations,
    \textit{i.e.}, if $X,Y\in S_m$, then $X\cup Y\in S_m$ and
    $\IR^m\ssm X\in S_m$.
  \item[(ii)] If $X\in S_m$, then $\IR\times X\in S_m$ and $X\times
    \IR\in S_m$.
  \item[(iii)] If $X\in S_{m}$ and $\pi\colon
    \IR^{m}\rightarrow\IR^k$ is the projection to $k$ distinct
    coordinates, then 
    $\pi(X)\in S_k$.
  \item[(iv)] All real semi-algebraic subsets of $\IR^m$ are in
    $S_m$.\footnote{A real semi-algebraic subsets of $\IR^m$ is a
      finite union of sets each of which is set of 
      solutions of a finite number of polynomial equalities and
      strict inequalities.}    
  \end{enumerate}
  We say that a subset of  $\IR^m$ is \emph{definable} in $\cS$ if it
  is a member of $S_m$.
\end{definition}

\begin{example} Let us consider the two extremes.
  \begin{enumerate}
  \item [(i)] Set $S_m$ to be the set of all subsets of $\IR^m$ for
    all $m\ge 0$. Then $(S_m)_m$ is a structure. It is useless in this
    context and will play no role
    in what follows.
  \item[(ii)]
    Set $S_m$ to be the set of all real semi-algebraic subsets of $\IR^m$
    for all $m\ge 0$. By definition each $S_m$ is closed under taking
    unions. Moreover, each $S_m$ is closed under taking intersections.    
    Say $X\in S_m$. Let us  verify that $\IR^m \ssm X$. Indeed, by
    closedness under union and intersection we may assume that either 
    \begin{equation*}
      X = f^{-1}(0)=\{ x\in\IR^m : f(x) = 0\}
      \quad\text{or}\quad       X = f^{-1}((0,\infty))=\{ (x)\in\IR^m : f(x) > 0\}
    \end{equation*}
    for a single $f\in\IR[X_1,\ldots,X_m]$. In the first case
    we have $\IR\ssm X=\{x\in\IR^m : f(x)^2 > 0\}$ and in the second
    case
    $\IR\ssm X = \{x\in\IR^m : f(x)=0\} \cup \{x\in \IR^m: -f(x)^2 >
    0\}$; both are real semi-algebraic.
    Property (ii) in the definition of structure can be verified
    directly. Property (iii) is also true, but harder to show. It
    is called the Seidenberg--Tarski Theorem.
  \end{enumerate}
\end{example}



\begin{exercise}
  Find and understand a proof of Seidenberg--Tarski Theorem. For
  example, one can be found in Chapter 2, \S 2~\cite{D:oMin}. This
  proof runs via a cell-decomposition, a concept we will discuss
  further down. 
\end{exercise}

Here are some direct consequences of the definition of a structure.  

\begin{lemma}
  \label{lem:structprops}
  Let $\cS$ be a structure. Let $X\subset\IR^m$ and $Y\subset\IR^n$ be definable subsets of $\cS$.
  \begin{enumerate}   
  \item [(i)] Then $X\times Y$ is definable.
  \item[(ii)] If $m=n$, then $X\cap Y$ is definable.
  \item[(iii)] The image of $X$ under any polynomial map
    $\IR^m\rightarrow \IR^n$ is definable. 
  \end{enumerate}
\end{lemma}
\begin{proof}
  For (i) note that $X\times Y =  (X\times \IR^n)\cap (\IR^m\times
  Y)$. For (ii) we need $X\cap Y = \IR^m\ssm ((\IR^m\ssm X)\cup (\IR^m
  \ssm Y))$.
  Let $p_1,\ldots,p_n\in\IR[X_1,\ldots,X_m]$ and consider the graph
  \begin{equation*}
    \Gamma = \{(x_1,\ldots,x_m,x_{m+1},\ldots,x_{m+n})\in\IR^{m+n} :
    x_{m+i}-p_i(x_1,\ldots,x_m)= 0 \text{ for all }
    i\} 
  \end{equation*}
  of $p=(p_1,\ldots,p_n)\colon \IR^m\rightarrow\IR^n$. Then $\Gamma$ is
  semi-algebraic and it is definable in $\cS$. Now $p(X)$ is the
  projection of $\Gamma \cap (X\times\IR^n)$ to $\IR^n$ and hence also
  definable. This gives (iii). 
\end{proof}

\begin{definition}
  Let $\cS$ be a structure and $X\subset\IR^m$ definable.
  A map $f\colon X\rightarrow\IR^n$ is
  called \emph{definable} in $\cS$, if its graph $\Gamma(f) =
  \{(x,f(x)) : x\in X\}\subset\IR^{m+n}$ is definable.
\end{definition}

Domain $X$ and image $f(X)$ of a definable map must also be definable
by an argument similar to the one in the proof of Lemma~\ref{lem:structprops}(iii).

\begin{example}
  Let $\cS$ be an arbitrary structure. 
  On a definable domain, 
  all polynomial functions, all rational functions without poles on the
  domain, and all algebraic functions are definable.
  In particular, $x\mapsto \sqrt{x}$ is definable on $X=[0,\infty)$.
  The absolute value function $x\mapsto |x|$ is definable on $\IR$.
\end{example}



While structures are very flexible, the definition does not contain
any clause that ``tempers'' the collection of sets.

\begin{definition}
  A {structure} $\cS=(S_m)_m$ is call \emph{o-minimal}
  \begin{enumerate}
  \item [(v)]  if each 
  element in $S_1$ is a finite union of points and
  intervals, that are possibly unbouded.
  \end{enumerate}
\end{definition}

\begin{example}
  By the Seidenberg--Tarski Theorem we already know that the real
  semi-algebraic sets constitute a structure. Say $f\in\IR[X]$.
  The zero locus of $f$ is finite or $\IR$. Moreover,
  $f^{-1}((0,\infty))$ is open in $\IR$ and has only finitely many
  boundary points. So it is a finite union of intervals. Thus this
  structure is o-minimal, we will denote it by $\IRalg$. 
\end{example}

The final axiom (v) in does an excellent job in ``taming'' subsets of
$\IR^m$. This will hopefully be apparent after seeing the
applications. But the axiom comes at a heavy price.
We immediately stake out the limits of o-minimality.

\begin{nonexample}
  Let $\cS$ be an o-minimal structure. Then $\IZ$ is not definable in
  $\cS$. Moreover, the function $\sin \colon\IR\rightarrow\IR$
  is not definable in any o-minimal structure.  
\end{nonexample}
%\begin{definition}
%  Let 
%\end{definition}

\section{Structures and First-Order Logic}

The various sets appearing in Definition~\ref{def:structure}(i) --
(iv) can equivalently be described using first-order logic formulas. In fact, the
terminology ``structure'' comes from mathematical logic. Our main
emphasis will be on the geometry.  But we  take some time to give
an informal overview of the connection to logic as it is equivalent
and often very useful. 

Say $\cS$ is a structure and suppose $X\subset\IR^m$ is definable.
We think of the formula $x\in X$ as an $m$-ary relation signifying
membership of the $m$-tuple $x$ in $X$. Subject to the usual rules we
can combine two formulas using the logical symbols $\wedge$ (for
``and'') and $\vee$ (for ``or''), we can negate them by using $\neg$
(for ``not''), and we may use the existential ($\exists$) and
universal ($\forall x$) quantifier.

In the language of mathematical logic, $\IR$ is the universe of our
structure. This is were any element lies ``by default'' in expressions
such as $\exists x:\phi(x)$ and $\forall x:\phi(x,y)$. By abuse of
notation we often also quantify over tuples in $\IR^m$.

Let $Y\subset\IR^m$ be definable
and let $\pi\colon\IR^m\rightarrow \IR^{m-1}$ project to the first
$m-1$ coordinates (if $m\ge 1$). 

\begin{center}
\begin{tabular}{c|c|c}
  $X\cap Y$ \text{ is definable} & $((x_1,\ldots,x_m)\in X)\wedge ((x_1,\ldots,x_m)\in Y)$ &
                                                               $x_1,\ldots,x_m$  are free\\
  $\IR^m\ssm X$ \text{ is definable} &  $\neg((x_1,\ldots,x_m)\in X)$ & $x_1,\ldots,x_m$ are free \\
  $X\times\IR$ \text{ is definable} & $(x_1,\ldots,x_m)\in X$&
                                                               $x_1,\ldots,x_{m+1}$
                                                               are free\\
  $\pi(X)$ \text{ is definable} &$\exists x_m : (x_1,\ldots,x_{m})\in
                                  X$  & $x_1,\ldots,x_{m-1}$ are free
  \\ \hline  &\text{Further examples} &  \\ \hline
    $X\cup Y$ \text{ is definable} & $((x_1,\ldots,x_m)\in X)\vee ((x_1,\ldots,x_m)\in Y)$ &
                                                               $x_1,\ldots,x_m$
                                                                                             are free\\
  $\IR\ssm \pi(\IR^2\ssm X)$ & $\forall y : \phi(x,y)$ & $\phi$ is a binary
                                                 rel. defining
                                                 $X\subset\IR^2$  
\end{tabular}
\end{center}

\begin{warning}
  It is not allowed to quantify over sets that are not definable in
  the ambient structure. For example, if $\cS$ is an o-minimal
  structure, then $\IZ$ is not definable in $\cS$. So a formula like
  $\forall x \in\IZ:\forall y\in\IZ :\forall z\in \IZ :
  (xyz=0) \vee \neg(x^n+y^n=z^n)$ is illegal.
\end{warning}

Here is an example how to apply this point of view to quickly show
that the locus continuity of a definable function is definable.

\begin{lemma}
  Let $X$ be a definable set and $f\colon X\rightarrow\IR$ a definable
  function. Then $\{ x \in X : f \text{ is continuous at }x\}$ is
  definable. 
\end{lemma}
\begin{proof}
  Let $\|(x_1,\ldots,x_n)\| = (x_1^2+\cdots+x_n^2)^{1/2}$.
  The set in question is described by the formula
  \begin{equation*}
 \forall \epsilon : \exists \delta : (\epsilon \le 0)
    \vee
    (\forall y : \|x-y\|^2\ge \delta \vee |f(x)-f(y)|<\epsilon).\qedhere
  \end{equation*}
\end{proof}

\section{Further Reading and Open Problems}

% \begin{itemize}
% \end{itemize}

%%% Local Variables:
%%% TeX-master: "main"
%%% End:
