% MSRI - SGS Sparsity week 1, Wednesday, 1 lecture, 60 minutes

\documentclass{beamer}

\usepackage{amsmath,amssymb,amsthm,mathrsfs,amscd,mathtools}
\usepackage{datetime}
\usepackage{csquotes}
\usepackage{hyperref}
\usepackage{graphicx}
\usepackage{tikz,tikz-cd}
\usetikzlibrary{arrows,shapes}

\usetheme{Metropolis}
\metroset{block=fill}




\newcounter{maincounter}
\newcounter{excounter}

\newtheorem{conjecture}{Conjecture}
% \setbeamercolor{block body}{bg=mDarkTeal!15}
% \setbeamercolor{block title}{bg=mDarkTeal,fg=black!2}


\newcounter{maincounter}
\newcounter{excounter}
\numberwithin{maincounter}{chapter}
\numberwithin{equation}{chapter}
\numberwithin{excounter}{chapter}
\renewcommand{\theexcounter}{\thechapter.\Alph{excounter}}
\newtheorem{lemma}[maincounter]{Lemma}
\newtheorem{proposition}[maincounter]{Proposition}
\newtheorem{corollary}[maincounter]{Corollary}
\newtheorem{remark}[maincounter]{Remark}
\newtheorem{theorem}[maincounter]{Theorem}
\newtheorem{exercise}[excounter]{Exercise}
\newtheorem{example}[maincounter]{Example}

\newtheorem*{crucial}{Crucial Observation}

\newtheorem{conjecture}[maincounter]{Conjecture}
\newtheorem{definition}[maincounter]{Definition}

\def\AA{\mathbb{A}}
\def\BB{\mathbb{B}}
\def\EE{\mathbb{E}}
\def\HH{\mathbb{H}}
\def\DD{\mathbb{D}}
\def\NN{\mathbb{N}}
\def\RR{\mathbb{R}}
\def\TT{\mathbb{T}}
\def\CC{\mathbb{C}}
\def\ZZ{\mathbb{Z}}
\def\PP{\mathbb{P}}
\def\QQ{\mathbb{Q}}
\def\FF{\mathbb{F}}
\def\GG{\mathbb{G}}
\def\LL{\mathbb{L}}
\def\MM{\mathbb{M}}
\def\SS{\mathbb{S}}
\def\UU{\mathbb{U}}
\def\XX{\mathbb{X}}


%%% Philipp's macros

\newcommand{\dom}[1]{{\mathrm {dom}}({#1})}
\newcommand{\sman}[1]{{#1}^{\mathrm{sm,an}}}
\newcommand{\ansm}[1]{{#1}^{\mathrm{an,sm}}}
\newcommand{\sm}[1]{{#1}^{\mathrm{sm}}}
\newcommand{\anE}{\mathrm{an}}
\newcommand{\an}[1]{{#1}^{\anE}}
\newcommand{\stab}[1]{{\mathrm{Stab}(#1)}}


\newcommand{\hgtexp}{S}

\newcommand{\rank}{{\rm rank}\,}
\newcommand{\Hpoly}[2]{{H^{}_{#1}({#2})}}
\newcommand{\poly}[2]{{#1^{}({#2})}}
\newcommand{\polyt}[2]{{#1^{\sim}({#2})}}
\newcommand{\polytiso}[2]{{#1^{\sim,{\rm iso}}({#2})}}
%\renewcommand{\graph}[1]{\Gamma({#1})}
\newcommand{\atopx}[2]{{\genfrac{}{}{0pt}{}{#1}{#2}}}
\newcommand{\IP}{{\PP}}
\newcommand{\IG}{{\GG}}
\newcommand{\IH}{{\HH}}
\newcommand{\IC}{{\CC}}
\newcommand{\IR}{{\RR}}
\newcommand{\IT}{{\TT}}
\newcommand{\IRan}{{{\RR}_{\rm an}}}
\newcommand{\IRanexp}{{{\RR}_{\rm an,exp}}}
\newcommand{\RRan}{{\IRan}}
\newcommand{\RRanexp}{{\IRanexp}}
\newcommand{\IRalg}{{\RR}_{\rm alg}}
\newcommand{\IQbar}{{\overline{\QQ}}}
\newcommand{\Kbar}{{\overline{K}}}
\newcommand{\IZ}{{\ZZ}}
\newcommand{\IN}{{\NN}}
\newcommand{\IA}{{\AA}}
\newcommand{\IQ}{{\QQ}}
\newcommand{\IQpbar}{{\overline{\QQ}_p}}
\newcommand{\IQp}{{\QQ_p}}
\newcommand{\ts}[1]{{T}_0({#1})}

\newcommand{\cC}{{\mathcal C}}
\newcommand{\cE}{{\mathcal E}}
\newcommand{\cF}{{\mathcal F}}
\newcommand{\cK}{{\mathcal K}}
\newcommand{\cL}{{\mathcal L}}
\newcommand{\cM}{{\mathcal M}}
\newcommand{\cO}{{\mathcal O}}
\newcommand{\cV}{{\mathcal V}}
\newcommand{\cW}{{\mathcal W}}
\newcommand{\cX}{{\mathcal{X}}}
\newcommand{\cY}{{\mathcal Y}}
\newcommand{\cZ}{{\mathcal Z}}



\newcommand{\defZ}{Z}
\newcommand{\defF}{F}
\newcommand{\defW}{W}
\newcommand{\defC}{C}
\newcommand{\defE}{E}
%\newcommand{\deffam}{F}


\newcommand{\re}[1]{{\rm Re}({#1})}
\newcommand{\imS}{{\rm Im}}
\newcommand{\im}[1]{\imS({#1})}
\newcommand{\imageS}{{\rm im}}
\newcommand{\image}[1]{\imageS({#1})}
\newcommand{\volS}{{\rm vol}}
\newcommand{\vol}[1]{\volS({#1})}
\newcommand{\orth}[1]{{#1}^{\bot}}
\newcommand{\mat}[2]{{\rm Mat}_{#1}({#2})}
\newcommand{\ssm}{\setminus}
\newcommand{\ord}[1]{{\rm ord}({#1})}
\newcommand{\opt}[2]{{\rm Opt}_{#2}({#1})}
\newcommand{\Height}[1]{{H}({#1})}
\newcommand{\trdeg}{{\rm trdeg\,}} 
\newcommand{\geo}[1]{\langle {#1}\rangle_{{\rm geo}}}
\newcommand{\defect}{\delta}
\newcommand{\geodef}{{\delta_{\rm geo}}}
\newcommand{\en}[1]{{\rm End}({#1})}
\newcommand{\Hom}[1]{{\rm Hom}({#1})}
\newcommand{\hommaxR}[1]{\text{\rm Hom}({#1})^{*}_{\IR}}
\newcommand{\arith}{\rm arith}
\newcommand{\sgu}[2]{{#1}^{[{#2}]}}
\newcommand{\oa}[1]{{#1}^{\rm oa}}
\newcommand{\codim}{{\rm codim}}
\newcommand{\lgo}{LGO}
\newcommand{\zcl}[1]{{\rm Zcl}({#1})}


\newcommand{\trans}[1]{{#1}^{T}}

\newcommand{\red}[1]{\textcolor{red}{#1}}

\renewcommand{\subset}{\subseteq} %%% Some people think \subset
%%% excludes equality
\renewcommand{\supset}{\supseteq}

\newcommand{\gra}[1]{\mathrm{Gr}({#1})}


\newcommand{\gl}[2]{{\mathrm {GL}}_{#1}({#2})}
\renewcommand{\sp}[2]{{\mathrm {Sp}}_{#1}({#2})}
\newcommand{\autS}{{\mathrm {Aut}}}
\newcommand{\aut}[1]{\autS({#1})}

\newcommand{\spec}[1]{\mathrm{Spec}\,{#1}}

\newcommand{\tor}[1]{{#1}_{\mathrm{tor}}}
\newcommand{\gal}[1]{{\mathrm{Gal}}({#1})}


\newcommand{\zeroset}[1]{\mathscr{Z}({#1})}


\newcommand{\jac}{\mathrm{Jac}}

\newcommand{\bfzeta}{{\boldsymbol{\zeta}}}

\newcommand{\mattt}[4]
{\left(
  \begin{array}{cc}
    {#1} & {#2} \\ {#3} & {#4} 
  \end{array}
\right)}

\newcommand{\matto}[2]
{\left(
  \begin{array}{c}
    {#1} \\ {#2}
  \end{array}
\right)}

\newcommand{\matot}[2]
{\left(
  \begin{array}{cc}
    {#1} & {#2}
  \end{array}
\right)}


\title{MSRI Summer Graduate School \\ Sparsity of Algebraic Points \\
  Day 5: Unlikely Intersections}
\author{Philipp~Habegger \\ University of Basel \\ \texttt{philipp.habegger@unibas.ch}}
\date{Friday, June 11, 2021}

\begin{document}

\setlength{\abovecaptionskip}{0pt} 
\setlength{\belowcaptionskip}{0pt} 

\renewcommand{\figurename}{Fig.}


\begin{frame}
  \titlepage
\end{frame}

\section{Further Developments}

\begin{frame}
  In his 2011 Annals of Math paper proved Andr\'e--Oort for
  all subvarieties of $Y(1)^m$.

  \begin{theorem}[Pila]
    Let $V$ be an irreducible subvariety of $Y(1)^m$. Then
    $$V(\IC)\cap
    Y(1)^{m}_{\mathrm{special}}\text{ is Zariski dense in
      $V$}\Longleftrightarrow V\text{ is special.}$$
  \end{theorem}

  $V$ is  special in $Y(1)^m$ if, after a  permutating coordinates, we get
  % it is an irreducible component of the algebraic set defined by
  % \begin{itemize}
  % \item $X_{1},\ldots,X_{k_1-1}$ are constant on $V$ and take values in
  %   $Y(1)_{\mathrm{special}}$
  % \item for $j\ge 1$ there exist $N_{j,1},N_{j,2},\ldots\in\IN$ with 
  %   $\Phi_{N_{1,1}}(X_{k_j},X_{k_j+1})=\Phi_{N_{1,2}}(X_{k_j+1},X_{k_1+2})
  %   = \cdots =\Phi_{N_{1,k_{j+1}-k_j+1}}(X_{k_{j+1}-2},X_{k_{j+1}-1})$
  % \end{itemize}
  ``packets'' that are constant or linked by an isogeny:
  \vspace{-.2cm}
  \begin{center}
  $(\underbrace{X_1,\ldots,X_{k_1-1}}_{\text{fixed}},\underbrace{X_{k_1},\ldots,X_{k_2-1}}_{\text{pw
      isogenous}},\ldots,
  \underbrace{X_{k_{s-1}},\ldots,X_m}_{\text{pw isogenous}})$      
  \end{center}  
\end{frame}


\begin{frame}
  Pila's Theorem on Ax--Lindemann Weierstrass
  was generalized by Pila and Tsimerman to a (weak) Ax--Schanuel
  Theorem for the $j$-function. The formulation here is chosen to
  mimic Ax's Theorem.

  \begin{theorem}[Pila--Tsimerman]
    Let $\tau_1,\ldots,\tau_m\colon\Delta\rightarrow\IH$ be
    non-constant holomorphic maps on the open unit disk $\Delta$.
    Suppose that there does not exist $N\in\IN$ and distinct $k,l$
    such that $\Phi_N(j\circ\tau_k,j\circ \tau_l)=0$ holds
    identically. Then
    \begin{equation*}
      \mathrm{trdeg}\,\IC(\tau_1,\ldots,\tau_m,j\circ\tau_1,\ldots,j\circ\tau_m)/\IC
      \ge m+1.
    \end{equation*}
  \end{theorem}   
\end{frame}

\begin{frame}{Andr\'e--Oort for $\mathcal{A}_g$}
  The coarse moduli space $\mathcal{A}_g$ of principally polarized abelian varieties
  of dimension $g$ is a quotient of
  \begin{equation*}
    \IH_g = \{Z\in \mat{g}{\IC} : Z^t = Z \text{ and }
    \mathrm{Im}(Z) \text{ pos. definite}\}
  \end{equation*}
  by the action of the symplectic group $\mathrm{Sp}_{2g}(\IZ)$.
  
  \begin{definition}
    A point in $\mathcal{A}_g(\IC)$ is called \alert{special} or
    \alert{CM}
    if the abelian
    variety  it represents has complex
    multiplication. 
  \end{definition}

  We know $\mathcal{A}_g$ is an irreducible quasi-projective variety of
  dimension $g(g+1)/2$ defined over $\IQ$.
  
  There is a uniformization map
    $\IH_g \rightarrow \mathcal{A}_g(\IC)$,
  special points lift to algebraic points in $\mat{g}{\IC}$ (of
  bounded degree).
\end{frame}

\begin{frame}
  % Special points are points  $\mathcal{A}_g(\IC)$ where the
  % associated endomorphism ring is as ``large as possible''.
  Very roughly speaking, one can think of special subvarieties of
  $\mathcal{A}_g$ as irreducible subvarieties on which the ``generic''
  endomorphism ring is constant. (The actual definition is more
  involved.)

  \begin{example}[Abelian surfaces, so $g=2$,  here $\dim \cA_2=3$]
    The diagonal period matrices
    \begin{equation*}
      \left\{\mattt{\tau_1}{0}{0}{\tau_2} : \tau_1,\tau_2\in\IH
      \right\} \subset \IH_2
    \end{equation*}
    correspond to products of elliptic curves $E_1\times E_2$. This
    determines 
    a \alert{special surface} in $\cA_2$. Generically, 
    \begin{equation*}
      \mathrm{End}(E_1\times E_2) =\IZ\times\IZ. 
    \end{equation*}
    If in addition $\tau_2=2\tau_1$, say, then we are on a \alert{special curve}.
    The $E_1$ and $E_2$ are related by an isogeny of degree $2$
    and $\mathrm{End}(E_1\times E_2)$ has rank $4$. 
  \end{example}
\end{frame}

\begin{frame}
  After many important contributions by:

  \begin{itemize}
  \item  Pila--Tsimerman, Klinger--Ullmo--Yafaev (Ax--Schanuel)
  \item   Andreatta--Goren--Howard--Madapusi-Pera, Yuan--Zhang (Colmez
    Conjecture on the average: Large Galois Orbit)
  \item Masser--Wüstholz (Endomorphism estimates: Large Galois Orbit)
  \item   Peterzil--Starchenko (definability)    
  \end{itemize}

  Tsimerman proved the Andr\'e--Oort Conjecture for $\cA_g$. 
  \begin{theorem}[Tsimerman]
    Let $V$ be an irreducible subvariety of $\mathcal{A}_g$, then
    \begin{equation*}
      V(\IC)\cap \mathcal{A}_{g,\mathrm{special}} \text{ is Zariski
        dense in $V$}\Longleftrightarrow V\text{ is special.}
    \end{equation*}
  \end{theorem}

  There is still work to do: Shimura varieties.
\end{frame}

\section{Special Points vs. Unlikely Intersections}

\begin{frame}

  Let $X$ be some ``arithmetically interesting'' variety, such as an
  abelian variety, or $\IG_{\mathrm{m}}^m$, or $Y(1)^m$ etc.
  Then $X$  often has
  a ``special points'' $X_{\mathrm{special}}$ (e.g. torsion points) and ``special
  subvarieties'' (torsion cosets).

  The basic template for a result of Andr\'e--Oort and Manin--Mumford
  type goes as follows:
  
  Let $V$ be an irreducible subvariety of
  $X$. Then
  \begin{equation*}
    V(\IC) \cap X_{\mathrm{special}}\text{ is Zariski dense in
      $V$}\Leftrightarrow \text{$V$ is special}.
  \end{equation*}

  In the 2000s,
  Zilber, Pink, and Bombieri--Masser--Zannier asked: what if we
  replace $X_{\mathrm{special}}$ by all special subvarieties
  \begin{equation*}
    \bigcup_{\substack{S\subset X \\ \text{$S$ special}}} V\cap S?
  \end{equation*}

  In all our cases $X$ itself was special, so we must restrict
  $S$ in the union.  
\end{frame}

\begin{frame}{Conjecture on Unlikely Intersection}

  \begin{tabular}{l|l|l}
    $X$ & \text{special points} & \text{special subvar.}  \\
    \hline
    $\IG_{\mathrm{m}}^m$ \text{ or abel. var.} &\text{torsion points} & \text{torsion cosets}
    \\
    $Y(1)^m$ & come from CM elliptic curves & $\Phi_N, $ CM
    \\
    \alert{families of ab. var.} & \alert{torsion points in CM fibers}
                                & \alert{tba}
    \\
    \vdots & \vdots & \vdots 
  \end{tabular}

  \begin{conjecture}[Zilber--Pink or Conjecture on Unlikely Intersections]
    Let $V$ be an irreducible subvariety of $X$. Suppose that $V$ is
    not contained in a proper special subvariety of $X$. Then
    \begin{equation*}
      \bigcup_{S : \dim S  < \dim X - \dim V} V \cap S
    \end{equation*}
    is not Zariski dense in $V$.     
  \end{conjecture}

  Special points are just special subvarieties of dimension $0$. 
\end{frame}

\section{Group Setting}

\begin{frame}{Unlikely Intersections in $\IG_{\mathrm{m}}^3$}  
  Consider a curve $C$ in  the ambient variety $\IG_{\mathrm{m}}^3$.
  Unlikely intersections asks:  is $$\bigcup_{\dim H \le 1} C\cap
  H$$ finite where $H$ ranges over algebraic subgroups of
  $\IG_{\mathrm{m}}^3$?

  If $\dim H=0$, we recover torsion points and Manin--Mumford.
  
  If $\dim H=1$, the unit component of $H$ is
  \begin{equation*}
    \{(t^{a},t^{b},t^{c}) : t\in \IC^\times \}. 
  \end{equation*}
  Dually we see that  $H$ is described by two equations
  \begin{equation*}
    x^{\alpha_1} y^{\alpha_2} z^{\alpha_3} =
    x^{\beta_1} y^{\beta_2} z^{\beta_3} = 1.
  \end{equation*}
  for    \alert{linearly independent}
  $(\alpha_1,\alpha_2,\alpha_3), (\beta_1,\beta_2,\beta_3)\in\IZ^3$.
\end{frame}

\begin{frame}
  \begin{example}
    \begin{enumerate}
    \item [(i)]
      Say $C$ is the line $\left\{(x,1+x,2+x) : x\in \IC\ssm\{0,-1,-2\}\right\}$.
      
      Find solutions  $x\in\IC\ssm\{0,-1,-2\}$ of 
      \begin{equation*}
        x^{\alpha_1} (1+x)^{\alpha_2} (2+x)^{\alpha_3} =
        x^{\beta_1} (1+x)^{\beta_2} (2+x)^{\beta_3} = 1.
      \end{equation*}
      with \alert{varying} and \alert{independent}  vectors in the
      exponents.

      An early result of Bombieri--Masser--Zannier (1999) implies
      that there are only \alert{finitely} many solutions $x$.

    \item[(ii)] Say $C$ is $\{(x,1+x,1) : x\in \IC\ssm \{0,-1\}\}$.
      There are \alert{infinitely} many solutions:  let
      $\zeta\in\mu_\infty$ have order $N\ge 3$, then 
      \begin{equation*}
        \zeta^{0} (1+\zeta)^{0} 1^{1} = \zeta^{N} (1+\zeta)^{0} 1^0 = 0
      \end{equation*}
      with linearly independent exponent vectors $(0,0,1),(N,0,0)$.

      What goes wrong? $C$ is contained in the proper algebraic subgroup
      $\IG_{\mathrm{m}}^2\times \{1\}$ of $\IG_{\mathrm{m}}^3$. 
    \end{enumerate}
  \end{example}  
\end{frame}

\begin{frame}{Bounding the Height}
  \begin{lemma}
    Let 
    $x\in\IC\ssm\{0,-1,-2\}$ with
    \begin{equation*}
      x^{\alpha_1} (1+x)^{\alpha_2} (2+x)^{\alpha_3} =
      x^{\beta_1} (1+x)^{\beta_2} (2+x)^{\beta_3} = 1.
    \end{equation*}
    where
    $(\alpha_1,\alpha_2,\alpha_3),(\beta_1,\beta_2,\beta_3)\in\IZ^3$
    are linearly independent. Then $x\in\IQbar$ and $H(x)\le \cdots$. 
  \end{lemma}
  \begin{proof}\renewcommand{\qedsymbol}{}
    \vspace{2cm}
  \end{proof}  
\end{frame}

\begin{frame}
  \begin{proof}
    \vspace{6cm}
  \end{proof}  
\end{frame}

\begin{frame}{What is known?}
  \begin{theorem}
    Let $G$ be $\IG_\mathrm{m}^m$ or an abelian variety defined over
    $\IC$. Let $C$ be a curve in $G$ that is not contained in a proper
    algebraic subgroup, then
    \begin{equation*}
      \bigcup_{\substack{H\subset G \\\dim H\le \dim G-2}} C\cap H
      \qquad\text{is finite.}
    \end{equation*}
  \end{theorem}
  \begin{itemize}
  \item Maurin: $G=\IG_{\mathrm{m}}^m$ and $C/\IQbar$
  \item Bombieri--Masser--Zannier: $G=\IG_{\mathrm{m}}^m$ and $C/\IC$
  \item Carrizosa, Galateau, Ratazzi, R\'emond,
    R\'emond--Viada,  Viada: $G=$ product of elliptic
    curves and other cases when $C/\IQbar$. 
  \item H.--Pila: $G=$ abelian variety and $C/\IQbar$ (uses
    Pila--Wilkie). 
  \item Barroero--Dill: $G=$ abelian variety and $C/\IC$ (uses Gao's
    Ax--Schanuel for ``mixed Shimura Varieties''). 
  \end{itemize}
  
\end{frame}

\section{The Legendre Family of Elliptic Curves}

\begin{frame}{Legendre Family}
  Abelian varieties can vary in algebraic families.
  \begin{example}
    Let $Y(2) = \IP^1\ssm\{0,1,\infty\}$ and $\lambda\in
    Y(2)(\IC)=\IC\ssm\{0,1\}$. Then
    \begin{equation*}
      y^2 = x(x-1)(x-\lambda)
    \end{equation*}
    determines 
    the \alert{Legendre family of elliptic
      curves} $\cE\subset \IP^2\times Y(2)$, it is a surface.
    
    We denote the fiber above $\lambda$  by $\cE_\lambda
    \subset\IP^2$ and    
    define
    \begin{equation*}
      \cE_{\mathrm{tors}} = \bigcup_{\lambda\not=0,1}
      \cE_{\lambda,\mathrm{tors}}. 
    \end{equation*}
    This is \alert{not} a group.
  \end{example}
\end{frame}

\begin{frame}
  Let $C\subset \mathcal{E}$ be a curve. Does a
  Manin--Mumford type statement hold for $C\cap
  \mathcal{E}_{\mathrm{tors}}$? In other words, is this intersection
  finite ``most of the time''?

  The answer is \alert{no}! 

  Heuristics: For $N\in\IZ$,  multiplication-by-$N$ is a
  morphism
  \begin{equation*}
    \begin{tikzcd}[ampersand replacement=\&,column sep=small, row sep=small] 
    \cE \arrow{dr}[swap]{\pi}  \arrow{rr}{[N]} \&  \& \cE\arrow{dl}{\pi}  \\
     \& Y(2)   
  \end{tikzcd}  
  \end{equation*}
  For $N\ge 1$ the kernel  $\ker[N]$ is a curve
  in $\cE$, \textit{e.g.},
  \begin{alignat*}1
    \ker[2] = &\{[0:0:1],[1:0:1],[0:1:0]\}\times Y(2)\\
    &\cup
    \{([\lambda:0:1],\lambda) : \lambda \in Y(2)(\IC)\}.
  \end{alignat*}
  % and
  % $$\cE_{\mathrm{tors}}=\bigcup_{N\ge 1}\ker[N].$$
  We expect $C\cap \ker[N]\not=\emptyset$
  since $\dim C + \dim \ker[N] = 2 = \dim \cE$. The intersection is
  \alert{not} unlikely. 
  
  We expect $C\cap \cE_{\mathrm{tors}} = \bigcup_{N\ge 1}C\cap \ker[N]
  \text{ to be infinite.}$
\end{frame}

\begin{frame}
  This is consistent with the Zilber--Pink Conjecture as for a curve
  $C\subset \cE$ we take the union over all special subvarieties of
  dimension $\le \dim\cE-\dim C - 1 = 0$. 

  So let us work instead in  \alert{fibered product}
  $\cE\times_{Y(2)}\cE=\cE^2$.

  \begin{example}
    The fiber of $\cE^2\rightarrow Y(1)$ above  $\lambda\in
    \IC\ssm\{0,1\}$
    is just the product $\cE_\lambda\times \cE_\lambda=\cE^2_\lambda$ of Legendre
    elliptic curves, given by
    \begin{equation*}
      \bigl\{([x_1:y_1:z_1],[x_2:y_2:z_2]) \in\IP^2\times\IP^2 : y_i^2z_i =
      x_i(x_i-z_i)(x_i-\lambda z_i)\bigr\}      
    \end{equation*}

    This is an abelian surface and we can add and invert points on it.

    We \alert{cannot} add points on distinct fibers $\cE^2_{\lambda}$
    and $\cE^2_\mu$.

    So $\cE$ is a family of abelian two-folds parametrized by $Y(2)$.
    In particular, $\dim \cE = 3$. 
  \end{example}
\end{frame}

\begin{frame}{Relative Manin--Mumford for Curves}
  
  \begin{theorem}[Masser--Zannier]
    % Consider the curve $C\subset\cE^2$ determined by
    % \begin{equation*}
    %   \{((2,y_1),(3,y_2),\lambda) : y_1^2 = 2(2-\lambda) \text{ and }
    %   y_2 = 6(3-\lambda) \}
    % \end{equation*}
    There are at most finitely many $\lambda\in\IC\ssm\{0,1\}$ such
    that
    \begin{equation*}
      (2,\pm \sqrt{2(2-\lambda)}) \text{ \alert{and} } (3,\pm \sqrt{6(3-\lambda)})
    \end{equation*}
    are torsion on the Legendre curve $\cE_\lambda : y^2 =
    x(x-1)(x-\lambda)$. 
  \end{theorem}

  Stoll proved that there are \alert{no} such $\lambda$.

  \begin{theorem}[Relative M--M for Curves, Masser--Zannier]
    Let $C\subset\cE^2$ be an irreducible curve  defined over $\IC$ with
    $\pi(C)=Y(2)$.
    
    If $C\cap \cE^2_{\mathrm{tors}}$ is
    \alert{infinite},  there exists $(a,b)\in\IZ^2\ssm\{0\}$ with
    \begin{equation*}
      C \subset \ker{\bigl((P,Q)\mapsto [a](P)+[b](Q)\bigr)} 
    \end{equation*}    
  \end{theorem} 
\end{frame}

\begin{frame}{A Uniformization}
  How can the Pila--Zannier strategy help with this problem?
  We  first need a uniformization of $\cE(\IC)$.

  The base $Y(2) =\IP^{1}\ssm\{0,1,\infty\}$ is hyperbolic, its
  universal cover is given by  $\lambda\colon \IH\rightarrow
  \IC\ssm\{0,1\}$, a cousin of Klein's $j$-function, with

  \begin{equation*}
        \lambda\left(\frac{a\tau+b}{c\tau +d}\right) =
        \lambda(\tau)\quad\text{if}\quad
    \mattt{a}{b}{c}{d}   \equiv \mattt{1}{0}{0}{1} \mod 2.
  \end{equation*}
  
  The universal covering of each $\cE^2_\lambda$ is $\IC^2$.   Together we find
  \begin{equation*}
    \begin{tikzcd}[ampersand replacement=\&,column sep=small, row sep=small] 
      \& \IC^2\times\IH  \arrow{dd}{} \arrow{r}{u} \& \cE^2(\IC)\arrow{dd}{}\\
      \&  \&\\
      \&    \IH \arrow[swap]{r}{\lambda}  \& Y(2)(\IC).
    \end{tikzcd}
  \end{equation*}

\end{frame}

\begin{frame}
  Each fiber $\cE_{\lambda(\tau)}$ of $\cE\rightarrow Y(2)$ is
  uniformized by $\IC\rightarrow \cE_{\lambda(\tau)}(\IC)$ with kernel
  $\IZ+\tau\IZ$.
  We also obtain a relative fundamental domain
  \begin{equation*}
     \left\{(z_1,z_2,\tau) \in
    \IC^2\times \cF_{\lambda} \text{ with }z_i=a_i+b_i\tau\text{
      where }a_1,b_1,a_2,b_2\in [0,1]  \right\}. 
  \end{equation*}

  We do a real change of coordinates,
  $$ (a_1+\tau b_1,a_2 + \tau
  b_2,\tau)\leadsto (a_1,b_1,a_2,b_2,\tau).$$
  and get a uniformizing map $u\colon\IR^4 \times \IH\rightarrow\cE(\IC^2)$
  and a fundamental domain
  \begin{equation*}
    \cF = [0,1]^4\times \cF_{\lambda}. 
  \end{equation*}

  Let $C$ be an irreducible curve in $\cE^2$ defined over $K$. Then
  \begin{equation*}
    X = u|_{\cF}^{-1}(C(\IC)) \subset \cF
  \end{equation*}
  is definable in $\IRanexp$ and of dimension $2$. 
\end{frame}


\begin{frame}{Semi-rational Points}
  
  \begin{equation*}
    X = u|_{\cF}^{-1}(C(\IC)) \subset \cF
  \end{equation*}

  Let $(P,Q) \in C(\overline K)$ where $(P,Q)$ has finite order $N$
  in $\cE_\lambda$. 
  These points lift to $(a_1,b_1,a_2,b_2,\tau)\in
  ([0,1)^4\cap\IQ^4)\times\IH$.

  
  \begin{itemize}
  \item  We need a lower bound for $[\IQ(P):\IQ]$ that grows like
    some small but fixed power of $N$.

    Masser's result: $[\IQ(P):\IQ] \ge c(A) N^{\delta(A)}$ if
    $P\in A_{\mathrm{tors}}$ has order $N$. 

    New difficulty: the
    elliptic curve $\cE_{\lambda}$ \alert{varies}. How does
    $c(\cE_{\lambda})$ depend on $\lambda$?
    
  \item The point $(a_1,b_1,a_2,b_2,\tau)$ constructed form $P$ is
    most likely
    \alert{not} a rational point. Only the first four
    coordinates $(a_1,b_1,a_2,b_2)$ are rational
    (torsion condition).
  \end{itemize}
\end{frame}

\begin{frame}{A Lower Bound for the Degree}
  For elliptic curves a result of David makes Masser's bound quantitative.
  
  \begin{theorem}[David]
    There exists $c>0$ with the following. 
    Let $K$ be a number field
    $$E=\cE_\lambda: y^2=x(x-1)(x-\lambda)    \text{ with }
    \lambda\in K\ssm\{0,1\}.$$
    If $P\in
    E(K)$ has finite order $N$, then
    \begin{equation}
      \label{eq:davidKQlb}
      [K:\IQ] \ge c \frac{N^{1/2}}{\alert{\max\{1,\log H(\lambda)\}}}.
    \end{equation}    
  \end{theorem}

  Luckily, using a theorem of \alert{Silverman}, we may assume that
  $H(\lambda)$ is bounded in our case.
  
%   \begin{theorem}[Silverman]
%     Let $V\subset \cE^2$ be a curve defined over $\IQbar$. There exists 
%     $c(V)>0$ such that if $P\in V\cap\mathcal{E}^2_{\mathrm{tors}}$  has
%     order $>c(V)$, then $\pi(P)\in \IQbar\ssm\{0,1\}$ and $H(\pi(P))\le c(V)$. 
%   \end{theorem}

% \e
\end{frame}


\begin{frame}{Semi-rational Pila--Wilkie}
  Let $X\subset \IR^m\times\IR^n$
  be definable in an o-minimal structure. 
  We have points in this set where the
  first $m$ coordinates are rational, but the final $n$ most likely not.

  Let $\pi_1\colon\IR^m\times\IR^n\rightarrow\IR^m$ and
  $\pi_2\colon\IR^m\times\IR^n\rightarrow\IR^n$ be projections.

  \begin{theorem}
    Let $\epsilon>0$, there exists $C(X,\epsilon)$ with the following. 
    Let $T\ge 1$ and let $\Sigma\subset X$ be a subset containing
    \alert{semi-rational points}, \textit{i.e.},
    $$
    \#\pi_2(\Sigma) > c(X,\epsilon) T^\epsilon \text{ and }
    (x,y)\in\Sigma\Rightarrow x\in \IQ^m \text{ and }H(x)\le T
    $$  
    There is a continuous and definable function $\beta\colon
    [0,1]\rightarrow X$ with
    \begin{center}
      $\pi_1\circ\beta$ is real semi-algebraic and
      $\pi_2\circ\beta$ is non-constant.  
    \end{center}  
  \end{theorem}
\end{frame}

\begin{frame}{Putting Everything Together}
  The strategy is now familiar. Suppose $C\subset\cE$ is defined over
  $\IQ$. 
  \begin{equation*}
    X = u|_{\cF}^{-1}(C(\IC)) \subset \cF \quad\text{(defined in $\IRanexp$)}.
  \end{equation*}

  We assume that $C\subset\cE^2$ contains infinitely many torsion points.

  Let $P=(x_1,y_1,x_2,y_2,\lambda)\in C(\IC)$
  be of sufficiently large order $N$.
  Then $u(z) = P$
  for some $$z=(a_1,b_1,a_2,b_2,\tau)\in X\cap ([0,1]^4\cap\IQ^4)\times \IH$$
  where $H(a_1,b_1,a_2,b_2)\le N$.


  If $\sigma\in \mathrm{Gal}(\IQbar/\IQ)$, then $\sigma(P) =
  (\sigma(x_1),\ldots,\sigma(y_2),\sigma(\lambda))$  again lies on
  $C\subset\cE^2$ and has finite order $N$.
  $$\leadsto z_\sigma =
  (\underbrace{a_{1,\sigma},b_{1,\sigma},a_{2,\sigma},b_{2,\sigma}}_{\text{Height}
    = N},\tau_\sigma)
  \in X\cap ([0,1]^4\cap\IQ^4)\times \IH$$
\end{frame}

\begin{frame}

  Take for $\Sigma$ the set of $z_\sigma$.
  \begin{equation*}
    \Rightarrow \left\{
      \begin{array}{l}
        \pi_1(z_\sigma) =
        (a_{1,\sigma},b_{1,\sigma},a_{2,\sigma},b_{2,\sigma})\in\IQ^4
        \text{ of height $N$},\\
   \pi_2(z_\sigma)=\tau_\sigma  \text{ varies over $\gg [\IQ(P):\IQ]$ values.}
      \end{array}\right.
  \end{equation*}
  
  From David's and Silverman's Theorem we get   $[\IQ(P):\IQ]\gg N^{1/2}$.

  Semi-rational Pila--Wilkie applies with $T=N$.
  
  We find a definable curve  in $X$ whose
  first four components are real semi-algebraic and the rest is
  non-constant.
  
  There is a suitable Ax--Lindemann--Weierstrass Theorem for this too.
  But in low dimension Masser and Zannier follow an \textit{ad hoc}
  using monodromy properties of the family $\cE\rightarrow Y(2)$. The
  conclusion is that
  $$ C \subset \ker{(P,Q)\mapsto [a](P)+[b](Q)}$$
  for some
  $(a,b)\in\IZ^2\ssm\{0\}$. This is precisely the conclusion we want.
\end{frame}

% \begin{frame}
%   We can pass to an $m$-th fibered power of $\cE$ for $m\in\IN$. 
%   \begin{definition}  
%     \begin{enumerate}
%     \item [(i)] The $m$-fold \alert{fibered power} of $\cE\rightarrow Y(2)$
%       is denoted by $\cE^m = \cE\times_{Y(2)}\cdots\times_{Y(2)}\cE$.
%       Let $\pi\colon\cE^m\rightarrow Y(2)$ be the natural projection.
%     \item[(ii)] For $a=(a_1,\ldots,a_m)\in\IZ^m$ we have a morphism
%       $\varphi_a \colon \cE^m\rightarrow \cE$ over $Y(2)$
%       determined  by $P\mapsto
%       [a_1](P_1)+\cdots +[a_m](P_m)$. We also defined $H_a =
%       \{(P_1,\ldots,P_m) \cE^m(\IC) : \varphi_a(P_1,\ldots,P_m)=0\}$. 
%     \item[(iii)]  We write $\cE^m_{\mathrm{tors}} = \bigcup_{\lambda\in Y(2)(\IC)}
%       \cE^m_{\mathrm{tors}}$; these will be the \alert{special points} of $\cE^m$.
%     \end{enumerate}
%   \end{definition}
% \end{frame}

% \begin{frame}
  
% \end{frame}

\begin{frame}
  \begin{center}
    Thanks for your attention. See you next week at Hector's
    lectures. 
  \end{center}
\end{frame}


\end{document}
