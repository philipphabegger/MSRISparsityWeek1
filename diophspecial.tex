\chapter{Diophantine Equations and Special Points}

\section{Overview}

In this chapter we consider two classes of special points. First,
points of finite order in $\IG_m^2(\IC)=(\IC^\times)^2$ and their distribution on
algebraic curves in $\IG_m^2$; here $\IG_m$ denotes the multiplicative
group. This will lead us to the
Ihara--Serre--Tate Theorem and later to the Manin--Mumford Conjecture.
Second, we look a moduli problems. Special points here are tuples of
$j$-invariants of elliptic curves with complex multiplication. This
will lead us to the Andr\'e--Oort Conjecture, but more specifically to
a theorem of Andr\'e and Edixhoven that treats the first interesting
case. 

\section{The Ihara--Serre--Tate Theorem}
\label{sec:ist}

Here's a quote from a paper of Serge Lang from
1965~\cite{Lang:Division}.

\begin{displayquote}
  ``A few years ago, \textsc{Mumford} asked me the following question:
  If a curve in its Jacobian contains infinitely many points of finite
  period, is the curve of genus $1$? The same question arose in
  \textsc{Manin's} investigation of the \textsc{Picard-Fuchs}
  equations [\ldots]''
\end{displayquote}
And so was born the Manin--Mumford Conjecture. The conjecture was
proved by Raynaud~\cite{Raynaud:MM} and it will be a guiding principle of this
course. 

Before moving on to Jacobians let us attempt to translate this question
from the world of jacobians to that  of the algebraic group $\IG_m^2$. By
curve we consider an irreducible algebraic curve $C\subset
\IG_m^2$ defined over $\IC$. Then $C$ is given by the zero set of
an almost uniquely determined polynomial $P\in \IC[X,Y]$.
In fact, it is more natural to work with the ring of
Laurent polynomials $\IC[X^{\pm 1},Y^{\pm 1}]$, this is the ring of
regular functions of $\IG_m^2$.
What is the analog of the points of finite period in $\IG_m^2$?
These are just the pairs of roots of unity
$$ \IG^2_{m,\mathrm{tors}}= \mu_\infty^2  \text{ where
}\mu_\infty = \{\zeta\in\IC : \text{there is $n\in\{1,2,3,\ldots\}$
  with }\zeta^n=1 \}.$$
Finally, we need to translate ``genus'' to this setting. First let us
look at some examples.

\begin{example}
  \begin{itemize}
  \item [(i)] Consider the curve $C$ defined by $P = X+Y-1$. If
    $(z,w)\in(\IC^\times)^2$ has finite order with $P(z,w)=0$, then
    both $z$ and $1-z=w$ lie on the unit circle $S^1= \{z\in\IC :
    |z|=1\}\supset\mu_\infty$. 
    Considering a picture we find exactly two possibilities    
    $$
    (z,w) \in \left\{ (e^{2\pi i/6},e^{-2\pi i/6}),(e^{-2\pi
        i/6},e^{2\pi i/6})\right\}.
    $$
    So $C\cap \IG^2_{m,\mathrm{tors}}$ has two elements.

  \item[(ii)] Consider the curve $C$ defined by
    $P=X^{2021}Y^{-2022}-1$. Then
    $P(z,w)=0$ for all $(z,w)\in \{(e^{4044\pi iq},e^{4042\pi i
      q}) : q\in\IQ\}$.
    So here we have infinitely many solutions in $\mu^2$.
  \end{itemize}
\end{example}

The second example suggests that curves of large degree may contain
infinitely many torsion points of $\IG_m^2$. But observe that the zero
set of $X^{r}Y^{s}-1$ in $\IG_m^2$ determines an algebraic subgroup for all
$r,s\in\IZ$. 

\begin{exercise}
  \label{exer:infinitemany}
  Let $\gamma\in \IC^\times$ and $(r,s)\in\IZ^2\ssm
  \{0\}$.  Show that $P=X^rY^s-\gamma$ is irreducible in $\IC[X^{\pm
    1},Y^{\pm 1}]$ if $\gcd(r,s)=1$. Show that its zero set $Z(P)$ is a coset in
  $\IG_m^2(\IC)$. Show that $Z(P)$ contains a point in
  $\IG^2_{m,\mathrm{tors}}$ if and only if $\gamma$ is a root of unity. 
\end{exercise}

\begin{definition}
  An algebraic curve in $\IG_m^2$
  is called \emph{special} or a \emph{torsion coset}
  if it is the translate of an algebraic subgroup by
  a point of finite order. 
\end{definition}

We have the following classical characterization of special curves,
presented here without proof. For a reference see Chapter 3.2~\cite{BG}.

\begin{lemma}
  \label{lem:specialGm2}
  If $Z(P)$ is special, then up-to a monomial factor
  $P=X^rY^s-\zeta$ where $(r,s)\in\IZ^2\ssm\{0\}$
  and $\zeta$ a root of unity. 
\end{lemma}

In the first example, the set $\{(z,w) \in S^1\times S^1 : z+w=1\}$
was finite. Write $Z(P) = \{(z,w) \in (\IC^\times)^2 : P(z,w)=0\}$. If
$Z(P)\cap (S^1)^2$ is finite, then so is $Z(P) \cap \IG^2_{m,\mathrm{tors}}$.

\begin{exercise}
  \label{exer:toralpoly}
  Find a non-zero polynomial $P\in \IC[X^{\pm 1},Y^{\pm 1}]$ such that
  $Z(P)$ is \emph{not} special but with $\{(z,w)\in S^1 \times S^1 :
  P(z,w)=0\}$ uncountable infinite.  
\end{exercise}
%% Hint: Blaschke products

Lang's paper contains a proof of the follow theorem which he
attributes to Ihara, Serre, and Tate.

\begin{theorem}
  \label{thm:ist}
  Let $C\subset \IG_m^2$ be an irreducible algebraic curve.
  Then
  \begin{equation*}
    C \cap \IG^2_{m,\mathrm{tors}}\text{ is infinite}\quad\Longleftrightarrow\quad \text{$C$ is special}. 
  \end{equation*}
\end{theorem}

The direction ``$\Longleftarrow$'' is the content of
Exercise~\ref{exer:infinitemany} combined with
Lemma~\ref{lem:specialGm2}. We will prove ``$\Longrightarrow$'' follow
the approach of Tate (which is similar to Serre's).

\begin{exercise}
  Read Ihara's proof~\cite{Lang:Division} of Theorem~\ref{thm:ist}.
\end{exercise}

We we split the proof sketch up into several steps. We assume that
$C=Z(P)$ intersects $\IG^2_{m,\mathrm{tors}}=\mu_\infty^2$
in an infinite set where $P\in
\IC[X^{\pm 1},Y^{\pm 1}]$ is irreducible.
Our aim is to show that $C$ is special, in other words, $P$ is up-to a
monomial factor of the form $X^rY^s-\zeta$ with $\zeta$ a root of
unity. 

The theorem is ultimately of arithmetic nature and Tate's proof uses
input from number theory. The basic observation is that
$\mathrm{G}_{m,\mathrm{tors}}\subset
(\IQbar^\times)^2$ with $\IQbar$ the algebraic closure
of $\IQ$ in $\IC$.

\subsubsection{Reduction to the Case $P \in \IQbar[X^{\pm
    1},Y^{\pm 1}]$.}

We show that $P$ lies in $\IQbar[X^{\pm 1},Y^{\pm 1}]$ up-to
multiplication by a non-zero complex number. To do this we may assume
that $P$ is a polynomial of degree $d\ge 1$.

Consider a polynomial $Q\in
\IQbar[X,Y]$ of degree $e\ge 1$ whose coefficients are unknowns. So
$Q$ has a total of $N = \frac 12 e(e+1)$ coefficients. 
By hypothesis there are infinitely many distinct
$\zeta_1,\zeta_2,\ldots \in Z(P)\cap (\IC^\times)^2$.
The equations $Q(\zeta_1) = \cdots = Q(\zeta_{N-1})=0$
impose $N-1$ linear conditions on the $N$ coefficients of
$Q$.
This is an underdetermined system of homogeneous linear equations. So
there exists $Q\in \IQ[X,Y]$ with $Q\not=0$  and $\deg Q \le e$ that vanishes at
$\zeta_1,\ldots,\zeta_{N-1}$. But $P$ vanishes at these same points.
If $P$ does not divide $Q$, then B\'ezout's Theorem implies
\begin{equation*}
  \frac 12 e(e+1)-1=  N-1 \le \text{number of common zeros of $P$ and $Q$} \le de.
\end{equation*}
As the right-hand side grows quicker in $e$ than the left-hand side we obtain a
contradiction for $e$ large enough. So $P$ is a divisor of $Q$ and
therefore must have coefficients in $\IQbar$ by multiplying by a non-zero
scalar. 

So $P \in K[X^{\pm 1},Y^{\pm 1}]$ where $K$ is a number field that we may assume is
Galois over $\IQ$. To keep the notation light we will assume for
simplicity $K=\IQ$. The general case follows along similar lines with
some extra bookkeeping.

\subsubsection{Roots of Unity and Their Galois Conjugates}
\label{sec:rootsof1}
Each root of unity $\zeta$ has an order, the minimal
$n\in\IN=\{1,2,3,\ldots\}$
with
$\zeta^n=1$. The $\IQ$-minimal polynomial $\phi_n\in\IQ[X,Y]$
of $\zeta$ is a divisor of
$X^n-1$. Some important facts:

\begin{itemize}
\item The roots of $\phi_n$ are $\{\zeta^a : a\in
  (\IZ/n\IZ)^\times\}$. In particular, $\phi_n$ depends only on $n$
  and $\IQ(\zeta)/\IQ$ is a Galois extension. 
\item The degree $\deg \phi_n = \varphi(n)= n \prod_{p\mid n}
  \left(1-\frac 1p\right)$, where the product runs over prime divisors
  of $n$. Here $\varphi$ is also called Euler's totient function.
\end{itemize}

\begin{crucial}
  If $(\zeta,\xi)\in \mu_\infty^2$ and $P(\zeta,\xi)=0$ and if
  $\sigma\in \mathrm{Gal}(\IQbar/\IQ)$, then
  $P(\sigma(\zeta),\sigma(\xi))=0$.

  The action of the Galois groups on $\mu_\infty$ is by raising to a
  power. For any $a\in\IZ$ coprime to the order of $(\zeta,\xi)$  
  we find $P(\zeta^a,\xi^a)=0$. 
\end{crucial}

Given a single point in $Z(P)\cap\mu_\infty^2$ the action of the
Galois group producs more points. How many more? The number is
$\varphi(n)$ where $n$ is the order of $(\zeta,\xi)$. 

\begin{lemma}[Large Galois Orbit Lemma]
  \label{lem:lgo}
  We have $\varphi(n) \ge (n/2)^{1/2}$. 
\end{lemma}
\begin{proof}
  Factor $n=p_1^{e_1}\cdots p_g^{e_g}$ into primes $p_1<p_2<\cdots <
  p_g$ and $e_1,\ldots,e_g\in\IN$. Then $\varphi(n) = \prod_{i=1}^g
  p_i^{e_i-1}(p_i-1)$. Then $p_i^{e_i-1}(p_i-1)\ge p_i^{e_i/2}$ except
  if $p_i^{e_i}=2$, \textit{i.e.}, $p_i=2,e_i=1$, in which case we
  need the factor $2^{-1/2}$. 
\end{proof}


\begin{lemma}[Galois Homothety Lemma]
  Say $(\zeta,\xi)\in \mu_\infty^2$ has order $n\ge 2$
  with $P(\zeta,\xi)=0$. There exists a prime number $\ell=O((\log
  n)^2)$ with $P(\zeta^\ell,\xi^\ell)=0$.
  Moreover, we can arrange that $\ell \ge c (\log n)^2$ for some
  absolute constant $c>0$. 
\end{lemma}
\begin{proof}
  The Prime Number Theorem states that the number of primes $\le x$ is
  asymptotically equal to $x/\log x$ for $x\ge 2$.
  % We will only need that this number
  % is $\ge \delta x/\log x$, for fixed $\delta >0$. 
  The number of
  distinct prime divisors of $n$ is at most $(\log n)/\log 2$.
  So there is a  prime number $\ell$ that does not divide $n$
  that satisfies $\ell =
  O((\log n)^2)$. By the crucial observation we conclude
  $P(\zeta^\ell,\xi^\ell)=0$.
  We can arrange that $\ell\ge c(\log\ell)^2$ by exploiting the
  asymptotical lower bound in the Prime Number Theorem.
\end{proof}

Now we can put things together. Let us again multiply $P$ by a
monomial so that it becomes a polynomial.
Let $(\zeta,\xi)\in \mu_\infty^2$ have order $n\ge 2$ with
$P(\zeta,\xi)=0$. Let $\ell$ be a prime as in the Galois Homothety
Lemma.
Then
\begin{equation}
  \label{eq:intersectionGm}
  (\zeta,\xi)\text{ lies on the zero set of $P(X,Y)$ and of 
    $P(X^{\ell},Y^{\ell})$}.
\end{equation}
So does any Galois
conjugate by the Crucial Observation.
By the Large Galois Orbit Lemma there are at least
$(n/2)^{1/2}$ common roots. If $P$ does not divide $P(X^\ell,Y^\ell)$,
then B\'ezout's Theorem implies
\begin{equation*}
  n^{1/2} = O((\log n)^2)
\end{equation*}
and this means that $n$ is bounded. But by assumption we may take $n$
as large as we want and therefore $P$ must divide $P(X^\ell,Y^\ell)$.

\begin{exercise}
  Show that $X+Y-1$ does not divide  $X^\ell + Y^\ell-1$ for any prime
  $\ell$. 
\end{exercise}

\subsubsection{Concluding the Proof}
\label{subsub:functrans}

We return to a more geometric point of view in this final set.
If $C=Z(P)$ contains a torsion point of large enough order $n$, then
$P \mid P(X^\ell,Y^\ell)$ for some prime $\ell$.
Geometrically, $C \subset [\ell]^{-1}(C)$ where $[\ell]$ denotes the $\ell$-th
power endomorphism of $\IG_m^2$.
If $\bfzeta\in (\IC^\times)^2$ with $\bfzeta^\ell=1$, then the
translation $\bfzeta C\subset [\ell]^{-1}(C)$. These $\bfzeta C$ are
irreducible components of $[\ell]^{-1}(C)$. The number of $\bfzeta$ is
$\ell^2$. But the degree of $[\ell]^{-1}(C)$ is $\ell \deg C$. As soon
as $\ell > \deg C$ the Pigeonhole Principle provides distinct
$\bfzeta',\bfzeta''\in \mu_\infty$ with $\bfzeta' C = \bfzeta''C$.

Hence $\bfzeta C=C$ with $\bfzeta = \bfzeta'{\bfzeta''}^{-1}$ of order
$\ell$.

Let us now check that $C$ is the translate of an algebraic subgroup of
$\IG_m^2$ by a point of finite order. % After translating we may
% assume $1=(1,1)\in C$ and then it suffices to show that $C$ is a
% subgroup of $(\IC^\times)^2$.

To this end consider
\begin{equation*}
  G = \bigcap_{Q\in C} Q^{-1} C.
\end{equation*}

As a set, $G$ is an algebraic subgroup of $\IG_m^2$. But it is also Zariski
closed and of dimension at most $1$. So it is either finite or a
curve. Observe that $\bfzeta C = C$ implies $\bfzeta\in G$. As
$\bfzeta$ has order $\ell$. Although $\ell = O((\log n)^2)$, the
Galois Homothety lemma also implies that $\ell$ is unbounded as $n$
grows.
Therefore, $G$ must be
infinite and thus equal to a translate of $C$.

Therefore, $C$ is special, and this completes the proof of
Theorem~\ref{thm:ist}.

\begin{exercise}
  Prove that
  $\bigcap_{Q\in C} Q^{-1} C = \{Q\in (\IC^\times)^2 : QC=C\}$.
  This group is called the \emph{stabilizer} of $C$.
\end{exercise}


% So $P(\zeta_1 X,\zeta_2 Y)$ and $P(X,Y)$ define the same curve.
% These two are equal up-to to a scalar if we assume, as we may that $P$ is a polynomial.

% If $a_{ij}$ denotes a non-zero coefficient of $P$, then comparing
% coefficients gives
% $a_{ij}\zeta_1^i \zeta_2^j = \lambda a_{ij}$ for some $\lambda\not=0$
% independent of $i,j$. As there are at least two monomials we find
% $\zeta_1^{i}\zeta_2^j = \zeta_1^{i'}\zeta_2^{j'}$ with $(i,j)\not=(i',j')$.

% Then $\prod_{\sigma \in \mathrm{Gal}(K/\IQ)}
% \sigma(P)$ has coefficients in $\IQ$ and its zero set has infinitely
% many points in common with $(\IC^\times)^2_{\mathrm{tors}}$.

\section{The Modular Side}

Roots of unity are precisely images of rational numbers under the
holomorphic map $x\mapsto e^{2\pi i x}$. We often take for granted
that this is a very particular thing: this holomorphic map takes
rationals to algebraic numbers. Moreover, the deep Kronecker--Weber
Theorem from Class Field Theory states that any finite Galois
extension of $\IQ$ with abelian Galois group is contained in the field
generated by a root of unity. So the map $x\mapsto e^{2\pi i x}$ is
intricately connected to the abelian extensions of $\IQ$.

To understand abelian extensions of an imaginary quadratic number
$F=\IQ(\sqrt{-D})$ we need elliptic curves with complex
multiplication.

Let us regard an elliptic curve $E$ defined over $\IC$ from the
complex analytic point of view. Thus $E(\IC)$ carries the structure of
a complex Lie group and there is a holomorphic uniformation map
\begin{equation*}
  u \colon \IC \rightarrow E(\IC)
\end{equation*}
whose kernel equals rank $2$ discrete subgroup of $\IC$ and whose
image is all of $E(\IC)$. After a normalization we may assume that
\begin{equation*}
  \Omega = \mathrm{ker}(u) = \IZ+\tau\IZ
\end{equation*}
where $\tau\in \IH = \{z\in \IC : \mathrm{Im}(z)>0\}$. We call
$\Omega$ a period lattice for $E$.
The complex number $\tau$ is uniquely determined up-to the action of
$\mathrm{SL}_2(\IZ)$ on $\IH$ by a \emph{fractional linear transformation}.

What is a fractional linear transformation?
Let $\mattt{a}{b}{c}{d}\in\mathrm{GL}^+_2(\IR) = \{\gamma \in
\mathrm{Mat}_2(\IR) : \det \gamma > 0\}$, then
\begin{equation*}
  \mattt{a}{b}{c}{d} \tau = \frac{a\tau+b}{c\tau +d}
\end{equation*}
where $\tau\in\IH$; this 
is a $\IH\rightarrow\IH$.


Any endomorphism of $E$ is a morphism $f\colon E\rightarrow E$ of
algebraic varieties with $f(0)=0$. Let $\mathrm{End}(E)$ denote all
endomorphisms of $E$. Any $f\in \mathrm{End}(E)$ is
induced by a unique $\IC$-linear map
$\IC\rightarrow \IC$ that maps $\Omega$ to itself. In other words, we
identify
\begin{equation*}
  \mathrm{End}(E) = \{ \alpha\in\IC : \alpha \Omega \subset \Omega \}
  = \{\alpha\in \IC : \alpha\in \Omega, \alpha\tau\in \Omega\}. 
\end{equation*}
Clearly,  $\IZ$ is contained in the right-hand side. For example  $N\in\IN$
correspond to the multiplication-by-$N$ endomorphism $[N](P) =
\underbrace{P+\cdots +P}_{N\text{ times}}$ of $E$.
If $-N\in\IN$, then $[N](P) = -[-N](P)$ and finally $[0](P)=0$.  

\begin{lemma}
  \label{lem:tauimagquad}
  Suppose $\mathrm{End}(E) \supsetneq \IZ$, then
  $\IQ(\tau)=\IQ(\alpha)$
  is imaginary
  quadratic and $\alpha$ is an algebraic integer. 
\end{lemma}
\begin{proof}
  Let $\alpha\in \mathrm{End}(E)$, then there exists $A\in\mathrm{Mat}_2(\IZ)$ with
  \begin{equation*}
    \alpha \left(
      \begin{array}{c}
        1 \\ \tau 
      \end{array}
    \right) = % \left(
      % \begin{array}{cc}
      %   a & b \\ c & d
      % \end{array}\right)
    A\left(
      \begin{array}{c}
        1 \\ \tau 
      \end{array}
    \right).
  \end{equation*}
  Therefore, $\alpha$ is an eigenvalue of $A$; in particular
  $[\IQ(\alpha):\IQ]\le 2$. Moreover, $\matto{1}{\tau}$ lies in the
  kernel of $\alpha I-A$. If $\alpha\not\in \IZ$, then $\alpha
  I-A\not=0$ and thus $\tau \in \IQ(\alpha)$. The lemma follows as
  $\alpha\not\in\IR$. 
\end{proof}

\begin{exercise}
  Show the following statement. If $\tau \in\IH$  with
  $[\IQ(\tau):\IQ]=2$, then $\mathrm{End}(\IC/(\IZ+\tau \IZ))\supsetneq
  \IZ$. 
\end{exercise}

\begin{definition}
  We say that $E$ has \emph{complex multiplication} if
  $\mathrm{End}(E)\supsetneq \IZ$. 
\end{definition}

\begin{definition}
  If $E$ is presented algebraically via an Weierstrass equation $y^2 =
  x^3+ax+b$ with $a,b\in\IC$. Then the \emph{$j$-invariant} $j(E)$ of $E$ equals
  \begin{equation*}
    j(E) = 2^8 3^3 \frac{a^3}{4a^3+27b^2}. 
  \end{equation*}  
\end{definition}

\begin{example}
  The elliptic curve $E$ presented by $y^2=x^3+x$ has the additional
  endormorphism
  $(x,y)\mapsto (-x, i y)$. Its $j$-invariant equals $1728$. The
  complex picture goes as follows, we have $E(\IC)\cong \IC/(\IZ+i
  \IZ)$. 
\end{example}

We come to an important fact:
\begin{theorem}
  Two elliptic curves over $\IC$ are
  isomorphic if and only if their $j$-invariants are equal.
\end{theorem}

As each complex number is the $j$-invariant of an elliptic curve we
find: The $j$-invariant allows us to identify isomorphism classes of
elliptic curves over $\IC$ with the set of complex points of the
affine line. To emphasize this modular property of the affine line we
will sometimes denote it by $Y(1)=\IA^1$, which is standard notation
for a modular curve in a certain infinite family.

\begin{definition}
  The \emph{Klein's modular} $j$-function is the map
  $j\colon\IH\rightarrow\IC$ given by  $j(\tau) =
  j$-invariant of the elliptic curve $\IC/(\IZ+\tau\IZ)$. 
\end{definition}
So $j(\tau)$  invariant under the action of $\mathrm{SL}_2(\IZ)$.
% So we obtain a
% function $j\colon \IH\rightarrow \IC$ determined by $j(\tau) =
% j(\IC/(\IZ+\tau\IZ))$ for all $\tau\in\IH$; it is sometimes called
% Klein's $j$-functions.
\textit{I.e.}, 
\begin{equation*}
  j\left(\mattt{a}{b}{c}{d} \tau\right) = j\left(\frac{a\tau + b}{c\tau +d}\right) =
    \tau 
\end{equation*}
for all $\tau\in\IH$ and all $\mattt{a}{b}{c}{d}\in
\mathrm{SL}_2(\IZ)$.
In particular,  $j(\tau)=j(\tau+1)$ and $j(\tau) =
j(-1/\tau)$. 


The function equation $j(\tau)=j(\tau+1)$ shows that $j$ depends only
on $q=e^{2\pi i \tau}$. It is sometimes useful to consider $j$ as a
function in $q$.

It turns out that much more is true: Klein's $j$-function is
holomorphic on $\IH$ and meromorphic at infinity.
So $j$ is a modular function of weight $0$.
It has a famous 
a Taylor series in $q$, the first few terms\footnote{There is some
  dispute about where the constant terms $744$ is the ``right'' one.} of which are 
\begin{equation*}
  j(q) = \frac 1q + 744 + 196884q +\cdots. 
\end{equation*}
Moreover, any modular function that is meromorphic at $\infty$ is a
rational function in $j$.

\begin{definition}
  \label{def:specialY1}
  A \emph{special point} of $Y(1)$ is the $j$-invariant of an elliptic curve
  with complex multiplication. Equivalently: the set of special points
  of $Y(1)$ is $\{j(\tau) : \tau\in\IH \text{ and }[\IQ(\tau):\IQ]=2\}$. 
\end{definition}

The set of special points is at most countable infinite and it is
indeed infinite since $j$ is non-constant.

The following theorem extends the analogy between roots of unity and
special points of $Y(1)$. It is a consequence of class field theory. 
\begin{theorem}
  \label{thm:galoisspecial}
  Let $z\in \IC$ be a special point of $Y(1)$ with $z=j(\tau)$ where
  $\tau\in \IH$. Let $F=\IQ(\tau)$.
  \begin{enumerate}
  \item [(i)] Then $z$ is an algebraic integer.
  \item[(ii)] The field $F$ is
    imaginary quadratic and   $\mathrm{End}(E)$ is an order
    in $F$.
  \item[(iii)] The extension $F(z)/F$ is Galois with abelian Galois
    group.
  \item[(iv)] If $\mathrm{End}(E)$ is the maximal order in $F$,
    \textit{i.e.}, the ring of integers of $F$, then
    the class group $H_F$ of $F$ is isomorphic to
    $\mathrm{Gal}(F(z)/F)$ via the Artin homorphism.
    Moreover, $F(z)$ is the Hilbert Class Field of $F$. 
  \end{enumerate}
\end{theorem}

The Conjecture of Andr\'e--Oort is the analog (and generalization) of
the Theorem of Ihara--Serre--Tate in this modular side.

For this let us just consider $Y(1)^2$ instead of the algebraic group $\IG_m^2$.
\begin{definition}
  \label{def:specialY1m}
  The set of special points of  $Y(1)^m$ is
  $$Y(1)^m_{\mathrm{special}}=\{(z_1,\ldots,z_m) :
  z_i\in Y(1)(\IC)\text{ is special for all $i$} \}.$$
\end{definition}

Which algebraic curves in $Y(1)^2$ contain infinitely many special
points? There are some immediate classes.

\begin{example}
  \begin{enumerate}
  \item[(i)] The diagonal  $\{(z,z) : z\in\IC\}$ contains infinitely
    many special points of $Y(1)^2$. 
  \item [(ii)]  Let $j_1$ be a special point of $Y(1)$.  Then the curves
    $\{j_1\}\times Y(1)$ and $Y(1)\times \{j_1\}$ contain infinitely many
    special points of $Y(1)^2$. 
  \end{enumerate}
\end{example}

These are first candidates of special curves in $Y(1)^2$. 
But there is collection of non-trivial.

\begin{definition}
  An \emph{isogeny} $f\colon E_1\rightarrow E_2$ of elliptic curves,
  both defined over $\IC$, is a non-constant morphism of algebraic
  varieties with $f(0)=0$. If $E_i$ is represented by
  $\IC/(\IZ+\tau_i\IZ)$, for $i\in \{1,2\}$, then $f$ descends from the
  multiplication-by-$\alpha$ map $\IC\rightarrow\IC$ where
  $\alpha\in\IC^\times$ satisfies $\alpha(\IZ+\tau_1\IZ)\subset
  \IZ+\tau_2\IZ$.
  We call $f$ a \emph{cyclic isogeny}, if $\mathrm{ker}(f)$ is a cycle
  group.
\end{definition}

An isogeny $f$ is necessarily a finite morphism
that induces a homomorphism $E_1(\IC)\rightarrow E_2(\IC)$ 
with a finite kernel $\mathrm{ker}(f)$.
The degree $\deg f$ of $f$ is the order of $\mathrm{ker}(f)$. 


\begin{deflemma}
  \label{deflem:modtranspoly}
  Let $N\in\IN$. The set
  \begin{equation*}
    \left\{ (j(\tau_1),j(\tau_2)) : \text{there is a cyclic isogeny 
        $\IC/(\IZ+\tau_1\IZ)\rightarrow \IC/(\IZ+\tau_2\IZ)$ of degree $N$}\right\}
  \end{equation*}
  is an irreducible algebraic curve in $Y(1)^2$, \textit{i.e.}, it is
  the zero set of an irreducible polynomial $\Phi_N\in\IC[X,Y]$.
  Moreover, we have $\Phi_N\in\IZ[X,Y]$. We call $\Phi_N$
  \emph{modular transformation polynomial (of level $N$)}.
\end{deflemma}

\begin{example}
  For $N=1$ we recover the diagonal $\{(z,z) : z\in\IC\}$, so $\Phi_1
  = X-Y$. 
  Already for $N=2$ the polynomial is quite complicated. Indeed,
  \begin{alignat*}1
    \Phi_2 &= 
    X^3 - X^2Y^2 + 1488X^2Y - 162000X^2 + 1488XY^2+ 40773375XY +
    8748000000X + \\
    &Y^3 - 162000Y^2 + 8748000000Y -157464000000000.    
  \end{alignat*}

  For general $N$, the degree of $\Phi_N$ as a polynomial in $X$
  satisfies
  \begin{equation*}
    \deg_X \Phi_N = N\prod_{p\mid N}\left(1+\frac 1p\right). 
  \end{equation*}
  Moreover, $\Phi_N(X,Y) = \Phi_N(Y,X)$ for $N\ge 2$. And
  $\deg_X \Phi_N = \deg_Y \Phi_N$ for all $N\ge 1$. 

  The growth of the coefficients of $\Phi_N$ is described precisely in
  work of Cohen~\cite{Cohen}. There it is proved that the
  coefficients of $\Phi_N$ grow superexponentially in $N$. 
\end{example}

\begin{lemma}
  Let $N\in\IN$, then $\Phi_N(j(\tau),j(N\tau))=0$ for all
  $\tau\in\IH$. 
\end{lemma}
\begin{proof}
  The multiplication-by-$N$ map $\IC\rightarrow\IC$ factors through
  $\IC/(\IZ+\tau\IZ) \rightarrow \IC/(\IZ+N\tau\IZ)$.
  The kernel of the latter homomorphism is $\{\frac kN \IZ
  +(\IZ+\tau\IZ) : k\in \{0,\ldots,N-1\}\}$ which is cyclic of order
  $N$.  
\end{proof}

Suppose $j(\tau)$ is a special point of $Y(1)$, then so is $j(N\tau)$,
see Definition~\ref{def:specialY1}.

We conclude that for all $N\ge 1$, the zero set $Z(\Phi_N)\subset
Y(1)^2$ of $\Phi_N$ contains infinitely many special points.

\begin{definition}
  An algebraic curve  $C\subset Y(1)^2$ is called a \emph{special curve} of
  $Y(1)^2$ 
  \begin{enumerate}
  \item [(i)] if $C = \{z\}\times Y(1)$ for a special point $z$, or
  \item [(ii)] if $C =  Y(1)\times \{z\}$ for a special point $z$, or
  \item[(iii)] if $C=Z(\Phi_N)$ for some $N\in\IN$. 
  \end{enumerate}
\end{definition}

The content of the Andr\'e--Oort Conjecture is the converse of the
observations above.

\begin{theorem}[Andr\'e--Oort for $Y(1)^2$, Andr\'e~\cite{Andre:AO},
  Edixhoven~\cite{Edixhoven:AO} under GRH, Pila~\cite{Pila:AO} for
  subvarieties of $Y(1)^m$, effective versions by
  Bilu--Masser--Zannier~\cite{} and K\"uhne~\cite{}]
  \label{thm:ao}
  Let $C\subset Y(1)^2$ be an irreducible curve. Then
  \begin{equation*}
    C \cap Y(1)^2_{\mathrm{special}}\text{ is
      infinite}\quad\Longleftrightarrow\quad \text{$C$ is a special
      curve of $Y(1)^2$}. 
  \end{equation*}  
\end{theorem}

There are many different proofs of this theorem. We will later get to
know Pila's approach much better. But first let us review the proof of
Edixhoven. It is similar in spirit for the proof of Serre and Tate of
Theorem~\ref{thm:ist}. In addition it is conditional on the Generalized
Riemann Hypothesis (GRH).

As we have seen above, the deep part of Theorem~\ref{thm:ao} is the
implication
``$\Longrightarrow$''.

\subsubsection{Reduction to the Case $P \in \IQbar[X,Y]$.}
This step works just as in the proof of Theorem~\ref{thm:ist}
presented above.

For simplicity we will assume that $P\in \in\IQ[X,Y]$ is irreducible
in $\IC$. 

\subsubsection{Special Points and their Galois Conjugates}

Let $(z,w)$ be special point of $Y(1)^2$ with $P(z,w)=0$.

Again, for any $\sigma\in \mathrm{Gal}(\IQbar/\IQ)$ the conjugate
$(\sigma(z),\sigma(w))$ lies in $Z(P)$.

How can we measure the orbit of our point? Observe that the
endomorphism ring attached to $z$ is an order $\mathcal{O}$ in an imaginary
quadratic number field $F$. We define the \emph{discriminant of $z$}
to be 
\begin{equation*}
  \Delta_z = \mathrm{Disc}(\mathcal{O}) = [\mathcal{O}_F:\mathcal{O}] \Delta_F
\end{equation*}
where $\mathcal{O}_F$ is the maximal order of $F$, \textit{i.e.}, the
ring of algebraic integers of $F$. 

Let us assume for simplicity that the orders attached to 
both $z$ and $w$ are maximal.\footnote{This assumption is made in
  order to avoid technicalities when dealing with non-Dedekind
  orders.}
We also assume $\Delta =
\Delta_z=\Delta_w$ are equal. \footnote{This second reduction step
  somewhat more subtle.
  Although Andr\'e and Edixhoven's proofs are different, they both use
  this reduction step.}

\begin{lemma}[Large Galois Orbit]
  \label{lem:lgocm}
  We have
  \begin{equation*}
    \# \mathrm{Gal}(\IQ(z,w)/\IQ) \gg_\epsilon
    |\Delta|^{1/4}. 
  \end{equation*}
\end{lemma}
\begin{proof}
  The left-hand side is at least the order of the class group $H_F$ of $F$ by
  Theorem~\ref{thm:galoisspecial}(iv). The Landau--Siegel Theorem
  implies $\# H_F \gg_\epsilon |\Delta|^{1/2-\epsilon}$. 
  The same estimates hold for $w$ and this concludes the proof.
  The lemma follows with the choice $\epsilon =1/4$. 
\end{proof}

What is the analog of the Homothety Lemma? To find this out we need
the analog of multiplication-by-$N$ for $Y(1)^2$. However, there is
no group structure that is compatible with out moduli problem.
However, we do have the Modular Transformation Polynomials $\Phi_N$
and we can use them to transform our curve.

\begin{definition}
  Let $C\subset Y(1)^2$ be an algebraic curve defined over $\IC$. Then set
  \begin{equation*}
    T_N(C) = \{(z,w) : \text{there exists $(z',w')\in C$ with $\Phi_N(z,z')=\Phi_N(w,w')=0$}\}
  \end{equation*}  
\end{definition}

One can show that $T_N(C)$ is possible reducible algebraic curve.
Now we can formulate the Homothety Lemma in the modular case.

\begin{lemma}[Galois Homothety Lemma]
  Suppose the GRH.
  Say $(z,w)\in Y(1)^2$ is a special point as above, so % with $P(z,w)=0$ 
 $\Delta=\Delta_z=\Delta_w$ is a fundamental discriminant.
  There exists a prime number $\ell=O((\log
  |\Delta|)^3)$ and $\sigma\in \mathrm{Gal}(\IQbar/\IQ)$  with $\Phi_\ell(z,\sigma(z))=\Phi(w,\sigma(w))=0$. 
\end{lemma}
\begin{proof}[Proof sketch]
  Let $K$ be the imaginary quadratic field of discriminant $\Delta$.
  Suppose $\ell$ is split prime in $K$, \textit{i.e.}, $\ell\cO_K =
  \mathfrak{p}\mathfrak{q}$ with $  \mathfrak{p},\mathfrak{q}$
  distinct prime ideals of $\cO_K$.
  If we think of $\IC/\cO_K$ as an elliptic curve,
  then $\IC/\cO_K \rightarrow \IC/\mathfrak{p}$ is an isogeny of
  elliptic curves. Its kernel has order $[\cO_K:\mathfrak{p}]=\ell$
  and is hence cyclic. By the theory of complex multiplication of
  elliptic curves, we may reduce to the case where $z$ is
  represented by $\IC/\cO_K$ and $z'\in \IC$ is represented by
  $\IC/\mathfrak{p}$. So $\Phi_{\ell}(z,z')=0$. 
  Moreover, by Class Field Theory $\sigma(z)=z'$ for $\sigma \in
  \mathrm{Gal}(F(z)/F)$ coming from the image of $\mathfrak{p}$ under
  the Artin map. As $F(z)=F(w)$ by hypothesis, the same $\sigma$ works
  for $w$ and produces $\sigma(w)=w'$.

  So the goal is to find a small split prime in $F$. Approximately
  have of all primes split in $F$. So
  \begin{equation*}
    \{\ell \le x : \ell \text{ splits in }F\} = \frac 12 \underbrace{\int_2^x
      \frac{\mathrm{d} t}{\log t}}_{=\mathrm{Li}(x)} + o(\mathrm{Li}(x))
    \quad\text{as}\quad x\rightarrow\infty. 
  \end{equation*}
  But the problem is that $K$ is
  varying and we need a good dependency on $K$ in the error term
  $o(\cdots)$.
  A suitable version of GRH tells us that the error term can be taken
  as
  $\frac 16 \sqrt x (\log|\Delta| + 2\log(x))$ for $x$ large enough.
  The claim follows as $\mathrm{Li}(x)\sim x/\log x$. 
\end{proof}

Let us again put things together. Under the various simplying
assumptions made above (and GRH!) the Galois Homothety Lemma implies $P(z,w)=0, P(\sigma(z),\sigma(w))=0,$
and $\Phi_\ell(z,\sigma(z))=\Phi_\ell(w,\sigma(w))=0$ for some
$\sigma\in\mathrm{Gal}(\IQbar/\IQ)$ and some prime
$\ell = O((\log|\Delta|)^3)$. We are now in a situation reminiscent of (\ref{eq:intersectionGm}), so
\begin{equation*}
  (z,w) \in C \cap T_\ell(C),  
\end{equation*}
indeed $C$ is defined by an integral polynomial and by the definition of
$T_\ell$.
On the other hand, any Galois conjugate of $(z,w)$ is a member of the
intersection $C\cap T_\ell(C)$. The Large Galois Orbit Lemma in the
modular case implies\footnote{"$\ll$" below means that the left-hand side is
      at most the right-hand side up-to a fixed and positive
      multiplicative constant.}
\begin{equation}
  \label{eq:intersectionlb}
  \#|\Delta|^{1/4} \ll \# C\cap
  T_\ell(C).
\end{equation}

Let us apply B\'ezout's Theorem to estimate the cardinality $\# C\cap
T_\ell(C)$. For this we consider $C$ as fixed and $\ell$ as varying.
Some basic degree estimates yields $\deg T_\ell(C) \ll \ell^2$.
Therefore, B\'ezout's Theorem tells us $\#C\cap T_\ell(C)\ll \ell^2$.
But $\ell$ is not too large by the Galois Homothety Lemma, so
$\# C\cap T_\ell(C)\ll (\log|\Delta|)^6$. 
We combine this with (\ref{eq:intersectionlb}) and conclude
\begin{equation*}
  \#|\Delta|^{1/24} \ll \log|\Delta|.
\end{equation*}
As a polynomial always  beats a logarithm  we conclude that $|\Delta|$
is bounded and this means that $(z,w)$ is in a finite set that depends
only on $C$. In other words, $C$ contains at most finitely many
special points of $Y(1)^2$.

But this cannot be correct! We never assumed that $C$ is not a special
curve and we know that special curves contain infinitely many special
points. So the argument above contains a gap. But where is it? It is
in the application of B\'ezout's Theorem, we only get a cardinality
bound if $C\not\subset T_\ell(C)$. This is final step, the functional
Functional Transcendence argument, involves showing that $C\not\subset
T_\ell(C)$ for all sufficiently large primes $\ell$ if $C$ is not a
special curve. We skip Edixhoven's approach to this problem and will
revisit it later after doing some o-minimal geometry. 




\section{Further Reading and Open Problems}

\begin{itemize}
\item Elliptic Curves and Complex Multiplication \cite{Silverman:AEC,Silverman:Adv}
\item Class Field Theory for imaginary quadratic fields \cite{Cox}
\item Elliptic Functions, elliptic curves, complex multiplication \cite{Lang:elliptic}
\end{itemize}


%%% Local Variables:
%%% TeX-master: "main"
%%% End:
