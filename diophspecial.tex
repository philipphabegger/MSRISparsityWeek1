\chapter{Diophantine Equations and Special Points}

\section{Overview}

One of the most classical areas in mathematics, dating back to ancient
times, is the study of solutions of polynomials equations in integer
or rational unknowns. These \textit{diophantine equations} have
provided a rich source of new mathematics in the last 2000 years.
Number fields, \textit{i.e.}, finite extensions of the field of
rational numbers, entered the picture in the 19th century. They are
tools in studying diophantine equations. But it also makes sense to
investigate solutions of diophantine equations with values in number
fields or the ring of integers.

%One theme that we will explore in this week is the observation that
More recently, we began studying solutions of polynomial equations in
a further class of numbers, so-called \textit{special points}. Special
points is not a precise term but rather refers to complex numbers, or
more general points on certain varieties, that have rich arithmetic
properties.

In this one week course we will be interested in three classes of
special points:
\begin{itemize}
\item Roots of unity, \textit{i.e.}, complex numbers of the form
  $e^{2\pi i q}$ with $q$ a rational number. These are precisely the
  points of finite order of the multiplicative group $\IC^\times$. 

\item Consider an elliptic curve $E$ defined over a number field
  $K\subset\IC$. We regarding points in $E(\IC)$ of finite order as
  special points.

\item Finally, to any elliptic curve $E$ defined over $\IC$ we can
  attach its ring of endomorphisms
  $$\mathrm{End}(E) = \{\varphi \colon E\rightarrow E : \varphi\text{
    is a morphism of varieties and }\varphi(0)=0\},$$
  here $0\in E(\IC)$ is the neutral element.\footnote{Addition on
    $\mathrm{End}(E)$ is pointwise addition and multiplication is
    composition. It is a theorem that $\mathrm{End}(E)$ is a
    commutative ring for any elliptic curve defined over a field of
    characteristic $0$.}
  The  structure of $E$ as an algebraic group gives us for free
  multiplication-by-$N$ endomorphisms $[N] \in \mathrm{End}(E)$ for
  all $N\in\IZ$. So $\mathrm{End}(E) \supset \IZ$.
  We say that $E$ has complex multiplication if
  $\mathrm{End}(E)\not=\IZ$.
  Such an elliptic curve curve corresponds to a \textit{special point}
  on the parameter space of all elliptic curves.   
\end{itemize}

As we will see, all three cases have an important similarity: special
points are images of certain algebraic points under an analytic map.
We will discuss all three cases in more detail during the next week. 

\section{The Ihara--Serre--Tate Theorem}

Here's a quote from a paper of Serge Lang from
1965~\cite{Lang:Division}.

\begin{displayquote}
  ``A few years ago, \textsc{Mumford} asked me the following question:
  If a curve in its Jacobian contains infinitely many points of finite
  period, is the curve of genus $1$? The same question arose in
  \textsc{Manin's} investigation of the \textsc{Picard-Fuchs}
  equations [\ldots]''
\end{displayquote}
And so was born the Manin--Mumford Conjecture.

Before moving on to Jacobians let us attempt to translate this question
from the world of jacobians to that  of the algebraic group $(\IC^\times)^2$. By
curve we consider an irreducible algebraic curve $C\subset
(\IC^\times)^2$ defined over $\IC$. Then $C$ is given by zero set of
an almost uniquely determined polynomial $P\in \IC[X,Y]$.
In fact, it is more natural to work with the ring of
Laurent polynomials $\IC[X^{\pm 1},Y^{\pm 1}]$, this is the ring of
regular functions of $(\IC^\times)^2$ considered as an algebraic
variety.
What is the analog of the points of finite period in $(\IC^\times)$?
These are just the pairs of roots of unity
$$ (\IC^\times)^2_{\mathrm{tors}}= \mu_\infty^2  \text{ where
}\mu_\infty = \{\zeta : \text{there is $n\in\{1,2,3,\ldots\}$
  with }\zeta^n \}.$$
Finally, we need to translate ``genus'' to this setting. First let us
look at some examples.

\begin{example}
  \begin{itemize}
  \item [(i)] Consider the curve $C$ defined by $P = X+Y-1$. If
    $(z,w)\in(\IC^\times)^2$ has finite order with $P(z,w)=0$, then
    both $z$ and $1-z=w$ lie on the unit circle $S^1= \{z\in\IC :
    |z|=1\}\supset\mu_\infty$. 
    Considering a picture we find exactly two possibilities    
    $$
    (z,w) \in \left\{ (e^{2\pi i/6},e^{-2\pi i/6}),(e^{-2\pi
        i/6},e^{2\pi i/6})\right\}.
    $$
    So $C\cap (\IC^\times)^2_{\mathrm{tors}}$ has two elements and is
    finite.

  \item[(ii)] Consider the curve $C$ defined by
    $P=X^{2021}Y^{-2022}-1$. Then
    $P(z,w)=0$ for all $(z,w)\in \{(e^{4044\pi iq},e^{4042\pi i
      q}) : q\in\IQ\}$.
    So here we have infinitely many solutions in $(\IC^\times)^2$.
  \end{itemize}
\end{example}

The second example suggests that curves of large degree may contain
infinitely many points of $(\IC^\times)^2$. But observe that the zero
set of $X^{r}Y^{s}-1$ in $(\IC^\times)^2$ is a subgroup for all
$r,s\in\IZ$. 

\begin{exercise}
  \label{exer:infinitemany}
  Let $\gamma\in \IC^\times$ and $(r,s)\in\IZ^2\ssm
  \{0\}$. Show that $P=X^rY^s-\gamma$ is irreducible in $\IC[X^{\pm
    1},Y^{\pm 1}]$ and show that its zero set $Z(P)$ is a coset in
  $(\IC^\times)^2$. Show that $Z(P)$ contains a point in
  $(\IC^\times)^2$ if and only if $\gamma$ is a root of unity. 
\end{exercise}

\begin{definition}
  An algebraic curve in $(\IC^\times)^2$
  is called special if it is the translate of an algebraic subgroup by
  a point of finite order. 
\end{definition}

We have the following classical characterization of special curves,
presented here without proof. For a reference see !!!REF. 

\begin{lemma}
  \label{lem:specialGm2}
  If $Z(P)$ is special, then up-to a monomial factor
  $P=X^rY^s-\zeta$ where $(r,s)\in\IZ^2\ssm\{0\}$
  and $\zeta$ a root of unity. 
\end{lemma}


In the first example, the set $\{(z,w) \in S^1\times S^1 : z+w=1\}$
was finite. Write $Z(P) = \{(z,w) \in (\IC^\times)^2 : P(z,w)=0\}$. If
$Z(P)\cap (S^1)^2$ is finite, then so is $Z(P) \cap (\IC^\times)^2$.

\begin{exercise}
  \label{exer:toralpoly}
  Find a non-zero polynomial $P\in \IC[X^{\pm 1},Y^{\pm 1}]$ such that
  $Z(P)$ is \emph{not} special but with $\{(z,w)\in S^1 \times S^1 :
  P(z,w)=0\}$ uncountable infinite.  
\end{exercise}
%% Hint: Blaschke products

Lang's paper contains a proof of the follow theorem which he
attributes to Ihara, Serre, and Tate.

\begin{theorem}
  \label{thm:ist}
  Let $C\subset (\IC^\times)^2$ be an irreducible algebraic curve.
  Then
  \begin{equation*}
    C \cap (\IC^\times)^2_{\mathrm{tors}}\text{ is infinite}\quad\Longleftrightarrow\quad \text{$C$ is special}. 
  \end{equation*}
\end{theorem}

The direction ``$\Longleftarrow$'' is the content of
Exercise~\ref{exer:infinitemany} combined with
Lemma~\ref{lem:specialGm2}. We will prove ``$\Longrightarrow$'' follow
the approach of Tate (which is similar to Serre's).

\begin{exercise}
  Read Ihara's proof~\cite{Lang:Division} of Theorem~\ref{thm:ist}.
\end{exercise}

We we split the proof sketch up into several steps. We assume that
$C=Z(P)$ intersects $(\IC^\times)^2$ in an infinite set where $P\in
\IC[X^{\pm 1},Y^{\pm 1}]$ is irreducible.
Our aim is to show that $C$ is special, in other words, $P$ is up-to a
monomial factor of the form $X^rY^s-\zeta$ with $\zeta$ a root of
unity. 

The theorem is ultimately of arithmetic nature and Tate's proof uses
input from number theory. The basic observation is that
$(\IC^\times)^2_{\mathrm{tors}}\subset
(\IQbar^\times)^2_{\mathrm{tors}}$ with $\IQbar$ the algebraic closure
of $\IQ$ in $\IC$.

\subsubsection{Reduction to the case $P \in \IQbar[X^{\pm
    1},Y^{\pm 1}]$.}

We show that $P$ lies in $\IQbar[X^{\pm 1},Y^{\pm 1}]$ up-to
multiplication by a non-zero complex number. To do this we may assume
that $P$ is a polynomial of degree $d\ge 1$.

Consider a polynomial $Q\in
\IQbar[X,Y]$ of degree $e\ge 1$ whose coefficients are unknowns. So
$Q$ has a total of $N = \frac 12 e(e+1)$ coefficients. 
By hypothesis there are infinitely many distinct
$\zeta_1,\zeta_2,\ldots \in Z(P)\cap (\IC^\times)^2$.
The equations $Q(\zeta_1) = \cdots = Q(\zeta_{N-1})=0$
impose $N-1$ linear conditions on the $N$ coefficients of
$Q$.
This is an underdetermined system of homogeneous linear equations. So
there exists $Q\in \IQ[X,Y]$ with $Q\not=0$  and $\deg Q \le e$ that vanishes at
$\zeta_1,\ldots,\zeta_{N-1}$. But $P$ vanishes at these same points.
If $P$ does not divide $Q$, then B\'ezout's Theorem implies
\begin{equation*}
  \frac 12 e(e+1)-1=  N-1 \le \text{number of common zeros of $P$ and $Q$} \le de.
\end{equation*}
As the right-hand side grows quicker in $e$ than the left-hand side we obtain a
contradiction for $e$ large enough. So $P$ is a divisor of $Q$ and
therefore must have coefficients in $\IQbar$ by multiplying by a non-zero
scalar. 

So $P \in K[X^{\pm 1},Y^{\pm 1}]$ where $K$ is a number field that we may assume is
Galois over $\IQ$. To keep the notation light we will assume for
simplicity $K=\IQ$. The general case follows along similar lines with
some extra bookkeeping.

\subsubsection{Roots of unity and their Galois conjugates}

Each root of unity $\zeta$ has an order, the minimal
$n\in\IN=\{1,2,3,\ldots\}$
with
$\zeta^n=1$. The $\IQ$-minimal polynomial $\phi_n\in\IQ[X,Y]$
of $\zeta$ is a divisor of
$X^n-1$. Some important facts:

\begin{itemize}
\item The roots of $\phi_n$ are $\{\zeta^a : a\in
  (\IZ/n\IZ)^\times\}$. In particular, $\phi_n$ depends only on $n$
  and $\IQ(\zeta)/\IQ$ is a Galois extension. 
\item The degree $\deg \phi_n = \varphi(n)= n \prod_{p\mid n}
  \left(1-\frac 1p\right)$, where the product runs over prime divisors
  of $n$.
\end{itemize}

\begin{crucial}
  If $(\zeta,\xi)\in \mu_\infty^2$ and $P(\zeta,\xi)=0$ and if
  $\sigma\in \mathrm{Gal}(\IQbar/\IQ)$, then
  $P(\sigma(\zeta),\sigma(\xi))=0$.

  The action of the Galois groups on $\mu_\infty$ is by raising to a
  power. For any $a\in\IZ$ coprime to the order of $(\zeta,\xi)$  
  we find $P(\zeta^a,\xi^a)=0$. 
\end{crucial}

\begin{lemma}[Galois Homothety Lemma]
  Say $(\zeta,\xi)\in \mu_\infty^2$ has order $n\ge 2$
  with $P(\zeta,\xi)=0$. There exists a prime number $\ell=O((\log
  n)^2)$ with $P(\zeta^\ell,\xi^\ell)=0$. 
\end{lemma}
\begin{proof}
  The Prime Number Theorem states that the number of primes $\le x$ is
  asymptotically equal to $x/\log x$ for $x\ge 2$.
  We will only need that this number
  is $\ge \delta x/\log x$, for fixed $\delta >0$. 
  The number of
  distinct prime divisors of $n$ is at most $(\log n)/\log 2$.
  So the least prime number $\ell$ that does not divide $n$ satisfies $\ell =
  O((\log n)^2)$. By the crucial observation we conclude
  $P(\zeta^\ell,\xi^\ell)=0$. 
\end{proof}

Given a single point in $Z(P)\cap\mu_\infty^2$ the action of the
Galois group producs more points. How many more? The number is
$\varphi(n)$ where $n$ is the order of $(\zeta,\xi)$. 

\begin{lemma}[Large Galois Orbit Lemma]
  We have $\varphi(n) \ge (n/2)^{1/2}$. 
\end{lemma}
\begin{proof}
  Factor $n=p_1^{e_1}\cdots p_g^{e_g}$ into primes $p_1<p_2<\cdots <
  p_g$ and $e_1,\ldots,e_g\in\IN$. Then $\varphi(n) = \prod_{i=1}^g
  p_i^{e_i-1}(p_i-1)$. Then $p_i^{e_i-1}(p_i-1)\ge p_i^{e_i/2}$ except
  if $p_i^{e_i}=2$, \textit{i.e.}, $p_i=2,e_i=1$, in which case we
  need the factor $2^{-1/2}$. 
\end{proof}

Now we can put things together. Let us again multiply $P$ by a
monomial so that it becomes a polynomial.
Let $(\zeta,\xi)\in \mu_\infty^2$ have order $n\ge 2$ with
$P(\zeta,\xi)=0$. Let $\ell$ be a prime as in the Galois Homothety Lemma.
Then $(\zeta,\xi)$ is a common zero of $P(X,Y)$ and
$P(X^{\ell},Y^{\ell})$. By the Crucial Observation so is any Galois
conjugate. By the Large Galois Orbit Lemma there are at least
$(n/2)^{1/2}$ common roots. If $P$ does not divide $P(X^\ell,Y^\ell)$,
then B\'ezout's Theorem implies
\begin{equation*}
  n^{1/2} = O((\log n)^2)
\end{equation*}
and this means that $n$ is bounded. But by assumption we may take $n$
as large as we want and therefore $P$ must divide $P(X^\ell,Y^\ell)$.

\begin{exercise}
  Show that $X+Y-1$ does not divide  $X^\ell + Y^\ell-1$ for any prime
  $\ell$. 
\end{exercise}

\subsubsection{Functional Transcendence}

We return to a more geometric point of view in this final set.
If $C=Z(P)$ contains a torsion point of large enough order $n$, then
$P \mid P(X^\ell,Y^\ell)$ for some prime $\ell$.
Geometrically, $C \subset [\ell]^{-1}(C)$ where $[\ell]$ denotes the $\ell$-th
power endomorphism of $(\IC^\times)^2$.
If $\bfzeta\in (\IC^\times)^2$ with $\bfzeta^\ell=1$, then the
translation $\bfzeta C\subset [\ell]^{-1}(C)$. These $\bfzeta C$ are
irreducible components of $[\ell]^{-1}(C)$; their total number is
$\ell^2$. But the degree of $[\ell]^{-1}(C)$ is $\ell \deg C$. As soon
as $\ell > \deg C$ the Pigeonhole Principle provides distinct
$\bfzeta',\bfzeta''\in \mu_\infty$ with $\bfzeta' C = \bfzeta''C$.

Hence $\bfzeta C=C$ with $\bfzeta = \bfzeta'{\bfzeta''}^{-1}$ of order
$\ell$.

Let us now check that $C$ is the translate of an algebraic subgroup of
$(\IC^\times)^2$ by a point of finite order. % After translating we may
% assume $1=(1,1)\in C$ and then it suffices to show that $C$ is a
% subgroup of $(\IC^\times)^2$.

To this end consider
\begin{equation*}
  G = \bigcap_{P\in C} P^{-1} C.
\end{equation*}

As a set, $G$ is a subgroup of $(\IC^\times)^2$. But it is also Zariski
closed and of dimension at most $1$. So it is either finite or a
curve. Observe that $\bfzeta C = C$ implies $\bfzeta\in G$. As
$\bfzeta$ has order $\ell$. Although $\ell = O((\log n)^2)$ the proof
can easily be modified to produce $\ell$ that is larger than any
prescribed constant that depends on $C$. Therefore, $G$ must be
infinite and thus equal to a translate of $C$.

Therefore, $C$ is special, and this completes the proof of
Theorem~\ref{thm:ist}.



% So $P(\zeta_1 X,\zeta_2 Y)$ and $P(X,Y)$ define the same curve.
% These two are equal up-to to a scalar if we assume, as we may that $P$ is a polynomial.

% If $a_{ij}$ denotes a non-zero coefficient of $P$, then comparing
% coefficients gives
% $a_{ij}\zeta_1^i \zeta_2^j = \lambda a_{ij}$ for some $\lambda\not=0$
% independent of $i,j$. As there are at least two monomials we find
% $\zeta_1^{i}\zeta_2^j = \zeta_1^{i'}\zeta_2^{j'}$ with $(i,j)\not=(i',j')$.

% Then $\prod_{\sigma \in \mathrm{Gal}(K/\IQ)}
% \sigma(P)$ has coefficients in $\IQ$ and its zero set has infinitely
% many points in common with $(\IC^\times)^2_{\mathrm{tors}}$.

\section{The Modular Side}

Roots of unity are precisely images of rational numbers under the
holomorphic map $x\mapsto e^{2\pi i x}$. We often take for granted
that this is a very particular thing: this holomorphic map takes
rationals to algebraic numbers. Moreover, the deep Kronecker--Weber
Theorem from Class Field Theory states that any finite Galois
extension of $\IQ$ with abelian Galois group is contained in the field
generated by a root of unity. So the map $x\mapsto e^{2\pi i x}$ is
intricately connected to the abelian extensions of $\IQ$.

To understand abelian extensions of an imaginary quadratic number
$F=\IQ(\sqrt{-D})$ we need elliptic curves with complex
multiplication.

Let us regard an elliptic curve $E$ defined over $\IC$ from the
complex analytic point of view. Thus $E(\IC)$ carries the structure of
a complex Lie group and there is a holomorphic uniformation map
\begin{equation*}
  u \colon \IC \rightarrow E(\IC)
\end{equation*}
whose kernel equals rank $2$ discrete subgroup of $\IC$ and whose
image is all of $E(\IC)$. After a normalization we may assume that
\begin{equation*}
  \Omega = \mathrm{ker}(u) = \IZ+\tau\IZ
\end{equation*}
where $\tau\in \IH = \{z\in \IC : \mathrm{Im}(z)>0\}$. We call
$\Omega$ a period lattice for $E$. The complex number $\tau$ is
uniquely determined up-to the action of $\mathrm{SL}_2(\IZ)$ on $\IH$
by fractional linear transformations. 

Any endomorphism of $E$ is a morphism $f\colon E\rightarrow E$ of
algebraic varieties with $f(0)=0$. Let $\mathrm{End}(E)$ denote all
endomorphisms of $E$. Any $f\in \mathrm{End}(E)$ is
induced by a unique $\IC$-linear map
$\IC\rightarrow \IC$ that maps $\Omega$ to itself. In other words, we
identify
\begin{equation*}
  \mathrm{End}(E) = \{ \alpha\in\IC : \alpha \Omega \subset \Omega \}
  = \{\alpha\in \IC : \alpha\in \Omega, \alpha\tau\in \Omega\}. 
\end{equation*}
Clearly,  $\IZ$ is contained in the right-hand side. For example  $N\in\IN$
correspond to the multiplication-by-$N$ endomorphism $[N](P) =
\underbrace{P+\cdots +P}_{N\text{ times}}$ of $E$.

\begin{lemma}
  Suppose $\mathrm{End}(E) \supsetneq \IZ$, then
  $\IQ(\tau)=\IQ(\alpha)$
  is imaginary
  quadratic and $\alpha$ is an algebraic integer. 
\end{lemma}
\begin{proof}
  Let $\alpha\in \mathrm{End}(E)$, then there exists $A\in\mathrm{Mat}_2(\IZ)$ with
  \begin{equation*}
    \alpha \left(
      \begin{array}{c}
        1 \\ \tau 
      \end{array}
    \right) = % \left(
      % \begin{array}{cc}
      %   a & b \\ c & d
      % \end{array}\right)
    A\left(
      \begin{array}{c}
        1 \\ \tau 
      \end{array}
    \right).
  \end{equation*}
  Therefore, $\alpha$ is an eigenvalue of $A$; in particular
  $[\IQ(\alpha):\IQ]\le 2$. Moreover, $\matto{1}{\tau}$ lies in the
  kernel of $\alpha I-A$. If $\alpha\not\in \IZ$, then $\alpha
  I-A\not=0$ and thus $\tau \in \IQ(\alpha)$. The lemma follows as
  $\alpha\not\in\IR$. 
\end{proof}

\begin{exercise}
  Show the following statement. If $\tau \in\IH$  with
  $[\IQ(\tau):\IQ]$, then $\mathrm{End}(\IC/(\IZ+\tau \IZ))\supsetneq
  \IZ$. 
\end{exercise}

\begin{definition}
  We say that $E$ has complex multiplication if
  $\mathrm{End}(E)\supsetneq \IZ$. 
\end{definition}

If $E$ is presented algebraically via an Weierstrass equation $y^2 =
x^3+ax+b$ with $a,b\in\IC$. Then the $j$-invariant $j(E)$ of $E$ equals
\begin{equation*}
  j(E) = 2^8 3^3 \frac{a^3}{4a^3+27b^2}. 
\end{equation*}

\begin{example}
  The elliptic curve $E$ presented by $y^2=x^3+x$ has the additional
  endormorphism
  $(x,y)\mapsto (-x, i y)$. Its $j$-invariant equals $1728$. The
  complex picture goes as follows, we have $E(\IC)\cong \IC/(\IZ+i
  \IZ)$. 
\end{example}

We come to an important fact: Two elliptic curves over $\IC$ are
isomorphic if and only if their $j$-invariants are equal.

As each complex number is the $j$-invariant of an elliptic curve we
find: The $j$-invariant allows us to identify  isomorphism classes of
elliptic curves over $\IC$ with the set of complex points of the
affine line.
To emphasize this modular property of the affine line we will
sometimes denote it by $Y(1)=\IA^1$, which is standard notation for a
modular curve in a certain infinite family. 


So $j(E) = j(\IC/(\IZ+\tau\IZ))$ is a
function in $\tau$ that is invariant under the action of
$\mathrm{SL}_2(\IZ)$. So we obtain a function $j\colon \IH\rightarrow
\IC$ determined by $j(\tau) = j(\IC/(\IZ+\tau\IZ))$ for all
$\tau\in\IH$; it is sometimes called Klein's $j$-functions.
Observe that $j(\tau)=j(\tau+1)$ and $j(\tau) =
j(-1/\tau)$. The first equation shows that $j$ depends only on
$q=e^{2\pi i \tau}$. It is sometimes useful to consider $j$ as a
function in $q$. 

It turns out that much more is true: Klein's $j$-function is
holomorphic on $\IH$ and meromorphic at infinity. It has a famous 
a Taylor series in $q$, the first few terms\footnote{There is some
  dispute about where the constant terms $744$ is the ``right'' one.} of which are 
\begin{equation*}
  j(q) = \frac 1q + 744 + 196884q +\cdots. 
\end{equation*}

\begin{definition}
  \label{def:specialY1}
  A special point of $Y(1)$ is the $j$-invariant of an elliptic curve
  with complex multiplication. Equivalently: the set of special points
  of $Y(1)$ is $\{j(\tau) : \tau\in\IH \text{ and }[\IQ(\tau):\IQ]=2\}$. 
\end{definition}

The set of special points is at most countable infinite and it is
indeed infinite since $j$ is non-constant.

The following theorem extends the analogy between roots of unity and
special points of $Y(1)$. It is a consequence of class field theory. 
\begin{theorem}
  Let $z\in \IC$ be a special point of $Y(1)$ with $z=j(\tau)$ where
  $\tau\in \IH$. Let $F=\IQ(\tau)$.
  \begin{enumerate}
  \item [(i)] Then $z$ is an algebraic integer.
  \item[(ii)] The field $F$ is
    imaginary quadratic and   $\mathrm{End}(E)$ is an order
    in $F$.
  \item[(iii)] The extension $F(z)/F$ is Galois with abelian Galois
    group.
  \item[(iv)] If $\mathrm{End}(E)$ is the maximal order in $F$,
    \textit{i.e.}, the ring of integers of $F$, then
    the class group $H_F$ of $F$ is isomorphic to
    $\mathrm{Gal}(F(z)/F)$ via the Artin homorphism. 
  \end{enumerate}
\end{theorem}

The Conjecture of Andr\'e--Oort is the analog (and generalization) of
the Theorem of Ihara--Serre--Tate in this modular side.

For this let us just consider $Y(1)^2$ instead of $(\IC^\times)^2$.
\begin{definition}
  The set of special points of  $Y(1)^n$ is $Y(1)^n_{\mathrm{special}}=\{(z_1,\ldots,z_n) :
  z_i\in Y(1)(\IC)\text{ is special} \}$. 
\end{definition}

Which algebraic curves in $Y(1)^2$ contain infinitely many special
points? There are some immediate classes.

\begin{example}
  \begin{enumerate}
  \item[(i)] The diagonal  $\{(z,z) : z\in\IC\}$ contains infinitely
    many special points of $Y(1)^2$. 
  \item [(ii)]  Let $j_1$ be a special point of $Y(1)$.  Then the curves
    $\{j_1\}\times Y(1)$ and $Y(1)\times \{j_1\}$ contain infinitely many
    special points of $Y(1)^2$. 
  \end{enumerate}
\end{example}

These are first candidates of special curves in $Y(1)^2$. 
But there is collection of non-trivial.

Recall that an isogeny of two elliptic curves $E_1$ and $E_2$, both
defined over $\IC$, is a non-constant morphism $f\colon E_1\rightarrow
E_2$ of algebraic varieties with $f(0)=0$. It turns out that $f$ must
then be finite and has a finite kernel $\mathrm{ker}(f)$.
We call $f$ a \emph{cyclic isogeny}, if $\mathrm{ker}(f)$ is a cycle
group. The degree $\deg f$ of $f$ is the cardinality of $\mathrm{ker}(f)$. 

In $E_i$ is represented by $\IC/(\IZ+\tau_i\IZ)$, for $i\in \{1,2\}$,
then $f$ is represented by a $\IC$-linear map $\IC\rightarrow\IC$ that
maps $\IZ+\tau_1\IZ$ to $\IZ+\tau_2\IZ$.

Let $N\in\IN$. 
It turns out that the set
\begin{equation*}
  \left\{ (j(\tau_1),j(\tau_2)) : \text{there is a cyclic isogeny 
    $\IC/(\IZ+\tau_1\IZ)\rightarrow \IC/(\IZ+\tau_2\IZ)$ of degree $N$}\right\}
\end{equation*}
is an irreducible algebraic curve in $Y(1)^2$, \textit{i.e.}, it is
the zero set of an irreducible polynomial $\Phi_N\in\IC[X,Y]$.
Moreover, we have $\Phi_N\in\IZ[X,Y]$. 

\begin{example}
  For $N=1$ we recover the diagonal $\{(z,z) : z\in\IC\}$, so $\Phi_1
  = X-Y$. 
  Already for $N=2$ the polynomial is quite complicated. Indeed,
  \begin{alignat*}1
    \Phi_2 &= 
    X^3 - X^2Y^2 + 1488X^2Y - 162000X^2 + 1488XY^2+ 40773375XY +
    8748000000X + \\
    &Y^3 - 162000Y^2 + 8748000000Y -157464000000000.    
  \end{alignat*}
  See Cohen's work~\cite{Cohen} where it is proved that the
  coefficients of $\Phi_N$ grow superexponentially in $N$. 
\end{example}

\begin{lemma}
  Let $N\in\IN$, then $\Phi_N(j(\tau),j(N\tau))=0$ for all
  $\tau\in\IH$. 
\end{lemma}
\begin{proof}
  The multiplication-by-$N$ map $\IC\rightarrow\IC$ factors through
  $\IC/(\IZ+\tau\IZ) \rightarrow \IC/(\IZ+N\tau\IZ)$.
  The kernel of the latter homomorphism is $\{\frac kN \IZ
  +(\IZ+\tau\IZ) : k\in \{0,\ldots,N-1\}\}$ which is cyclic of order
  $N$.  
\end{proof}

Suppose $j(\tau)$ is a special point of $Y(1)$, then so is $j(N\tau)$,
see Definition~\ref{def:specialY1}.

We conclude that for all $N\ge 1$, the algebraic  curve $Z(\Phi_N)
\subset Y(1)^n$ contains infinitely many special points.

\begin{definition}
  An algebraic curve  $C\subset Y(1)^2$ is called a special curve of
  $Y(1)^2$ 
  \begin{enumerate}
  \item [(i)] if $C = \{z\}\times Y(1)$ for a special point $z$, or
\item [(ii)] if $C =  Y(1)\times \{z\}$ for a special point $z$, or
\item[(iii)] if $C=Z(\Phi_N)$ for some $N\in\IN$. 
\end{enumerate}
\end{definition}

The content of the Andr\'e--Oort Conjecture is the converse of the
observations above.

\begin{theorem}[Andr\'e--Oort for $Y(1)^2$, Andr\'e~\cite{Andre:AO},
  Edixhoven~\cite{Edixhoven} under GRH, Pila~\cite{Pila:AO} for
  subvarieties of $Y(1)^n$, effective versions by
  Bilu--Masser--Zannier~\cite{} and K\"uhne~\cite{}]
  Let $C\subset Y(1)^2$ be an irreducible curve. Then
  \begin{equation*}
    C \cap Y(1)^2_{\mathrm{special}}\text{ is
      infinite}\quad\Longleftrightarrow\quad \text{$C$ is a special
      curve of $Y(1)^2$}. 
  \end{equation*}  
\end{theorem}

\section{Further Reading and Open Problems}


%%% Local Variables:
%%% TeX-master: "main"
%%% End:
