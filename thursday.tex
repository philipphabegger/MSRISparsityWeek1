% MSRI - SGS Sparsity week 1, Wednesday, 1 lecture, 60 minutes

\documentclass{beamer}

\usepackage{amsmath,amssymb,amsthm,mathrsfs,amscd,mathtools}
\usepackage{datetime}
\usepackage{csquotes}
\usepackage{hyperref}
\usepackage{graphicx}
\usepackage{tikz,tikz-cd}
\usetikzlibrary{arrows,shapes}

\usetheme{Metropolis}
\metroset{block=fill}




\newcounter{maincounter}
\newcounter{excounter}

\newtheorem{conjecture}{Conjecture}
% \setbeamercolor{block body}{bg=mDarkTeal!15}
% \setbeamercolor{block title}{bg=mDarkTeal,fg=black!2}


\newcounter{maincounter}
\newcounter{excounter}
\numberwithin{maincounter}{chapter}
\numberwithin{equation}{chapter}
\numberwithin{excounter}{chapter}
\renewcommand{\theexcounter}{\thechapter.\Alph{excounter}}
\newtheorem{lemma}[maincounter]{Lemma}
\newtheorem{proposition}[maincounter]{Proposition}
\newtheorem{corollary}[maincounter]{Corollary}
\newtheorem{remark}[maincounter]{Remark}
\newtheorem{theorem}[maincounter]{Theorem}
\newtheorem{exercise}[excounter]{Exercise}
\newtheorem{example}[maincounter]{Example}

\newtheorem*{crucial}{Crucial Observation}

\newtheorem{conjecture}[maincounter]{Conjecture}
\newtheorem{definition}[maincounter]{Definition}

\def\AA{\mathbb{A}}
\def\BB{\mathbb{B}}
\def\EE{\mathbb{E}}
\def\HH{\mathbb{H}}
\def\DD{\mathbb{D}}
\def\NN{\mathbb{N}}
\def\RR{\mathbb{R}}
\def\TT{\mathbb{T}}
\def\CC{\mathbb{C}}
\def\ZZ{\mathbb{Z}}
\def\PP{\mathbb{P}}
\def\QQ{\mathbb{Q}}
\def\FF{\mathbb{F}}
\def\GG{\mathbb{G}}
\def\LL{\mathbb{L}}
\def\MM{\mathbb{M}}
\def\SS{\mathbb{S}}
\def\UU{\mathbb{U}}
\def\XX{\mathbb{X}}


%%% Philipp's macros

\newcommand{\dom}[1]{{\mathrm {dom}}({#1})}
\newcommand{\sman}[1]{{#1}^{\mathrm{sm,an}}}
\newcommand{\ansm}[1]{{#1}^{\mathrm{an,sm}}}
\newcommand{\sm}[1]{{#1}^{\mathrm{sm}}}
\newcommand{\anE}{\mathrm{an}}
\newcommand{\an}[1]{{#1}^{\anE}}
\newcommand{\stab}[1]{{\mathrm{Stab}(#1)}}


\newcommand{\hgtexp}{S}

\newcommand{\rank}{{\rm rank}\,}
\newcommand{\Hpoly}[2]{{H^{}_{#1}({#2})}}
\newcommand{\poly}[2]{{#1^{}({#2})}}
\newcommand{\polyt}[2]{{#1^{\sim}({#2})}}
\newcommand{\polytiso}[2]{{#1^{\sim,{\rm iso}}({#2})}}
%\renewcommand{\graph}[1]{\Gamma({#1})}
\newcommand{\atopx}[2]{{\genfrac{}{}{0pt}{}{#1}{#2}}}
\newcommand{\IP}{{\PP}}
\newcommand{\IG}{{\GG}}
\newcommand{\IH}{{\HH}}
\newcommand{\IC}{{\CC}}
\newcommand{\IR}{{\RR}}
\newcommand{\IT}{{\TT}}
\newcommand{\IRan}{{{\RR}_{\rm an}}}
\newcommand{\IRanexp}{{{\RR}_{\rm an,exp}}}
\newcommand{\RRan}{{\IRan}}
\newcommand{\RRanexp}{{\IRanexp}}
\newcommand{\IRalg}{{\RR}_{\rm alg}}
\newcommand{\IQbar}{{\overline{\QQ}}}
\newcommand{\Kbar}{{\overline{K}}}
\newcommand{\IZ}{{\ZZ}}
\newcommand{\IN}{{\NN}}
\newcommand{\IA}{{\AA}}
\newcommand{\IQ}{{\QQ}}
\newcommand{\IQpbar}{{\overline{\QQ}_p}}
\newcommand{\IQp}{{\QQ_p}}
\newcommand{\ts}[1]{{T}_0({#1})}

\newcommand{\cC}{{\mathcal C}}
\newcommand{\cE}{{\mathcal E}}
\newcommand{\cF}{{\mathcal F}}
\newcommand{\cK}{{\mathcal K}}
\newcommand{\cL}{{\mathcal L}}
\newcommand{\cM}{{\mathcal M}}
\newcommand{\cO}{{\mathcal O}}
\newcommand{\cV}{{\mathcal V}}
\newcommand{\cW}{{\mathcal W}}
\newcommand{\cX}{{\mathcal{X}}}
\newcommand{\cY}{{\mathcal Y}}
\newcommand{\cZ}{{\mathcal Z}}



\newcommand{\defZ}{Z}
\newcommand{\defF}{F}
\newcommand{\defW}{W}
\newcommand{\defC}{C}
\newcommand{\defE}{E}
%\newcommand{\deffam}{F}


\newcommand{\re}[1]{{\rm Re}({#1})}
\newcommand{\imS}{{\rm Im}}
\newcommand{\im}[1]{\imS({#1})}
\newcommand{\imageS}{{\rm im}}
\newcommand{\image}[1]{\imageS({#1})}
\newcommand{\volS}{{\rm vol}}
\newcommand{\vol}[1]{\volS({#1})}
\newcommand{\orth}[1]{{#1}^{\bot}}
\newcommand{\mat}[2]{{\rm Mat}_{#1}({#2})}
\newcommand{\ssm}{\setminus}
\newcommand{\ord}[1]{{\rm ord}({#1})}
\newcommand{\opt}[2]{{\rm Opt}_{#2}({#1})}
\newcommand{\Height}[1]{{H}({#1})}
\newcommand{\trdeg}{{\rm trdeg\,}} 
\newcommand{\geo}[1]{\langle {#1}\rangle_{{\rm geo}}}
\newcommand{\defect}{\delta}
\newcommand{\geodef}{{\delta_{\rm geo}}}
\newcommand{\en}[1]{{\rm End}({#1})}
\newcommand{\Hom}[1]{{\rm Hom}({#1})}
\newcommand{\hommaxR}[1]{\text{\rm Hom}({#1})^{*}_{\IR}}
\newcommand{\arith}{\rm arith}
\newcommand{\sgu}[2]{{#1}^{[{#2}]}}
\newcommand{\oa}[1]{{#1}^{\rm oa}}
\newcommand{\codim}{{\rm codim}}
\newcommand{\lgo}{LGO}
\newcommand{\zcl}[1]{{\rm Zcl}({#1})}


\newcommand{\trans}[1]{{#1}^{T}}

\newcommand{\red}[1]{\textcolor{red}{#1}}

\renewcommand{\subset}{\subseteq} %%% Some people think \subset
%%% excludes equality
\renewcommand{\supset}{\supseteq}

\newcommand{\gra}[1]{\mathrm{Gr}({#1})}


\newcommand{\gl}[2]{{\mathrm {GL}}_{#1}({#2})}
\renewcommand{\sp}[2]{{\mathrm {Sp}}_{#1}({#2})}
\newcommand{\autS}{{\mathrm {Aut}}}
\newcommand{\aut}[1]{\autS({#1})}

\newcommand{\spec}[1]{\mathrm{Spec}\,{#1}}

\newcommand{\tor}[1]{{#1}_{\mathrm{tor}}}
\newcommand{\gal}[1]{{\mathrm{Gal}}({#1})}


\newcommand{\zeroset}[1]{\mathscr{Z}({#1})}


\newcommand{\jac}{\mathrm{Jac}}

\newcommand{\bfzeta}{{\boldsymbol{\zeta}}}

\newcommand{\mattt}[4]
{\left(
  \begin{array}{cc}
    {#1} & {#2} \\ {#3} & {#4} 
  \end{array}
\right)}

\newcommand{\matto}[2]
{\left(
  \begin{array}{c}
    {#1} \\ {#2}
  \end{array}
\right)}

\newcommand{\matot}[2]
{\left(
  \begin{array}{cc}
    {#1} & {#2}
  \end{array}
\right)}


\title{MSRI Summer Graduate School \\ Sparsity of Algebraic Points \\
  Day 4: Manin--Mumford and Andr\'e--Oort}
\author{Philipp~Habegger \\ University of Basel \\ \texttt{philipp.habegger@unibas.ch}}
\date{Thursday, June 10, 2021}

\begin{document}

\setlength{\abovecaptionskip}{0pt} 
\setlength{\belowcaptionskip}{0pt} 

\renewcommand{\figurename}{Fig.}


\begin{frame}
  \titlepage
\end{frame}

\begin{frame}{Manin--Mumford Conjecture}
  Let $A$ be an abelian variety defined over $\IC$.
  
  \begin{definition}
    The subgroup of points of finite order in $A(\IC)$ is
    $A_{\mathrm{tors}}$.
    
    A \alert{torsion coset} $K$ of $A$
    is
    $$ T+B $$
    with $B$ an abelian subvariety of $A$ and $T\in\mathrm{A}_{\mathrm{tors}}$. 
  \end{definition}

  \begin{theorem}

  Suppose $K\subset\IC$ is a number field and $A$ is an abelian
  variety defined over $K$. 
  Let $V\subset A$ be an irreducible closed subvariety, also defined over a
  number field $K$. Then
  \begin{equation*}
    V(\IC) \cap A_{\mathrm{tors}} \text{ lies Zariski dense in
      $V$}\Leftrightarrow 
    \text{$V$ is a torsion coset of $A$.}
  \end{equation*}
\end{theorem}
\end{frame}

\section{Arithmetic of Torsion Points}

\begin{frame}{Group-theoretic Properties} 
  We collect some facts about the torsion points of an
  abelian variety $A\subset\IP^n$ defined over a number field
  $K\subset \IC$.

  As groups we have $A(\IC) \cong \IR^{2g}/\IZ^{2g}$ with $g=\dim A$.

  For $N\in\IZ$ let $A[N] = \ker \text{(multiplication-by-$N$)}
  \subset A(\overline K)$.
  Then
  \begin{equation*}
    A_{\mathrm{tors}} = \bigcup_{N\in\IN} A[N]
  \end{equation*}
  \begin{lemma}
    $A[N] \cong (\IZ/N\IZ)^{2g}$ if $N\ge 1$.  
  \end{lemma}
\end{frame}


\begin{frame}{The Galois Action}
  Let
  $P=[x_0:\cdots:x_n]\in \IP^n(\overline{K})$ with  $(x_0,\ldots,x_n) \in
  \overline{K}^{n+1}\ssm\{0\}$.
  Say $\sigma\in \absgalk$, then
  \begin{equation*}
    \sigma(P)=  \sigma([x_0:\cdots:x_n]) = [\sigma(x_0):\cdots:\sigma(x_m)]
  \end{equation*}
  is a well-defined action of $\absgalk$ on $\IP^n(\overline K)$.

  If $P\in \IP^n(K)$ we may assume $\forall i:x_i\in K$, so
  $\sigma(P)=P$ for all $\sigma$.
  Conversely, 
  $P\in \IP^n(K)$ with $\sigma(P)=P$ for all $\sigma$ implies
  $P\in \IP^n(K)$.

  \begin{definition}
    For $P\in \IP^n(\overline K)$. 
    We set $K(P)$ to be the fixed field of the stabilizer
    $\{\sigma\in \absgalk : \sigma(P)=P\}$. 
  \end{definition}

  We have
  \begin{equation*}
    [K(P):K]  = \#\{\sigma(P) : \sigma\in\absgalk \}. 
  \end{equation*}

\end{frame}

\begin{frame}
  As $A\subset\IP^n$ is defined over $K$, it is the zero set 
  of  homogeneous polynomials $f_1,\ldots,f_m$ with
  coefficients in $K$. Let $P\in A(\overline K)$. 

  $\Rightarrow \forall i: f_i(P) = 0 
  \Rightarrow \forall \sigma\in\absgalk \forall i: f_i(\sigma(P))=\sigma(f_i(P))=0$

  So $\sigma(P) \in A(\overline K)$ for all $\sigma\in \absgalk$. 

  For $N\in\IN$ the morphism
  $[N]\colon A\rightarrow A$ is, (Zariski-locally), presented
  by a collection of homogeneous polynomials. This implies
  $$\sigma([N](P)) = 
  [N](\sigma(P))$$
  holds for {all} $P\in A(\overline K)$.

  We conclude 
  $\sigma(A[N])\subset A[N]$ and
  $\sigma(A_{\mathrm{tors}}) \subset A_{\mathrm{tors}}$ for all $N$
  and all $\sigma$. 

  \begin{definition}
    The Galois action on $A[N]$ induces a representation
    $\rho_N \colon \absgalk\rightarrow\mathrm{Aut}(A[N]) \cong
    \mathrm{GL}_{2g}(\IZ/N\IZ)$.
  \end{definition}
\end{frame}
\begin{frame}
  
  \begin{lemma}
    \begin{enumerate}
    \item[(i)]
      For all $N\in\IN$, the extension  $K(A[N])/K$ is Galois
      of degree at most $N^{4g^2}$.
    \item[(ii)] If $P\in A[N]$, then $[K(P):K]\le N^2$.
    \end{enumerate}
  \end{lemma}
  \begin{proof}
    \vspace{3cm}
  \end{proof}

  We need a \alert{Large Galois Orbit Lemma} as in the proof of the
  Ihara--Serre--Tate Theorem. 
\end{frame}

\begin{frame}{Intermezzo: Serre's Theorem}
  \begin{theorem}[Serre]
    Let $E$ be  an elliptic curve defined over a number $K$
    \alert{without} complex multiplication.
    There exists $c(E)>0$ such that
    if $p\ge c(E)$ is prime number, then $\rho_p \colon
    \absgalk\rightarrow \mathrm{Aut}(E[p]) \cong \mathrm{GL}_2(\IF_p)$
    is surjective.  
  \end{theorem}

  In contrast to the multiplicative setting:
  \begin{itemize}
  \item 
    $K(A[N])/K$ will usually \alert{not} have  abelian Galois group
  \item 
     $K(A(P))/K$ is usually  \alert{not}  even Galois if $P$ has finite
     order. If  $E$
     and $p$ are as in Serre's Theorem and if $P$ has order $p$, then
     \begin{equation*}
       \mathrm{Gal}(\overline K/K(P))=       \rho_p^{-1} \left\{\mattt{1}{*}{0}{*}\right\}
     \end{equation*}
     We have $[K(P):K] = p(p-1) \le p^2$. 
   \end{itemize}   
\end{frame}

\begin{frame}{Masser's Theorem}
  For a root of unity $\zeta$ of order $n$ we have $[\IQ(\zeta):\IQ] =
  \#(\IZ/n\IZ)^\times =\varphi(n)$.
  
  In general there is no ``simple formula'' for the degree of the
  extension $K(A(P))/K$ if $P\in \mathrm{A}_{\mathrm{tors}}$ has order
  $N$.
  Serre's Theorem
  suggests that the degree $[K(P):K]$ of a point of order $N$ grows
  polynomially in terms of $N$ at least in some cases.

  
  \begin{theorem}[Masser]
    Let $A$ be an abelian variety defined over a number field $K$. There
    exist $c=c(A)>0$ and $\delta=\delta(A)>0$ with the following
    property. If $P \in A_{\mathrm{tor}}$ has order $N$, then
    $[K(P):K]\ge c N^{\delta}$. 
  \end{theorem}
\end{frame}

\begin{frame}{Proof of the Manin--Mumford Conjecture}
  Let $A$ be an abelian variety defined over a number field
  $K\subset\IC$.
  Let $V$ be an irreducible subvariety of $A$.

  
  \begin{lemma}
    Suppose $V\subset A$ is a torsion coset. Then $V(\overline K)\cap
    A_{\mathrm{tors}}$ lies Zariski dense in $V$. 
  \end{lemma}
  \begin{proof}
    The torsion points of an abelian variety lie dense with respect to
    the Archmiedean topology and so in particular with respect to the
    Zariski topology. The same holds for torsion cosets. 
  \end{proof}

  This is the easy direction of Manin--Mumford. 
\end{frame}

\begin{frame}{We have seen this slide before.}
  \begin{itemize}
  \item We can immerse $A$ into some projective space $\IP^n$.

  \item The complex Lie group $A(\IC)$ is a complex torus, \textit{i.e.} a
    quotient $\IC^g/\Omega$ where $\Omega$ is a discrete subgroup of
    rank $2g$.

  \item The quotient map $u\colon \IC^g\rightarrow \IC^g/\Omega
    =A(\IC)$ can be presented by quasi-periodic holomorphic functions
    $\vartheta_0,\ldots,\vartheta_n \colon \IC^g\rightarrow\IC$.

  \item Fix a $\IZ$-basis $(\omega_1,\ldots,\omega_{2g})$ of $\Omega$.
    This is also an $\IR$-basis of $\IC^g$ and allows us to work with
    \alert{real coordinates}, also called \alert{Betti coordinates} $\IR^{2g}$.
    Write $\tilde u\colon \IR^{2g}\rightarrow A(\IC)$, so
    $\tilde u^{-1}(A_{\mathrm{tors}})=\IQ^{2g}$. 

  \item If $V\subset A$ is an algebraic subset, then the preimage
    $\tilde u^{-1}(V(\IC)) \subset\IC^g$ is $\IZ^{2g}$-periodic.
    The restriction $u|_{[0,1)^{2g}}^{-1}(V(\IC))$ is definable in $\IRan$.
  \end{itemize}
\end{frame}

\begin{frame}
  \begin{lemma}
    Let $P\in A(\IC)$ be a torsion point of order $N$.
    There exist $x\in \IQ^{2g}$ with $H(x)=N$ and $u(x)=P$. 
  \end{lemma}
  \begin{proof}
    \vspace{2cm}
  \end{proof}
\end{frame}

\begin{frame}
  The \alert{Ueno locus} $V^{\circ}$  is $\bigcup_{\cK\subset V} \cK$ where $\cK$ is
  a coset of positive dimension. 


  \begin{lemma}
    There exists $N_0(V,A)$ with the following property. If
    $P\in V(\overline K)\cap A_{\mathrm{tors}}$ has order $N$
    and $N\ge N_0(V,A)$, then $P\in V^{\circ}$.
  \end{lemma}
  \begin{proof}\renewcommand{\qedsymbol}{}
    $X = \tilde u^{-1}|_{[0,1)^{2g}}(V(\IC)) \subset \IR^{2g}$
    is definable in $\IRan$.

    Recall $\tilde u(X^{\mathrm{alg}}) = V^{\circ}$.

    \vspace{3cm}

    % Let $\sigma\in \absgalk$. As $V\subset A\subset\IP^n$ is
    % vanishing locus of homogeneous polynomials over $K$, we have $\sigma(P)\in
    % V(\overline K) \Rightarrow P\in V(\overline K)$.

    % Let $P\in V(\overline K) \cap A_{\mathrm{tors}}$ be a point of order
    % $N$, with $N$ large.

    % Say $\sigma\in \absgalk$ then $\sigma(P)\in V(\overline K)$
    % also has order $N$.  The number of
    % Galois conjugates is $[K(P):K]\ge c(A) N^{\delta(A)}$ by
    % Masser's
    % Theorem.
  \end{proof}
\end{frame}

\begin{frame}
  \begin{proof}[Proof continued]
    \vspace{6cm}
% For any $\sigma$, Lemma~\ref{lem:rationalpreimage} provides $x_\sigma
%     \in X\cap\IQ^{2g}$ of height $N$ with $u(x_\sigma) = \sigma(P)$. The
%     number of rational points we obtain in this way is at least
%     \begin{equation}
%       \label{eq:rtlptub}
%       c(N) N^{\delta(A)}. 
%     \end{equation}

%     Now we are ready to apply the Pila--Wilkie Theorem,
%     Theorem~\ref{thm:pilawilkie}. It states
%     \begin{equation}
%       \label{eq:rtlptlb}
%       N(X\ssm X^{\mathrm{alg}},N) \le
%       c(X,\epsilon)N^\epsilon 
%     \end{equation}
%     for all $N\ge 1$, here $c(X,\epsilon)$ depends on $X$ and on $\epsilon
%     > 0$.

%     We set $\epsilon = \delta/2$, which is an invariant that depends only
%     on $A$. We compare (\ref{eq:rtlptub}) and (\ref{eq:rtlptlb}). As we
%     may assume that $N$ is large we must have $x_\sigma \in
%     X^{\mathrm{alg}}$ for some $\sigma\in \absgalk$. 
%     So $\sigma(P) = u(x_\sigma)  \in u(X^{\mathrm{alg}})$.

%     Now we apply Theorem~\ref{thm:imagealg}, a consequence of the
%     Ax--Lindemann--Weierstrass. It follows that $\sigma(P)$ lies in a
%     coset $\sigma(P)+B$ here $B$ is an abelian subvariety of $A$ with
%     $\dim B\ge 1$. As $P$ and also $\sigma(P)$ are torsion points we see
%     that $\sigma(P)+B$ is a torsion coset.
  \end{proof}

\end{frame}

\begin{frame}{Proof of Manin--Mumford}
  Let $V\subset A$ be an irreducible subvariety defined over $K$
  such that $V(\overline K)\cap A_{\mathrm{tors}}$ lies Zariski dense
  in $V$. 
  
  All but finitely many torsion
  points in $A_{\mathrm{tors}}$ have order at least $N(A,V)$.  
  By the last Lemma,
  the Ueno locus $V^{\circ}$ lies Zariski dense in $V$.

  But the Ueno locus is always Zariski closed, so 
  $V^{\circ}=V$ and
  $\dim \mathrm{Stab}(V) \ge 1$ from earlier.

  Let $B$ be the connected component of $\mathrm{Stab}(V)^{\circ}$
  containing $0$.
  If $\dim V = \dim B$, then $V$ is a translate of $B$ by a point of
  finite order, we are done.

  Otherwise, the theorem follows by induction on $\dim V$ on
  considering
  $\varphi(V)$ where 
    $\varphi\colon A\rightarrow A/B$ is the quotient morphism. \qed
\end{frame}


\section{Unconditional Andr\'e--Oort for $Y(1)^2$}

\begin{frame}{Recall from Monday}
  \begin{theorem}[Andr\'e--Oort for $Y(1)^2$]
    Let $C\subset Y(1)^2$ be an irreducible curve. Then
    \begin{equation*}
      C \cap Y(1)^2_{\mathrm{special}}\text{ is
        infinite}\quad\Longleftrightarrow\quad \text{$C$ is a special
        curve of $Y(1)^2$}. 
    \end{equation*}  
  \end{theorem}

  Recall also: the set of special points $Y(1)_{\mathrm{special}}$
  is the image under Klein's $j$-function of all $\tau\in\IH$ with
  $[\IQ(\tau):\IQ]$, \textit{e.g.},
  $$\tau = \sqrt{-D}$$
  when $-D\equiv 2,3 \mod 4$. 
  

  We would like to apply the same approach to prove Andr\'e--Oort.

  But $(\mathrm{Re}(\tau),\mathrm{Im}(\tau))$ is
  algebraic but \alert{no longer} rational!   
\end{frame}

\begin{frame}{Pila--Wilkie for Algebraic Points}
  \begin{definition}
    \begin{enumerate}
    \item [(i)] Let $x\in\IC$ be an algebraic number. The
      $\IZ$-minimal polynomial of $x$ is the
      unique polynomial $P=p_dX^d+\cdots+p_0\in \IZ[X]$ that is
      irreducible in $\IQ[X]$ with $p_d\ge
      1,
      \gcd(p_0,\ldots,p_d)=1,$ and $P(x)=0$. The \alert{height} of $x$
      is
      \begin{equation*}
        H(x) = \left(p_d \prod_{z\in \IC : P(z)=0}
          \max\{1,|z|\}\right)^{1/d}\ge 1
      \end{equation*}
      where the product runs over all complex roots of $P$.
    \item[(ii)] Let $(x_1,\ldots,x_m)$ with $x_j$
      algebraic for all $j$. The \alert{height} of
      $(x_1,\ldots,x_m)$ is
      $H(x_1,\ldots,x_m) = \max_{1\le j\le m}H(x_j)$. 
    \end{enumerate}
  \end{definition}
  The exponent $1/d$ is there for normalization purposes. So authors
  work with $h(x)=\log H(x)$. 
\end{frame}

\begin{frame}
  \begin{itemize}
  \item $H(2^{1/d}) = 2^{1/d}$, just take $P=X^d-2$.

  \item $H(\text{root of unity}) = 1$. Indeed, all roots of the
    $\IZ$-minimal polynomial of a root of unity are on the unit
    circle, moreover, the leading term is $1$.

  \item $H(\sqrt{3}/2) = 2$ as $P = 4X^2-3$ has both roots in
    $[-1,1]$.

  \item Lehmer's polynomial $P=X^{10}+X^9-X^7-X^6-X^5-X^4-X^3+X+1$ is
    is irreducible and has exactly one root $x=1.1762808\ldots$ outside the
    unit circle. So $H(x)^{10} = x$. Moreover, $x$ is the least known
    value $>1$ of $H(x)^{[\IQ(x):\IQ]}$. 
    
  \item If $x=p/q$ with $p,q\in\IZ,q\ge 1,\gcd(p,q)=1$, then
    $P=qX-p$ and $H(x) = q\max\{1,|p/q|\} = \max\{|p|,q\}$, as before. 

    But if $m\ge 2$ this bound \alert{differs}  from the one defined
    earlier. The difference is harmless for our purposes.
    
  \item The height is Galois invariant:
    $H(\sigma(x))=H(x)$ if $\sigma\in\mathrm{Gal}(\IQbar/\IQ)$. 
  \end{itemize}
\end{frame}

\begin{frame}
  Clearly, any root of unity is contained in
  \begin{equation*}
    \{x\in\IQbar : H(x)\le 1\}.
  \end{equation*}
  So a set of algebraic numbers of bounded height need \alert{not} be finite.
  
  But we can salvage Northcott's Theorem by in addition
  bounding the degree.
  
  \begin{theorem}[Northcott's Theorem]
    Let $d\ge 1$ and $T\ge 1$. The set
    \begin{equation*}
      \{x\in \IC : x \text{ is algebraic with }[\IQ(x):\IQ]\le d\text{
        and }H(x)\le T\}
    \end{equation*}
    is finite. 
  \end{theorem}
\end{frame}


\begin{frame}{The Pila--Wilkie Theorem for Algebraic Points}
  \begin{definition}
    Let $X\subset\IR^m$ be any subset, let $d\ge 1$, and $T\ge 1$. Set
    \begin{equation*}
      N_d (X,T) = \#\{ x\in X\cap \IQbar^m : [\IQ(x):\IQ]\le d
      \text{ and }
      H(x)\le T\}.
    \end{equation*}
  \end{definition}

  Note that $N_d(X,T)<\infty$ for all $T\ge 1$ by Northcott's Theorem. 

  \begin{theorem}[Pila]
    Let $\cS$ be an o-minimal structure, let $d\ge 1$, let $\epsilon
    >0$, and   let $X\subset\IR^m$ be
    definable in $\cS$. 
    There exists  $c(\epsilon,d,X)>0$ such that
    \begin{equation*}
      N_d(X\ssm X^{\mathrm{alg}},T)\le c(\epsilon,d,X) T^\epsilon \quad\text{for
        all}\quad T\ge 1.
    \end{equation*}
  \end{theorem}
\end{frame}

\begin{frame}
  \begin{lemma}
    Let $j(\tau)$ be a special point of $Y(1)$.
    Let $\Delta$ be the discriminant of
    the endomorphism ring of the elliptic curve attached to $j(\tau)$.
    Then $\tau$ is imaginary quadratic. Moreover, if $\tau =x+yi$ with
    $x,y\in\IR$ then $[\IQ(x,y):\IQ]\le 4$ and $H(x,y)\le 16|\Delta|^{5/2}$.
  \end{lemma}
  \begin{proof}\renewcommand{\qedsymbol}{}
    Set $E = \IC/(\IZ+\tau\IZ)$, then
    $\mathrm{End}(E) = \IZ + \omega \IZ$ where $\omega = f
    \frac{\sqrt{D}+D}{2}$ and $D$ is the discriminant of the field of
    fractions of $\mathrm{End}(E)$ and $f\in\IN$ is its conductor.
    Then $\Delta = f^2D$.
    \begin{equation*}
      \mathrm{Re}(\omega) = f\frac{D}{2} \quad\text{and}\quad
      \mathrm{Im}(\omega) = f\frac{\sqrt{|D|}}{2}
      \quad\text{and}\quad
      |\omega| \le \frac{|\Delta|}{\sqrt{2}}\le|\Delta|
    \end{equation*}
    There exist $a,b,c,d\in\IZ$ with
    $\omega=a+b\tau$ and $\omega\tau = c+d\tau$, so
    \begin{equation*}
      b\tau^2+(a-d)\tau-c=0.
    \end{equation*}
    So $H(\tau) \le (|b| \max\{1,|\tau|\}^2)^{1/2}$ as $b\not=0$.
  \end{proof}
\end{frame}

\begin{frame}
  \begin{proof}[Proof continued]       
    $$\frac{fD}{2}=\mathrm{Re}(\omega) =
    a+b\mathrm{Re}(\tau)\text{ and }\frac{f\sqrt{|D|}}{2}=\mathrm{Im}(\omega)
    = b \mathrm{Im}(\tau).$$
    
Since $\tau$ is in the fundamental domain we
    find $|\mathrm{Re}(\tau)|\le 1/2$ and $|\mathrm{Im}(\tau)|\ge
    \sqrt{3}/2$. Therefore,
    \begin{equation*}
      b=|b|\le {f \sqrt{|D|/3}} = \sqrt{|\Delta|/3}
    \end{equation*}
    and
    \begin{equation*}
    |a|\le \frac{b+f|D|}{2} \le \frac{\sqrt{|\Delta|/3} +
        |\Delta|}{2} \le |\Delta|. 
    \end{equation*}
    So
    $\tau = (\omega-a)/b$ and $|\tau|\le
    |\omega-a|\le|\omega|+|a| \le 2|\Delta| $.

    We find $H(\tau) \le (|\Delta|/3)^{1/4} (2|\Delta|)$.
    The estimates for $x=(\tau+\overline\tau)/2$ and $y=(\tau-\overline\tau)/(2i)$ follow from basic height
    inequalities. Finally, $\IQ(x,y) =\IQ(\tau,i)$ has degree $\le 4$
    over $\IQ$. 
  \end{proof}
\end{frame}

\begin{frame}
  Let $V\subset Y(1)^2$ be a curve defined over a number field $K$
  that contains infinitely many special points.
  We want to show that $V$ is special.
  
  The set
  \begin{equation}
    \label{def:Xao}
    X = \{(\tau_1,\tau_2) \in \cF^2 : (j(\tau_1),j(\tau_2)) \in V(\IC) \} =
    j|_{\cF^2}^{-1}(V(\IC))
  \end{equation}
  is \alert{definable} in $\IRanexp$.
  we also use $j$ to denote $j\times j\colon
  \IH^2\rightarrow\IC^2$.

  
  \begin{lemma}
    Let $(z_1,z_2)\in V(\IC)$ be special and let $\Delta_i$ denote the discriminant
    of the endomorphismring of the elliptic curve attached to $z_i$.
    If $\max\{|\Delta_1|,|\Delta_2|\}$ is sufficiently large in terms of
    $V$, then there exists $\sigma\in \absgalk$ with
    $\sigma(z_1,z_2)\in j(X^{\mathrm{alg}})$.   
  \end{lemma}

  The proof is as in the Manin--Mumford Theorem, but we need to use
  Pila's version of Pila--Wilkie for algebraic points (of degree $\le
  4$).
  Large Galois orbits are afforded by the \alert{Landau--Siegel} Theorem.
\end{frame}

\begin{frame}{Ax--Lindemann--Weierstrass for the $j$-function}
  But what about $j(X^{\mathrm{alg}})$?

  \begin{theorem}[Pila, 2011]
    Let $W\subset\IH^m$ be a connected neighborhood of the
    intersection of an
    irreducible subvariety of $\IC^m$ with $\IH^m$.
    Let $\tau_1,\ldots,\tau_m$ denote the coordinate functions of $\IH^m$.
    Suppose that $j\circ {\tau_1|}_{W},\ldots,j\circ {\tau_m}|_{W}$
    are \alert{not} algebrically independent.
    Then there exist $N\in\IN$ and 
    $k\not=l$ with $\Phi_N(j\circ \tau_k|_W,j\circ \tau_l|_W)=0$ as
    functions on $W$.    
  \end{theorem}
  \vspace{-.2cm}
  
  To apply this theorem we take $m=2$ and  $W$ comes from $\IH^2$
  intersected the complexification of a semi-algebraic curve in $X$.
  
  Observe that $j\circ \tau_1|_W$ and $j\circ \tau_2|_W$ are not
  algebraically independent since their image lies in the algebraic curve
  $V\subset Y(1)^2$.
  So one of $j\circ \tau_k|_W$ must be constant or the two coordinates
  are related by a modular transformation polynomial. In particular,
  $V$ is special.  
\end{frame}

\section{Further Developments}

\begin{frame}
  In his 2011 Annals of Math paper proved Andr\'e--Oort for
  all subvarieties of $Y(1)^m$.

  \begin{theorem}[Pila]
    Let $V$ be an irreducible subvariety of $Y(1)^m$. Then
    $$V(\IC)\cap
    Y(1)^{m}_{\mathrm{special}}\text{ is Zariski dense in
      $V$}\Longleftrightarrow V\text{ is special}$$
  \end{theorem}

  $V$ special in $Y(1)^m$ if, after a  permutating coordinates
  it is an irreducible component of the algebraic set defined by
%  the algebraic set determined by the following relations
  \begin{itemize}
  \item $X_{1},\ldots,X_{k_1-1}$ are constant on $V$ and take values in
    $Y(1)_{\mathrm{special}}$
  \item for $j\ge 1$ there exist $N_{j,1},N_{j,2},\ldots\in\IN$ with 
    $\Phi_{N_{1,1}}(X_{k_j},X_{k_j+1})=\Phi_{N_{1,2}}(X_{k_j+1},X_{k_1+2})
    = \cdots =\Phi_{N_{1,k_{j+1}-k_j+1}}(X_{k_{j+1}-2},X_{k_{j+1}-1})$
  \end{itemize}
  Coordinates ``packets'' are constant or linked by an isogeny:
  \vspace{-.2cm}
  \begin{center}
  $(X_1,\ldots,X_{k_1-1},X_{k_1},\ldots,X_{k_2-1},\ldots,
  X_{k_{s-1}},\ldots,X_m)$      
  \end{center}  
\end{frame}


\begin{frame}
  Pila's Theorem on Ax--Lindemann Weierstrass
  was generalized by Pila and Tsimerman to a (weak) Ax--Schanuel
  Theorem for the $j$-function. The formulation here is chosen to
  mimic Ax's Theorem.

  \begin{theorem}[Pila--Tsimerman]
    Let $\tau_1,\ldots,\tau_m\colon\Delta\rightarrow\IH$ be
    non-constant holomorphic maps on the open unit disk $\Delta$.
    Suppose that there does not exist $N\in\IN$ and distinct $k,l$
    such that $\Phi_N(j\circ\tau_k,j\circ \tau_l)=0$ holds
    identically. Then
    \begin{equation*}
      \mathrm{trdeg}\,\IC(\tau_1,\ldots,\tau_m,j\circ\tau_1,\ldots,j\circ\tau_m)/\IC
      \ge m+1.
    \end{equation*}
  \end{theorem}   
\end{frame}

\begin{frame}{Andr\'e--Oort for $\mathcal{A}_g$}
  The coarse moduli space $\mathcal{A}_g$ of principally polarized abelian varieties
  of dimension $g$ is a quotient of
  \begin{equation*}
    \IH_g = \{Z\in \mat{g}{\IC} : Z^t = Z \text{ and }
    \mathrm{Im}(Z) \text{ pos. definite}\}
  \end{equation*}
  by the action of the symplectic group $\mathrm{Sp}_{2g}(\IZ)$.
  
  \begin{definition}
    A point in $\mathcal{A}_g(\IC)$ is called \alert{special} or
    \alert{CM}
    if the abelian
    variety  it represents has complex
    multiplication. 
  \end{definition}

  Similarities to $Y(1)$:
  We know $\mathcal{A}_g$ is an irreducible quasi-projective variety of
  dimension $g(g+1)/2$ defined over $\IQ$.
  
  There is a uniformization map
    $\IH_g \rightarrow \mathcal{A}_g(\IC)$,
  special points lift to algebraic points in $\mat{g}{\IC}$ (of
  bounded degree), there are ``special subvarieties''.
  % From a technical point of view it makes sense to introduce level
  % structure. 
\end{frame}

\begin{frame}
  After many important contributions by:

  \begin{itemize}
  \item  Pila--Tsimerman, Klinger--Ullmo--Yafaev (Ax--Schanuel)
  \item   Andreatta--Goren--Howard--Madapusi-Pera, Yuan--Zhang (Colmez
    Conjecture on the average: Large Galois Orbit)
  \item Masser--Wüstholz (Endomorphism estimates: Large Galois Orbit)
  \item   Peterzil--Starchenko (definability)    
  \end{itemize}

  Tsimerman proved 
  \begin{theorem}[Tsimerman]
    Let $V$ be an irreducible subvariety of $\mathcal{A}_g$, then
    \begin{equation*}
      V(\IC)\cap \mathcal{A}_{g,\mathrm{special}} \text{ is Zariski
        dense in $V$}\Longleftrightarrow V\text{ is special.}
    \end{equation*}
  \end{theorem}
  
\end{frame}



\begin{frame}
  \begin{center}
    Thanks for your attention. The last lecture is on unlikely
    intersections. 
  \end{center}
\end{frame}

\end{document}
