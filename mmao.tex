\chapter{The Conjectures of Manin--Mumford and Andr\'e--Oort}

\section{Overview}

Let $A$ be an abelian variety defined over $\IC$.

\begin{definition}
  The \emph{torsion group} $A_{\mathrm{tors}}$ of $A$ is the subgroup of points
  of finite order of the abelian group $A(\IC)$.
\end{definition}

In this chapter we will sketch a proof of the following theorem,
originally due to Raynaud (who proved it without assume that the
everything in sight is defined over a number field).

\begin{theorem}[Manin--Mumford Conjecture]
  \label{thm:mmav}
  Suppose $K\subset\IC$ is a number field and $A$ is an abelian
  variety defined over $K$. 
  Let $V\subset A$ be an irreducible closed subvariety, also defined over a
  number field $K$. Then
  \begin{equation*}
    V(\IC) \cap A_{\mathrm{tors}} \text{ lies Zariski dense in
      $V$}\quad\Longleftrightarrow \quad
    \text{$V$ is a torsion coset of $A$.}
  \end{equation*}
\end{theorem}

Recall that a torsion coset of $A$ is the translate of an abelian
subvariety of $A$ by a point in $A_{\mathrm{tors}}$. 

This is the generalization of the translation to abelian varieties of
the Ihara--Serre--Tate Theorem, Theorem~\ref{thm:ist}. In view of this
analogy it is reasonable to declare torsion cosets of $A$ as the
special subvarieties of $A$.

There are many proofs of Theorem~\ref{thm:mmav}. We will follow the
o-minimal route and use the Pila--Wilkie Theorem (a second time). The
idea to use the Pila--Wilkie counting theorem in this context goes
back to Zannier. We will follow the proof by Pila and
Zannier~\cite{PilaZannier}. The basic idea is to reinterpret torsion
points of $A$ as points in $\mathrm{T}_0(\IC)$ that are rational with
respect to a fixed period lattice basis.

This powerful approach has many applications. For example, it led to
the first condition proof of the Andr\'e--Oort Conjecture for
subvarieties of $Y(1)^m$ by Pila~\ref{Pila:AO}. Later on, and after work of
many people including Daw, Klingler, Pila, Orr, Ullmo, Yafaev the
approach was extended to the moduli space of abelian varieties by
Tsimerman. Here important work of
Andreatta--Goren--Howard--Madapusi-Pera and Yuan--Zhang on the average
Colmez conjecture and Peterzil--Starchenko on definability was used.

We will stick to the case of subvarieties of $Y(1)^m$.
Recall that we defined special points in
Definition~\ref{def:specialY1m}. We will define special subvarieties
of $Y(1)^m$ here. The modular transformation polynomials $\Phi_N$
were introduced in Definition-Lemma~\ref{deflem:modtranspoly}. 

\begin{definition}
  An irreducible subvariety $V$ of $Y(1)^m$ is called \emph{special} if it is
  an irreducible component of the zero locus of
  \begin{equation*}
    \Phi_{N_j}(X_{i_j},X_{i'_j}) \quad (j\in \{1,\ldots,r\})
  \end{equation*}
  for some  $N_1\ldots,N_r\in\IN_0$ and $i_1,i'_1,\ldots,i_r,i'_r\in
  \{1,\ldots,r\}$. 
\end{definition}

\begin{example}
  \begin{enumerate}
  \item [(i)]
    For $m=3$ there is an irreducible component of the vanishing locus of
    $\Phi_2(X_1,X_1)$ and $\Phi_3(X_2,X_3)$ of the form
    \begin{alignat*}1
      \{(1728,x_2,x_3) : \,\,&\text{there is a degree $3$ isogeny between
        the elliptic curves over $\IC$}\\ &\text{with $j$-invariants $x_2$ and $x_3$}\}
    \end{alignat*}
    as the elliptic curve over $\IC$  of $j$-invariant 1728
    admits an endomorphisms of degree $2$. This endomorphism arises as
    $2 = (1+i)(1-i)$ in $\IZ[i]$. The other possible $X_1$ 
    coordinates are $8000$ (endomorphism ring $\IZ[\sqrt{-2}]$) and
    $-3375$ (endomorphism ring $\IZ[\sqrt{-7}/2+1/2]$).

    So allowing $i_j=i'_j$ in the definition of special subvariety of
    $Y(1)^m$ leads to constant coordinates. 
  \item[(ii)] The special points of $Y(1)^m$ are, taken as singletons,
    precisely the special subvarieties of dimension $0$.
  \item[(iii)] Note that $Y(1)^m$ is a special subvariety of $Y(1)^m$. 
  \end{enumerate}
\end{example}

\begin{theorem}[Andr\'e--Oort Conjecture for $Y(1)^m$, Pila's
  Theorem~\cite{Pila:AO}]
  \label{thm:ao}
  Let $V\subset Y(1)^m$ be an irreducible closed subvariety. Then
  \begin{equation*}
    V(\IC) \cap Y(1)^m_{\mathrm{special}} \text{ lies Zariski dense in
      $V$}\quad\Leftrightarrow \quad
    \text{$V$ is special.}
  \end{equation*}    
\end{theorem}

\section{Arithmetic of Torsion Points}
\label{sec:galoistorsionav}

Here we collect some fundamental facts about the torsion points of an
abelian variety $A$ defined over a number field $K\subset \IC$ of dimension $g$. We will
assume that $A$ is presented as a subvariety of $\IP^n$. 

As a complex Lie group $A(\IC)$ is isomorphic to
$\mathrm{T}_0(A)/\Omega$ where $\Omega$ is the period lattice, a
discrete subgroup of $\mathrm{T}_0(A)$ of rank $2g$. So the group
$A(\IC)$ is isomorphic to $(\IR/\IZ)^{2g}$.

Say $N\in\IN$ and let $A[N] =  \ker[N] = \{P\in A(\IC) : [N](P) = 0\}$, here
$[N]$ is the multiplication-by-$N$ morphism.
As a group we have $A[N] \cong (\IZ/N\IZ)^{2g}$ and in particular
$\#A[N] = N^{2g}$.

Let $\overline K$ be the algebraic closure of $K$ in $\IC$.
Then $A[N] = \{P\in A(\overline K) : [N](P)=0\}$. In fact, all torsion
points of $A$ are $\overline K$-valued points, \textit{i.e.},
$A_{\mathrm{tors}}\subset A(\overline K)$. 

Let $\absgalk$ denote the absolute Galois group of $K$. Let
$P\in \IP^n(\overline{K})$. Fix  $(x_0,\ldots,x_n) \in
\overline{K}^{n+1}\ssm\{0\}$ with
 $P=[x_0:\cdots:x_n]$. Say $\sigma\in \absgalk$, then
\begin{equation*}
\sigma(P)=  \sigma([x_0:\cdots:x_n]) = [\sigma(x_0):\cdots:\sigma(x_m)]
\end{equation*}
defines an action of $\absgalk$ on $\IP^n(\overline K)$; the point
$\sigma(P)$ is independent of the choice of projective coordinates. 

Clearly, $\sigma(P)=P$ for all $\sigma$ if $P\in \IP^n( K)$. If
conversely, $P\in \IP^n(K)$ with $\sigma(P)=P$ for all $\sigma$ then
we have $P\in \IP^n(K)$; indeed, we may assume that one projective
coordinate equals $1$.

\begin{definition}
  For $P\in \IP^n(\overline K)$. 
  We set $K(P)$ to be the fixed field of the stabilizer
  $\{\sigma\in \absgalk : \sigma(P)=P\}$. 
\end{definition}

We have
\begin{equation*}
  [K(P):K]  = \#\{\sigma(P) : \sigma\in\absgalk \}. 
\end{equation*}

If $P=[x_0:\cdots:x_n] \in \IP^n(\overline K)$ with
$x_0,\ldots,x_n\in\overline K$ and $x_i=1$ for some
$i$, then $K(P) = K(x_0,\ldots,x_n)$. 

\begin{exercise}
  Let $W\subset\IP^n$ be an irreducible subvariety defined
  over $\overline K$. Find a sensible definition for $\sigma(W)$,
  for $\sigma\in \absgalk$.  Show that $W$ is the zero locus of
  polynomials with coefficients in $K$ if and only if $\sigma(W)=W$
  for all $\sigma\in \absgalk$.
\end{exercise}


As $A$ is cut out in $\IP^n$ by homogeneous polynomials with
coefficients in $K$ we find that the action of $\absgalk$ restricts to
an action on $A(\overline K)$ that fixes $A(K)$.

The multiplication-by-$N$ morphism
$[N]\colon A\rightarrow A$ is, at least Zariski-locally, presented
by a collection of homogeneous polynomials. That is, in a Zariski
neighborhood of any point $P_0\in A(\overline K)$ there exist
$f_{N,0},\ldots,f_{N,n}\in K[X_0,\ldots,X_n]$, all homogeneous of the
same degree, such that $[N](P) = [f_{N0}(P):\cdots:f_{Nn}(P)]$ for all
$\overline K$-points $P$ in the said neighborhood. Patching together
neighborhoods yields We have
$$\sigma([N](P)) = [f_{N0}(\sigma(P)):\cdots:f_{Nn}(\sigma(P))] =
[N](\sigma(P))$$
holds for \textit{all} $P\in A(\overline K)$.


\begin{lemma}\label{lem:galoisactionav}
  \begin{enumerate}
  \item [(i)]
    The
    action of $\absgalk$ on $A(\overline K)$ restricts to an action on
    $A[N]$ for all $N\in\IN$.
  \end{enumerate}
  In particular, $\absgalk$ restricts to
  an action on $A_{\mathrm{tors}}$. 
  Let us denote this action by
  $\rho_N\colon \absgalk\rightarrow \mathrm{Aut}(A[N])$.
  \begin{enumerate}
  \item[(ii)]
    The extension $K(A[N])/K$ is Galois
    of degree at most $N^{4g^2}$.
  \item[(iii)] Suppose $P\in A[N]$, then $K(P)/K$ has degree at most
    $N^2$. 
  \end{enumerate}
\end{lemma}
\begin{proof}
  Part (i) has proved before the lemma.  For part (ii) observe that
  $A[N]\cong
  (\IZ/N\IZ)^{2g}$ implies
  $\#\mathrm{Aut}(A[N])\le \mat{2g}{\IZ/N\IZ}$.
  So  $\rho_N$ has finite image with order at most $N^{4g^2}$. The kernel
  of $\rho_N$ is the Galois group $\mathrm{Gal}(\overline K/K(A[N]))$.
  So $K(A[N])/K$ is normal of degree at most $N^{4g^2}$.
  Part (iii) follows from (i) and as $\#A[N]=N^2$  
\end{proof}

This is just the analog of the action of the Galois group $\mathrm{Gal}(\IQbar/\IQ)$
on roots of unity in $\IC^\times$ that we investigate in
Section~\ref{sec:rootsof1}.


For the sake of completeness let us state here a theorem of Serre for
elliptic curves.

\begin{theorem}[Serre]
  Suppose $E$ is an elliptic curve defined over $K$ such that $E$,
  taken as defined over $\overline K$ does
  not have complex multiplication. There exists $c(E)>0$ such that
  if $p\ge c(E)$ is prime number, then $\rho_p \colon
  \absgalk\rightarrow \mathrm{Aut}(E[p]) \cong \mathrm{GL}_2(\IF_p)$
  is surjective.  
\end{theorem}

This theorem illustrates some some notable differences between the
arithmetic of roots of unity and torsion points on abelian varieties.
First, in contrast to the multiplicative setting, there is no reason
to believe that $K(A[N])/K$ is an abelian extension and in general it
is not. In general, $K(A(P))/K$ is not Galois for a torsion point $P$
of $A$. In the context of
Serre's Theorem and for all $p$ large enough, the stabilizer of $P \in
E[p] \ssm\{0\}$ in $\mathrm{Gal}(K(E[p])/K)\cong \mathrm{GL}_2(\IF_p)$
is a group of upper triangular matrices. It is not normal in
$\mathrm{Gal}(K(E[p])/K)$.

\begin{exercise}
  \begin{enumerate}
  \item [(i)]  Let $p$ be a sufficiently large prime and let $E$ be
    an elliptic curve as in Serre's Theorem. Say $P\in E(\overline K)$ has
    order $p$. Show that $[K(P):K] = p(p-1)$.
  \item[(ii)] Show why Serre's Theorem fails if $E$ has complex
    multiplication.
  \item[(iii)] Let $E$ be an elliptic curve defined over $K$  with
    complex multiplication. Let $c>0$ and $\delta>0$ with
    $[K(P):K]\ge c N^{\delta}$ for all points $P\in E(\overline K)$ of
    finite order $N$. Show that $\delta \le 1$. 
  \end{enumerate}
\end{exercise}

But in general there is no simple formula for the degree of the
extension $K(A(P))/K$ if $P$ has order $N$. At least Serre's Theorem
suggests that the degree $[K(P):K]$ of a point of order $N$ grows
polynomially in terms of $N$, at least in some cases..

Using methods from transcendence theory, Masser proved the following
estimate.

\begin{theorem}[Masser~Corollary~\cite{Masser:smallvalues}]
  \label{thm:masser}
  Let $A$ be an abelian variety defined over a number field $K$. There
  exist $c=c(A)>0$ and $\delta=\delta(A)>0$ with the following
  property. If $P \in A_{\mathrm{tor}}$ has order $N$, then
  $[K(P):K]\ge c N^{\delta}$. 
\end{theorem}


% Now suppose $V$ is an  subvariety of $A$ defined over $K$. We can
% think of $V$ as the zero locus on $A(\overline K)$ of homogeneous
% polynomials with coefficients in $K$. 

% It is important to distinguish between $A[N]$ and the subgroup
% $A(K)[N] = \{P\in A(K) : [N](P)=0\}$. Indeed, the subgroup $A(K)[N]$ may be
% smaller than $A(K)$.


\section{Proof of the Manin--Mumford Conjecture}

Let $A$ be an abelian variety defined over a number field
$K\subset\IC$.

The direction ``$\Longrightarrow$'' of Theorem~\ref{thm:mmav} is covered in
the following lemma.

\begin{lemma}
  Suppose $V\subset A$ is a torsion coset. Then $V(\overline K)\cap
  A_{\mathrm{tors}}$ lies Zariski dense in $V$. 
\end{lemma}
\begin{proof}
  Let $B$ be an abelian subvariety of $A$. 
  As a topological group, $B(\IC)$ is isomorphic to
  $(\IR/\IZ)^{2\dim B}$. The torsion $B_{\mathrm{tors}}$ is identified
  with $(\IQ/\IZ)^{2\dim B}$ under an isomorphism. In particular,
  $B_{\mathrm{tors}}$ lies dense in $B(\IC)$ equipped with the
  archimedean topology. So in particular, $B_{\mathrm{tors}}$ lies
  Zariski dense in $B$. If $P\in A(\IC)$ has finite order, then
  $p+B_{\mathrm{tors}}$ lies Zariski dense in $p+B$. 
\end{proof}

The converse is the deep part. 
We return again to
the notation introduced in Example~\ref{ex:thetafunc}.
\begin{itemize}
\item We have an immersion $A\rightarrow\IP^n$
\item We have a uniformizing map $u\colon T_0(A)\rightarrow
  A(\IC)$  with the tangent space
  $\mathrm{T}_0(A)$. The map $u$ is a holomorphic, surjective group
  homomorphism.
\item The kernel $\ker u$ is a discrete subgroup $\Omega\subset\IC^g$
  of rank $2g$, the period lattice. We fix a basis $(\omega_1,\ldots,\omega_{2g})$ of
  $\Omega$ and obtain a fundamental domain.
\end{itemize}

% In this setup it has already proved useful to work in 2 distinct
% coordinate systems in $T_0(A)$.

\bigskip
\noindent The \textbf{complex coordinates} arise via an isomorphism
$\mathrm{T}_0(A)\rightarrow\IC^g$ of $\IC$-vector spaces. 



\bigskip
\noindent The \textbf{period coordinates} arise via the isomorphism
$\mathrm{T}_0(A)\rightarrow \IR^{2g}$ of $\IR$-vector spaces
induced by the basis $(\omega_1,\ldots,\omega_{2g})$. 

Let us see how to go back and forth between these two coordinates.

\begin{lemma}
  \label{lem:rationalpreimage}
  Let $P\in A(\IC)$ be a torsion point of order $N$.
  There exist $x\in \IQ^{2g}$ with $H(x)=N$ and $u(x)=P$. 
\end{lemma}
\begin{proof}
  Let $\lambda_1,\ldots,\lambda_{2g}\in [0,1)$ with
  $u(\lambda_1\omega_1+\cdots+\lambda_{2g}\omega_{2g}) = P$.
  As $[N](P)=0$ we have
  $u(N\lambda_1\omega_1+\cdots +N\lambda_{2g}\omega_{2g})=0$.
  So $N\lambda_1\omega_1+\cdots +N\lambda_{2g}\omega_{2g}$ lies in the
  period lattice $\Omega$. Thus there exist $p_j\in \IZ$ with
  $\lambda_j\in p_j/N$ for all $j\in\{1,\ldots,2g\}$.
  In period coordinates the point
  $\lambda_1\omega_1+\cdots+\lambda_{2g}\omega_{2g}$ becomes
  $x=(\lambda_1,\ldots,\lambda_{2g})\in \IQ^{2g}\cap [0,1)$.
  Because the order of $P$ is $N$ we find $\gcd(p_1,\ldots,p_{2g})=N$
  and so $H(x)=N$. 
\end{proof}


Let $V\subset A$ be an irreducible subvariety defined over $K$. Recall
that the Ueno locus $V^{\circ}$ was introduced  in Definition~\ref{def:ueno}.

\begin{lemma}
  \label{lem:apppw}
  There exists $N_0(V,A)$ with the following property. If
  $P\in V(\overline K)\cap A_{\mathrm{tors}}$ has order $N$
  and $N\ge N_0(V,A)$, then $P\in V^{\circ}$.
\end{lemma}
\begin{proof}
  % We will assume that $V(\IC)\cap A_{\mathrm{tors}}$ lies Zariski dense
  % in $V$. We will also assume that $\dim V\ge 1$ as the result is
  % trivial for $V$ a point.

  Working in period coordinates, we set
  \begin{equation*}
    X = u^{-1}|_{[0,1)^{2g}}(V(\IC)) \subset \IR^{2g}.
  \end{equation*}
  As we have seen in Example \ref{ex:thetafunc},
  the set $X$ is definable in $\IRan$. 


  Next we may use of the Galois action introduced in
  Section~\ref{sec:galoistorsionav}.

  Let $\sigma\in \absgalk$. As $V$ is ultimately defined by the
  vanishing of homogeneous polynomials, we find that $\sigma(P)\in
  V(\overline K)$ for all $P\in V(\overline K)$.

  Let $P\in V(\overline K) \cap A_{\mathrm{tors}}$ be a point of order
  $N$. As torsion points lie Zariski dense in $V$ by assumption and as
  $V(\overline K)$ is infinite, we may assume that $N$ is large in terms
  of any quantity derived from $V$.

  Say $\sigma\in \absgalk$ then $\sigma(P)\in V(\overline K)$ by the
  comment above. Moreover, $\sigma(P)\cap A_{\mathrm{tors}}$ by
  Lemma~\ref{lem:galoisactionav}(i). So $\sigma(P) \in V(\overline
  K)\cap A_{\mathrm{tors}}$ for all $\sigma\in \absgalk$. The number of
  Galois conjugates is $[K(P):K]\ge c(A) N^{\delta(A)}$ by
  Theorem~\ref{thm:masser}. 

  For any $\sigma$, Lemma~\ref{lem:rationalpreimage} provides $x_\sigma
  \in X\cap\IQ^{2g}$ of height $N$ with $u(x_\sigma) = \sigma(P)$. The
  number of rational points we obtain in this way is at least
  \begin{equation}
    \label{eq:rtlptub}
    c(N) N^{\delta(A)}. 
  \end{equation}

  Now we are ready to apply the Pila--Wilkie Theorem,
  Theorem~\ref{thm:pilawilkie}. It states
  \begin{equation}
    \label{eq:rtlptlb}
    N(X\ssm X^{\mathrm{alg}},N) \le
    c(X,\epsilon)N^\epsilon 
  \end{equation}
  for all $N\ge 1$, here $c(X,\epsilon)$ depends on $X$ and on $\epsilon
  > 0$.

  We set $\epsilon = \delta/2$, which is an invariant that depends only
  on $A$. We compare (\ref{eq:rtlptub}) and (\ref{eq:rtlptlb}). As we
  may assume that $N$ is large we must have $x_\sigma \in
  X^{\mathrm{alg}}$ for some $\sigma\in \absgalk$. 
  So $\sigma(P) = u(x_\sigma)  \in u(X^{\mathrm{alg}})$.

  Now we apply Theorem~\ref{thm:imagealg}, a consequence of the
  Ax--Lindemann--Weierstrass. It follows that $\sigma(P)$ lies in a
  coset $\sigma(P)+B$ here $B$ is an abelian subvariety of $A$ with
  $\dim B\ge 1$. As $P$ and also $\sigma(P)$ are torsion points we see
  that $\sigma(P)+B$ is a torsion coset.
\end{proof}


\begin{proof}[Proof of ``$\Longrightarrow$'' in
  Theorem~\ref{thm:mmav}]
  The proof is by induction on $\dim A$. The case $A=0$ is trivial. 
  We may also assume that $\dim V\ge 1$.

  All but finitely many torsion
  points in $A_{\mathrm{tors}}$ have order at least $N(A,V)$, the
  constant from Lemma~\ref{lem:apppw}.
  Say $V(\overline K)\cap A_{\mathrm{tors}}$ lies Zariski dense in
  $V$. By Lemma~\ref{lem:apppw} 
  the Ueno locus $V^{\circ}$ lies Zariski dense in $V$.

  By Theorem~\ref{thm:ueno}(ii) we have $V^{\circ}=V$ and part (iii)
  yields $\dim \mathrm{Stab}(V) \ge 1$.

  Let $B$ be the connected component of $\mathrm{Stab}(V)^{\circ}$
  containing $0$. Then $B$ is an abelian subvariety of $A$, defined
  over $\overline K$.

  If $\dim B = \dim V$, then $V$ is the translate of $B$ by  a point
  of finite order. We are done in this case.
  
  So assume $\dim B<\dim V$. 
  Let $\varphi\colon A\rightarrow A/B$ denote the quotient morphism.
  After replacing $K$ by a larger number field we may assume that
  everything is defined over $K$.
  The image $\varphi(V)$ is a subvariety of $A/B$. It contains a
  Zariski dense set of torsion points, the image of $V(\overline
  K)\cap A_{\mathrm{tors}}$ in $(A/B)(\overline K)$.
  Now $\dim \varphi(V) = \dim V-\dim B<\dim V$. By induction on $\dim V$,
  we conclude that $\varphi(V)$ is a torsion coset of $A/B$.
  But as the fibers of $\varphi|_V$ are translates of $B$ we can show 
  that $V$ is a torsion coset of $A$. This concludes the proof. 
\end{proof}

\section{Unconditional Andr\'e--Oort for $Y(1)^2$}

Here we sketch a proof of Theorem~\ref{thm:ao} for $m=2$. Here $\cF
\subset\IH$ is the fundamental domain introduced in
Example~\ref{ex:jfunction}. The special points of $Y(1)^2$ were
defined in Definition~\ref{def:specialY1m}. We recall
Convention~\ref{conv:identCCRR2}. 


The special points of $Y(1)^m$ are images of points in $\IH^m$ with
imaginary quadratic coordinest.
% correspond to point in $X$ with imaginary quadratic
% entries. 
We will require an appropriate version of the Pila--Wilkie
Theorem for algebraic points of higher degree.

Let us define the height of an algebraic number.

\begin{definition}
  \label{def:height2}  
  \begin{enumerate}
  \item [(i)] Let $x\in\IC$ be an algebraic number. There exists
    a unique polynomial $P=p_dX^d+\cdots+p_0\in \IZ[X]$ that is
    irreducible in $\IQ[X]$ with $p_d\ge
    1,
    \gcd(p_0,\ldots,p_d)=1,$ and $P(x)=0$. The \emph{height} of $x$
    is
    \begin{equation*}
      H(x) = \left(p_d \prod_{z\in \IC : P(z)=0} \max\{1,|z|\}\right)^{1/d}
    \end{equation*}
    where the product runs over all complex roots of $P$.
  \item[(ii)] Let $(x_1,\ldots,x_m)$ with $x_j$
    algebraic for all $j$. The \emph{height} of
    $(x_1,\ldots,x_m)$ is
    $H(x_1,\ldots,x_m) = \max_{1\le j\le m}H(x_j)$. 
  \end{enumerate}
\end{definition}

\begin{remark}
  \label{rmk:heightprops}
  \begin{enumerate}
  \item [(i)] The exponent $1/d$ is there for normalization purposes.

  \item[(ii)] If $m=1$ and $x_1\in\IQ$ then the height in part (i) of
    Definition~\ref{def:height2} is equivalent to the height from
    Definition~\ref{def:height1}(i). But for $m\ge 2$ we \textbf{do not} in
    general get the same value for $(x_1,\ldots,x_m)\in\IQ^m$.
    \footnote{I choose this
      slighty different height here purely for expository reasons.}
    However, both heights are
    bounded in terms of another by a positive multiplicative factor that may
    depend on $m$.
    The disrepancy is harmless in the context of our applications.

  \item[(iii)] The height is Galois invariant, \textit{i.e.}, if
    algebraic numbers $x$
    and $x'$ are Galois conjugate, then $H(x)=H(x')$.

  \item[(iv)] We record two inequalities that help to bound from above
    the height. For algebraic numbers
    $x$ and $x'$ we have
    \begin{equation*}
      H(x+x')\le 2H(x)H(x')\quad\text{and}\quad H(xx')\le H(x)H(x').
    \end{equation*}
    We refer to Chapter 1.5~\cite{BG} for proofs.     
  \end{enumerate}
\end{remark}

\begin{example}
  \begin{enumerate}
  \item [(i)] We have $H(2^{1/d})=2^{1/d}$. So clearly, the are
    infinitely many algebraic numbers of height at most $2$. 

  \item[(ii)] If $\zeta$ is a root of unity, then $H(\zeta)=1$.
  \end{enumerate}
\end{example}

\begin{lemma}
  \label{lem:altheight}
  Let $K$ be a number field and $x\in K$, then
  $$H(x) = \prod_{\substack{v\text{ place} \\\text{of $K$}}}
  \max\{1,|x|_v\}^{[K_v:\IQ_p]/[K:\IQ]}.$$
\end{lemma}
\begin{proof}
  See Proposition 1.6.6~\cite{BG}.
\end{proof}

\begin{exercise}
  \begin{enumerate}
  \item Prove $H(x)=H(1/x)$ for all $x\in\IQbar^\times$, you only need
    Definition~\ref{def:height2}.
  \item Prove $H(x^k) = H(x)^k$ for all $x\in\IQbar$ and $k\in\IN$,
    you may need to use Lemma~\ref{lem:altheight}.
  \item Prove $H(xy) \le H(x)H(y)$ and
    $H(xy)\le 2H(x)H(y)$ for all $x,y\in\IQbar$,
    you may need to use Lemma~\ref{lem:altheight}.
    
  \end{enumerate}
\end{exercise}

A number field has 2 kind of places. A finite place $v$ of $K$ can be
identified with a maximal ideal of the ring of integers of $K$. As
such it contains a unique rational prime $p$ and our convention is
$|p|_v=1/p$. An infinite place $v$ of $K$ can be identified with a
ring homomorphism $\sigma\colon K\rightarrow\IC$ up-to composition
with complex conjugation; here $|x|_v=|\sigma(x)|$.


While  there are infinitely many algebraic numbers of
bounded height we obtain finiteness after imposing a bound on the
degree.

\begin{theorem}[Northcott's Theorem]
  \label{thm:northcott}
  Let $d\ge 1$ and $T\ge 1$. The set
  \begin{equation*}
    \{x\in \IC : x \text{ is algebraic with }[\IQ(x):\IQ]\le d\text{
      and }H(x)\le T\}
  \end{equation*}
  is finite. 
\end{theorem}
\begin{proof}
  Let $P$ be the polynomial in Definition~\ref{def:height2} of an
  algebraic $x$ with $[\IQ(x):\IQ]\le d$ and $H(x)\le T$.

  The leading term of $P$ is at most $T^d$. We can bound the remaining
  terms of $P$ by observing that $\max\{1,|x'|\}$
  is also at most $T^d$ for all complex roots of $x'$.
  Finally, the coefficients of $P$ are, up-to the lead term, symmetric
  functions in the complex roots. So we get a bound for the absolute
  value of all coefficients of $P$ in terms of $T$ and $d$. As these
  coefficients are integers there are at most finitely many $P$ in
  question. So there are at most finitely many $x$. 
\end{proof}


\begin{exercise}
  Using Northcott's Theorem and $H(x^k)=H(x)^k$ show Kronecker's
  Theorem: If $x\in\IQbar$ with $H(x)=1$, then $x=0$ or $x$ is a root
  of unity. 
\end{exercise}

Next we make the statement of Lemma~\ref{lem:tauimagquad} on $\tau$
quantitative. This allows us to bound the height of $\tau$ if $\tau$
lies in the fundamental domain $\cF\subset \IH$ and $j(\tau)$ is special.

\begin{lemma}
  \label{lem:heighttaubound}
  Let $j(\tau)$ be a special point of $Y(1)$ with $\tau\in\cF$.
  Let $\Delta$ be the discriminant of
  the endomorphism ring of the elliptic curve attached to $j(\tau)$.
  Then $\tau$ is imaginary quadratic. Moreover, if $\tau =x+yi$ with
  $x,y\in\IR$ then $[\IQ(x,y):\IQ]\le 4$ and $H(x,y)\le 16|\Delta|^{5/2}$.
\end{lemma}
\begin{proof}  
  We already know from Lemma~\ref{lem:tauimagquad} that $\tau$ is
  imaginary quadratic.

  Let $E$ be the elliptic curve with $j$-invariant $j(\tau)$. Then
  $\mathrm{End}(E) = \IZ + \omega \IZ$ where $\omega = f
  \frac{\sqrt{D}+D}{2}$ and $D$ is the discriminant of the field of
  fractions of $\mathrm{End}(E)$ and $f\in\IN$ is its conductor.  
  Then $\Delta = f^2D$. Observe that
  \begin{equation*}
    \mathrm{Re}(\omega) = f\frac{D}{2} \quad\text{and}\quad
    \mathrm{Im}(\omega) = f\frac{\sqrt{|D|}}{2}
    \quad\text{and}\quad
    |\omega| \le \frac{|\Delta|}{\sqrt{2}}\le|\Delta|
  \end{equation*}
  for the real and imaginary parts. 

  As $\omega 1 \in\IZ+\tau\IZ$ there exist $a,b,c,d\in\IZ$ with
  $\omega=a+b\tau$ and $\omega\tau = c+d\tau$.
  So $\frac{fD}{2}=\mathrm{Re}(\omega) =
  a+b\mathrm{Re}(\tau)$ and $\frac{f\sqrt{|D|}}{2}=\mathrm{Im}(\omega)
  = b \mathrm{Im}(\tau)$. Since $\tau$ is in the fundamental domain we
  find $|\mathrm{Re}(\tau)|\le 1/2$ and $|\mathrm{Im}(\tau)|\ge
  \sqrt{3}/2$. Therefore,
  \begin{equation*}
    b=|b|\le {f \sqrt{|D|/3}} = \sqrt{|\Delta|/3}
    \quad\text{and}\quad
    |a|\le \frac{b+f|D|}{2} \le \frac{\sqrt{|\Delta|/3} +
      |\Delta|}{2} \le |\Delta|. 
  \end{equation*}
  Moreover, $b\not=0$ as $\omega\not\in\IR$. So
  $\tau = (\omega-a)/b$ and $|\tau|\le
  |\omega-a|\le|\omega|+|a| \le 2|\Delta| $.
%  This completes the estimates for $a,b,c$. 

  The polynomial $bX^2+(a-d)X-c$ is an integral multiple of the
  polynomial $P$ of Definition~\ref{def:height2} applied to $\tau$.
  So $H(\tau)^2 \le b\max\{1,|\tau|\}^2 \le |\Delta/3|^{1/2}
  (2|\Delta|)^2 \le 4|\Delta|^{5/2}$.
  
  Finally, note that if $\tau=x+yi$ with $x,y\in\IR$, then $x =
  (\tau+\overline\tau)/2$ and $y = (\tau-\overline \tau)/(2i)$. 
  Thus $\IQ(x,y)\subset \IQ(\tau,i)$ and thus $[\IQ(x,y):\IQ]\le 4$. 
  We use Remarks~\ref{rmk:heightprops}(iii) and (iv) to estimate
  \begin{equation*}
    H(x) = H((\tau+\overline\tau)/2)\le 2H(\tau+\overline\tau)\le
    4H(\tau)H(\overline\tau) = 4H(\tau)^2.
  \end{equation*}
  A similar computation yields $H(y)\le 4H(\tau)^2$. So
  $H(x,y)\le 4H(\tau)^2 \le 16 |\Delta|^{5/2}$, as desired. 
\end{proof}


Next let us formula the Pila--Wilkie Theorem for algebraic numbers in
$\IR^m$. For this we need to extend the definition of the counting
function in Definition~\ref{def:countingfunc} to algebraic points of
given degree.

\begin{definition}
  \label{def:countingfunc2}
  Let $X\subset\IR^m$ be any subset, let $d\ge 1$, and $T\ge 1$.
  The \emph{counting
    function} is 
  \begin{equation*}
    N_d (X,T) = \#\{ x\in X\cap \IQbar^m : [\IQ(x):\IQ]\le d
    \quad\text{and}\quad
    H(x)\le T\};
  \end{equation*}
  note that $N_d(X,T)<\infty$ by Northcott's Theorem. 
\end{definition}

The Pila--Wilkie Theorem holds verbatim for algebraic points of
bounded degree. 

\begin{theorem}[Pila, Theorem 1.6~\cite{Pila:AlgPoints}]
  \label{thm:pilaalg}
  Let $\cS$ be an o-minimal structure, let $d\ge 1$, let $\epsilon
  >0$, and   let $X\subset\IR^m$ be
  definable in $\cS$. 
  There exists  $c(\epsilon,d,X)>0$ such that
  \begin{equation*}
    N_d(X\ssm X^{\mathrm{alg}},T)\le c(\epsilon,d,X) T^\epsilon \quad\text{for
      all}\quad T\ge 1.
  \end{equation*}
\end{theorem}

We now  sketch a proof  of Theorem~\ref{thm:ao} for $m=2$.
We will only prove the deeper direction ``$\Longrightarrow$''.
If $V$ is a point or if $V=Y(1)^2$ the conclusion is clear.

So suppose that $V$ is a curve that  contains infinitely many special points
of $Y(1)^2$. As special points are algebraic numbers we find as in Section~\ref{sec:ist} that
$V$ is defined over a number field $K$, \textit{i.e.}, it is the
zero locus of a polynomial $A \in K[X,Y]$.

We set
\begin{equation}
  \label{def:Xao}
  X = \{(\tau_1,\tau_2) \in \cF^2 : (j(\tau_1),j(\tau_2)) \in V(\IC) \} =
  j|_{\cF^2}^{-1}(V(\IC)),
\end{equation}
where we sometimes use $j$ to denote the product $j\times j\colon
\IH^2\rightarrow\IC^2$. The set $X$ is definable in $\IRanexp$, see
Example~\ref{ex:jfunction}.
As is usual we identify $\IH$ with $\IR \times (0,\infty)$. 

The following lemma is the analog of Lemma~\ref{lem:apppw}. 
\begin{lemma}
  \label{lem:apppwao}
  Let $V\subset Y(1)^2$ be a geometrically irreducible curve defined
  over a number field $K$.
  Let $(z_1,z_2)\in V(\IC)$ be special and let $\Delta_i$ denote the discriminant
  of the endomorphismring of the elliptic curve attached to $z_i$.
  If $\max\{|\Delta_1|,|\Delta_2|\}$ is sufficiently large in terms of
  $V$, then there exists $\sigma\in \absgalk$ with
  $\sigma(z_1,z_2)\in j(X^{\mathrm{alg}})$.   
\end{lemma}
\begin{proof}%[]    
  We fix $(\tau_1,\tau_2) \in X$ with $j(\tau_1,\tau_2)=(z_1,z_2)$.
  In real coordinates $(\tau_1,\tau_2)$ corresponds to
  $(x_1,y_1,x_2,y_2) \in (\IR\times(0,\infty))^2$ where
  \begin{equation*}
    [\IQ(x_1,y_1,x_2,y_1):\IQ]\le 16
    \quad\text{and}\quad
    H(x_1,y_1,x_2,y_2) \le 16 \max\{|\Delta_1|,|\Delta_2|\}^{5/2}
  \end{equation*}
  by Lemma~\ref{lem:heighttaubound}.

  As in the proof of Theorem~\ref{thm:mmav} we can exploit the action of the
  Galois group to obtain new special points on $V$. Indeed, for all
  $\sigma\in \absgalk$ we have $\sigma(z_1,z_2) \in
  V(\IC)\cap Y(1)^2_{\mathrm{special}}$. We also find
  $(x_{1,\sigma},y_{1,\sigma},x_{2,\sigma},y_{2,\sigma})\in(\IR\times(0,\infty))^2$
  of degree at most $16$, height at most
  $16\max\{|\Delta_1|,|\Delta_2|\}^{5/2}$, and   
  that correspond to points in $\cF^2$.
  The number of points we obtain is
  \begin{equation*}
    [K(z_1,z_2):K] \ge \frac{[\IQ(z_1,z_2):\IQ]}{[K:\IQ]} \ge
    \frac{\max\{[\IQ(z_1):\IQ],[\IQ(z_2):\IQ]}{[K:\IQ]}
    \ge \frac{c_1}{[K:\IQ]}  \max\{|\Delta_1|,|\Delta_2|\}^{1/4},
  \end{equation*}
  the last bound is a suitable generalization of the Large Galois
  Orbit Lemma~\ref{lem:lgocm}
  to possibly non-maximal orders. The exponent $1/4$ can be increased
  to any value below $1/2$ at the cost of increasing the
  multiplicative constant $c_1$. But $1/4$, just as any fixed and
  positive exponent, is good enough for our purposes.


  We conclude
  \begin{equation}
    \label{eq:aonlb}
    N_{16}(X,T) \ge \frac{c_1}{[K:\IQ]} \max\{|\Delta_1|,|\Delta_2|\}^{1/4}
  \end{equation}
  for $T = 16 \max\{|\Delta_1|,|\Delta_2|\}^{5/2}\ge 1$. 

  We apply Pila's strengthening, Theorem~\ref{thm:pilaalg},  of the
  Pila--Wilkie Theorem to $d=16$ this $T$ and conclude
  \begin{equation}
    \label{eq:aonub}
    N_{16}(X\ssm X^{\mathrm{alg}}) \le c(X,\epsilon) T^\epsilon
    = c(X,\epsilon) (16\max\{|\Delta_1|,|\Delta_2|\}^{5\epsilon/2}.
  \end{equation}

  On comparing (\ref{eq:aonlb}) and (\ref{eq:aonub}) we want to 
  choose $\epsilon$ such that $5\epsilon/2
  <1/4$, say $\epsilon = 1/20$.
  So
  \begin{equation*}
    N_{16}(X\ssm X^{\mathrm{alg}}) \le
     c(X,\epsilon) (16\max\{|\Delta_1|,|\Delta_2|\}^{1/8}.
   \end{equation*}
   
  Thus if
  $\max\{|\Delta_1|,|\Delta_2|\}$ is sufficiently large in terms of
  $c_1,c(X,1/20),$ and $[K:\IQ]$, then  $\sigma(z_1,z_2) \in
  j(X^{\mathrm{alg}})$ for some $\sigma\in \absgalk$.  
\end{proof}

\begin{proof}[Sketch of Proof of Theorem~\ref{thm:ao} for $m=2$]
  Let $V$ be an irreducible curve in $Y(1)^2$ defined over a number
  field $K\subset\IC$. Suppose $V$ contains infinitely many special
  points.
  We may assume that $V$ is neither horizontal
  $Y(1)\times\{\text{point}\}$ nor vertical $\{\text{point}\}\times
  Y(1)$. 

  We keep the notation of Lemma~\ref{lem:apppwao}. For given
  $\Delta_i$ there are only finitely many possible $z_i$. So we may
  assume that $\max\{|\Delta_1|,|\Delta_2|\}$ is sufficiently large.
  Then some Galois conjugate of $(z_1,z_2)$ lies in
  $j(X^{\mathrm{alg}})$ where $X$ is as in (\ref{def:Xao}). In
  particular, $X^{\mathrm{alg}}\not=\emptyset$. 

  Now we need an Ax--Lindemann--Weierstrass theorem  for the
  $j$-function.
  In this low dimension ($m=2$) some extra simplifications are
  possible.
  After complexifying a real semi-algebraic curve in
  $X^{\mathrm{alg}}$ we find that, for dimesion reasons, 
  $j|_{\cF}^{-1}(V(\IC))$ is the intersection of an irreducible
  complex algebraic
  curve $C$ in $\IC^2$ with $\IH^2$. We must have
  $C(\IC) \subset j^{-1}(V(\IC))$. 

  We can proceed as in Section~\ref{sec:alwav} with some
  simplification made possible by $m=2$. We will only sketch the argument.

  One can also show that $C(\IC)$ passes through  ``many'' translates
  $\gamma\cF^2 = (\gamma_1\cF)\times (\gamma_2\cF)$ of the  fundamental domain
  $\cF^2$; here $\gamma=(\gamma_1,\gamma_2)\in \mathrm{SL}_2(\IZ)^2$.
  For example, if $\gamma_1\in \mathrm{SL}_2(\IZ)$ is arbitrary, then
  there exists $\gamma_2\in \mathrm{SL}_2(\IZ)$ such that 
  $C(\IC)\cap (\gamma_1,\gamma_2) \cF^2$ contains a neighborhood in
  $C(\IC)$. 
  
  The pre-image $j^{-1}(V(\IC))$ is invariant under the action of
  $\mathrm{SL}_2(\IZ)^2$ since $j$ is a modular function. 
  So $\gamma^{-1}C(\IC)\cap \cF^2 \subset j^{-1}(V(\IC))$. Both
  $\gamma^{-1}C(\IC)$ and $j^{-1}(V(\IC))$ are complex analytic
  subsets of $\IH^2$ of dimension $1$. So $\gamma^{-1}C(\IC) \subset
  j^{-1}(V(\IC))$.  Because $j^{-1}(V(\IC))\cap\cF^2$ has only finitely
  many connected component we find ``many'' $\gamma$ with
  $\gamma^{-1} C(\IC) = C(\IC)$. Many means that we can arrange the
  first component of $(\gamma_1,\gamma_2)$ to come from a finite index
  subgroup $\Gamma_1\subset\mathrm{SL}_2(\IZ)$.

  It follows that the image of
  $\Gamma_1$ lies in the projection of
  \begin{equation*}
    \mathrm{Stab}(C) = \{\gamma= (\gamma_1,\gamma_2) \in \mathrm{SL}_2(\IR) :
    \gamma (C(\IC)\cap\IH^2) =  C(\IC)\cap\IH^2 \}
  \end{equation*}
  to the first component. 
  By symmetry this works for the second component as well. So the
  projection of $\mathrm{Stab}(C)$ to the second component contains 
  the image of a finite index subgroup of $\mathrm{SL}_2(\IZ)$.
  
  % To complete the proof we recall a special case of Goursat's Lemma
  % from group theory.
  To complete the proof we need to lemmas from group theory.
  
  \begin{lemma}
    Let $H$ be a subgroup of $\mathrm{PSL}_2(\IR)^2$ that
    project surjectively to both factors. Suppose
    $\mathrm{PSL}_2(\IR)\times\{1\}\not\subset H$
    and $\{1\}\times \mathrm{PSL}_2(\IR)\not\subset H$. Then
    $H$ is the graph of an automorphism
    $\varphi\colon \mathrm{PSL}_2(\IR)\rightarrow \mathrm{PSL}_2(\IR)$. Moreover,
    there exists $\alpha\in \mathrm{GL}_2(\IR)$ with
    $\varphi(\gamma) = \alpha\gamma\alpha^{-1}$ for all
    $\gamma\in\mathrm{PSL}_2(\IR)$.
    % Let $H$ be a subgroup of $\mathrm{SL}_2(\IR)^2$ that
    % projects surjectively to both factors. Then $H$ is the graph of an
    % automorphism $G\rightarrow G$. 
    % Let $G$ be a group and let $H$ be a subgroup of $G^2$ that
    % projects surjectively under both projections $\pi_1$ and $\pi_2$.
    % The image of $H$ in $G/\ker({\pi_2}|_H) \times G/\ker({\pi_1}|_H)$
    % is the graph of an
    % isomorphism $G/\ker({\pi_2}|_H)\rightarrow G/\ker({\pi_1}|_H)$. 
  \end{lemma}
  \begin{proof}
    This is a consequence of Goursat's Lemma and the fact that
    $\mathrm{PSL}_2(\IR)$ is a simple group. Finally, any automorphism of
    $\mathrm{PSL}_2(\IR)$ is induced by conjugation by an element of
    $\mathrm{GL}_2(\IR)$ !!!CITE.
  \end{proof}

  We apply this lemma to the image of $\mathrm{Stab}(C)$ in
  $\mathrm{PSL}_2(\IR)^2$; indeed
  $\mathrm{PSL}_2(\IR)$ is
  a simple group.
  One can show that the image of $\mathrm{Stab}(C)$ surjects to both factors, for
  this we need that finite index subgroups of $\mathrm{SL}_2(\IZ)$ are
  Zariski dense in $\mathrm{SL}_2$.\footnote{This is a
    consequence of the fact that $\mathrm{SL}_2(\IZ)$ itself is
    Zariski dense in $\mathrm{SL}_2$.}
  Moreover, the image of $\mathrm{Stab}(C)$ does not contain $\{1\}\times
  \mathrm{PSL}_2(\IR)$ or  $\mathrm{PSL}_2(\IR) \times\{1\}$ because $C$
  is not a  vertical or horizonal curve. 

  It follows that the unit component $\mathrm{Stab}^\circ(C)$ of
  $\mathrm{Stab}(C)$
  contains
  $\{(\gamma,\alpha\gamma\alpha^{-1}) : \gamma \in\mathrm{SL}_2(\IR)
  \}$.

  Here is the second lemma which establishes that $\alpha$ has
  rational entries. 

  \begin{lemma}
    \label{lem:ratlmatrix}
    Let $\Gamma$ be a finite index subgroup of $\mathrm{SL}_2(\IZ)$
    and let $\alpha=\mattt{a}{b}{c}{d}\in \mathrm{GL}_2(\IC)$ with
    $\alpha\gamma\alpha^{-1}\in\mathrm{SL}_2(\IZ)$ for all
    $\gamma\in\Gamma$. Then $a-d,b,c\in\IQ$. 
  \end{lemma}
  \begin{proof}
    Write $n=[\mathrm{SL}_2(\IZ):\Gamma]!$. By the
    Pigeonhole Principle $A^n\in\Gamma$ and $B^n\in\Gamma$ for the  matrices
    \begin{equation*}
      A=\mattt{1}{1}{0}{1}
      \quad\text{and}\quad
      B=\mattt{1}{0}{1}{1}
    \end{equation*}
    lie in $\Gamma$. Observe
    $A^n = \mattt{1}{n}{0}{1}$ and $B^n=\mattt{1}{0}{n}{1}$.
    Now writing down the equalities 
    $\alpha A^n= A^n\alpha$ and $\alpha B^n=B^n\alpha$ leads to
    $nc,n(a-d)\in\IZ$ and $nb\in\IZ$, respectively, as desired. 
  \end{proof}

  We may scale $\alpha$ by a non-zero real number and therefore assume
  that the upper-left entry of $\alpha$ is $0$ or $1$. So $\alpha\in
  \mathrm{GL}_2(\IQ)$ by Lemma~\ref{lem:ratlmatrix}.
  After rescaling
  $\alpha$ again we may assume that it has integral and coprime
  coefficients.

  Let us assume for simplicity that $\det \alpha \ge 1$. 

  Fix an auxiliary point $(\tau_1,\tau_2)\in C(\IC)\cap \IH^2$. Its
  orbit under $\mathrm{Stab}(C)$ lies in $C$. So by our arguments
  above we have
  \begin{equation}
    \label{eq:stabilizerincurve}
    \{(\gamma\tau_1,\alpha\gamma\alpha^{-1}\tau_2) : \gamma \in \mathrm{SL}_2(\IR)\}
    \subset C(\IC). 
  \end{equation}

  As $\mathrm{SL}_2(\IR)$ acts transitively on $\IH$ we find
  $\eta_1,\eta_2$ in this group with $\tau_j = \eta_j \sqrt{-1}$ for
  both $j$. The stabilizer of $\sqrt{-1}$ is  $\mathrm{SO}_2(\IR)$ and
  the stabilizer of $\tau_1$ is $\eta_1\mathrm{SO}_2(\IR)\eta_1^{-1}$. So
  $\alpha \eta_1\mathrm{SO}_2(\IR) \eta_1^{-1} \alpha^{-1}$ must stabilize
  $\tau_2$. So it equals $\eta_2\mathrm{SO}_2(\IR)\eta_2^{-1}$.
  We conclude $\beta \mathrm{SO}_2(\IR)
  \beta^{-1} = \mathrm{SO}_2(\IR)$ with
  $\beta = \eta_2^{-1}\alpha\eta_1$. From this one can
  show that $\beta\sqrt{-1}=\sqrt{-1}$.

  %\eta_2^{-1}\alpha\eta_1=\beta\in \mathrm{SO}_2(\IR)$.
  
  In (\ref{eq:stabilizerincurve})  we can rewrite
  \begin{equation*}
    (\gamma \tau_1, \alpha \gamma(\eta_1
    \beta^{-1}\eta_2^{-1})\tau_2) =
    (\gamma\tau_1,\alpha\gamma \eta_1\beta^{-1}\sqrt{-1})
    =
        (\gamma\tau_1,\alpha\gamma \eta_1\sqrt{-1})=
    (\gamma\tau_1, \alpha \gamma\tau_1).
  \end{equation*}
  Thus
  $$    \{j(\gamma\tau_1,\alpha\gamma\tau_1) : \gamma \in \mathrm{SL}_2(\IR)\}
  \subset V(\IC). $$
  
  The elliptic curves $\IC/(\IZ+\gamma\tau_1\IZ)$ and
  $\IC/(\IZ+\alpha\gamma\tau_2\IZ)$ are related by a cyclic isogeny of degree
  $N=\det\alpha\in\IN$ for all $\gamma$. It follows that $V$ and meets
  the zero locus of $\Phi_N$ in infinitely many points. Therefore $V$
  equals the zero locus of $\Phi_N$.   So it follows that $V$ is special.
  % From this one can eventually conclude that $C(\IC)=\{\text{point}\} \times\IC$
  % or $C(\IC)=\IC\times\{\text{point}\}$ or the two coordinates on $C$
  % are related by a fractional linear transformation in
  % $\mathrm{GL}_2^+(\IQ)$. As $j(C)=V$ we see that $V$ must be special.  
\end{proof}
\section{Further Reading and Open Problems}

In the context of Masser's Theorem, is admissible in
Theorem~\ref{thm:masser}, it is an interesting problem to determine
the supremum of all possible exponents $\delta>0$ for which there
exists $c(A,\delta)>0$ with $[K(P):K]\ge c N^{\delta}$. Masser, in an
unpublished letter to Daniel Bertrand, showed that the supremum is at
least $1/\dim A$ if $A\not=0$. 

Here is an open problem on heights, sometimes called Lehmer's
Conjecture.\foonote{Although I know no evidence that Lehmer actually
  made this conjecture.}

\begin{conjecture}
  There exists a constant $c>1$ such that for all $x\in\IQbar$ we have
  $H(x)=1$ or $H(x) \ge c^{[\IQ(x):\IQ]}$.
\end{conjecture}

The best known result in this direct, up-to numerical constants, is
Dobrowolski's Theorem~\cite{Dobrowolsi:79}.
The related Schinzel--Zassenhaus Conjecture was recently proved by
Dimitrov.

\begin{theorem}[Dimitrov~\cite{Dimitrov:SZ}]
  Let $x$ be a non-zero algebraic integer that is not a root of unity,
  then the $\IZ$-minimal polynomial of $x$ has a complex root $z$ with
  $|z|\ge 2^{1/(4[\IQ(x):\IQ])}$. 
\end{theorem}

%%% Local Variables:
%%% TeX-master: "main"
%%% End:
