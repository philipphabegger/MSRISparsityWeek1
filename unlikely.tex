

\chapter{Unlikely Intersections: Relative Manin--Mumford}

\section{Overview}

Let $Z$ be some ``arithmetically interesting'' variety, such as an
abelian variety, or $\IG_{\mathrm{m}}^m$, or $Y(1)^m$ etc. Then $Z$
often has a ``special points'' $Z_{\mathrm{special}}$ (e.g. torsion
points) and ``special subvarieties'' (torsion cosets).

The basic template for a result of Andr\'e--Oort and Manin--Mumford
type goes as follows:

Let $V$ be an irreducible subvariety of
$Z$. Then
\begin{equation*}
  V(\IC) \cap Z_{\mathrm{special}}\text{ is Zariski dense in
    $V$}\Leftrightarrow \text{$V$ is special}.
\end{equation*}

In the 2000s, Zilber~\cite{Zilber}, Pink~\cite{Pink}, and
Bombieri--Masser--Zannier~\cite{BMZgeometric} asked: what if instead
of intersecting $V$ with special points, we intersect it with special
subvarieties? In other words, we replace $Z_{\mathrm{special}}$ by all
special subvarieties and investigate
\begin{equation*}
  \bigcup_{\substack{S\subset Z \\ \text{$S$ special}}} V\cap S.
\end{equation*}

In all situations the ambient variety $Z$ itself was special, so we must restrict $S$ in
the union to obtain a meaning problem.

Perhaps the most natural restriction to make is on the dimension of
$S$. If $Z$ is a regular variety, then every irreducible component in
$V\cap S$ has dimension at least $\dim V+\dim S - \dim Z$. Recall that
$V\cap S$ is called a \emph{proper intersection} if every irreducible
component of $V\cap S$ has dimension equal to $\dim V+\dim S- \dim Z$.

Now if $\dim V + \dim S -\dim Z < 0$, then we would not $V$ and $S$ to
meet if they are in general position. However, it remains possible
that the intersection is non-empty but in this case we call it
\emph{unlikely}.

We have studied the following cases
\begin{center}
  \begin{tabular}{l|l|l}
    Ambient variety & special points & special subvarieties \\
    \hline
    $\IG_{\mathrm{m}}^m$ & $\mu_\infty^m$ (torsion points) & torsion cosets \\
    Abelian varieties & $A_{\mathrm{tors}}$ (torsion points) & torsion cosets \\
    $Y(1)^m$ & $Y(1)^m_{\mathrm{special}}$ (CM points) & special
                                                         subvarieties
    \\
    \vdots & \vdots & \vdots
  \end{tabular}
\end{center}

We will see more examples below coming from a family of elliptic
curves. In the  general setting studied by Pink~\cite{Pink}
$Z$ is a mixed Shimura variety.

The Conjectures on unlikely intersections make this heuristic precise.

\begin{conjecture}[Zilber--Pink Conjecture or Conjecture on Unlikely
  Intersections]
  \label{conj:ZP}
  Let $Z$ be onle of the ambient varieties listed above and suppose
  $V$ is an irreducible subvariety of $Z$. Suppose that $V$ is not
  contained in a proper special subvariety of $Z$.
  Then
  \begin{equation*}
    \bigcup_{\substack{S\subset Z \text{ special} \\ \dim S < \dim
        Z-\dim V}} V\cap S
  \end{equation*}
  is not Zariski dense in $V$. 
\end{conjecture}

\begin{exercise}
  Show that the conjecture above implies the Manin--Mumford Conjecture
  if $Z = \IG_{\mathrm{m}}$ or if $Z$ is an abelian variety. 
\end{exercise}

\begin{example}
  Conjecture~\ref{conj:ZP} is open in general, but some special cases
  are known. 
  \begin{itemize}
  \item If $\dim V = \dim Z-1$, then the condition $\dim S < \dim Z -
    \dim V$ means $\dim S =0$. So $S$ is a ``special point''. In this
    case the conjecture reduces to the Manin--Mumford Conjecture if
    $Z$ is an abelian variety or $\IG_{\mathrm{m}}^m$ and to
    Andr\'e--Oort if $Z=Y(1)^m$.
  \item If $\dim V = \dim Z-2$, then we refer  work of
    Bombieri--Masser--Zannier~\cite{BMZgeometric,BMZUnlikely} for $Z=\IG_{\mathrm{m}}^m$
    and
    Barroero--Dill~\cite{BarroeroDill:ENS} for $Z$ an abelian variety.
  \item Not much is known in the ``intermediate dimensions''.
  \item If $\dim V = 1$, so $V$ is a curve, we know more.
    If $Z=\IG_{\mathrm{m}}^m$, then Conjecture~\ref{conj:ZP} follows
    from work of Maurin~\cite{Maurin} (over base field $\IQbar)$ and
    Bombieri--Masser--Zannier~\cite{BMZUnlikely} (over $\IC$).
    If $Z=A$ is an abelian variety, see work of Pila and the author
    \cite{HabeggerPilaENS} (over the base field $\IQbar$) and
    Barroero--Dill~\cite{BarroeroDill:ENS}
    (over $\IC$).
    Earlier work in the abelian setting
    was done by
    Carrizosa, Galateau, Ratazzi, R\'emond, R\'emond--Viada,  and Viada.
    If $Z=Y(1)^m$ there are some partial results by Pila
    and the author~\cite{HabeggerPila12}. 
  \end{itemize}
\end{example}

\section{An Example in $\IG_{\mathrm{m}}^3$}

In this section we shift gears and consider
an explicit curve $C$ in  the ambient variety $\IG_{\mathrm{m}}^3$.
Unlikely intersections asks:  is $$\bigcup_{\dim H \le 1} C\cap
H$$ finite where $H$ ranges over algebraic subgroups of
$\IG_{\mathrm{m}}^3$?

If $\dim H=0$, we recover torsion points and Manin--Mumford.

If $\dim H=1$, the unit component of $H$ is
\begin{equation*}
  \{(t^{a},t^{b},t^{c}) : t\in \IC^\times \}. 
\end{equation*}
Dually we see that  $H$ is described by two equations
\begin{equation*}
  x^{\alpha_1} y^{\alpha_2} z^{\alpha_3} =
  x^{\beta_1} y^{\beta_2} z^{\beta_3} = 1.
\end{equation*}
for    \emph{linearly independent}
$(\alpha_1,\alpha_2,\alpha_3), (\beta_1,\beta_2,\beta_3)\in\IZ^3$.


\begin{example}\label{ex:GM3curve}
  \begin{enumerate}
  \item [(i)]
    Say $C$ is the line $\left\{(x,1+x,2+x) : x\in \IC\ssm\{0,-1,-2\}\right\}$.
    
    Find solutions  $x\in\IC\ssm\{0,-1,-2\}$ of 
    \begin{equation*}
      x^{\alpha_1} (1+x)^{\alpha_2} (2+x)^{\alpha_3} =
      x^{\beta_1} (1+x)^{\beta_2} (2+x)^{\beta_3} = 1.
    \end{equation*}
    with \emph{varying} and \emph{independent}  vectors in the
    exponents.

    An early result of Bombieri--Masser--Zannier (1999) implies
    that there are only \emph{finitely} many solutions $x$.

  \item[(ii)] Say $C$ is $\{(x,1+x,1) : x\in \IC\ssm \{0,-1\}\}$.
    There are \emph{infinitely} many solutions:  let
    $\zeta\in\mu_\infty$ have order $N\ge 3$, then 
    \begin{equation*}
      \zeta^{0} (1+\zeta)^{0} 1^{1} = \zeta^{N} (1+\zeta)^{0} 1^0 = 0
    \end{equation*}
    with linearly independent exponent vectors $(0,0,1),(N,0,0)$.

    What goes wrong? $C$ is contained in the proper algebraic subgroup
    $\IG_{\mathrm{m}}^2\times \{1\}$ of $\IG_{\mathrm{m}}^3$. 
  \end{enumerate}
\end{example}  

One fundamental intermediate step towards proving unlikely
intersection is bounded the height. We do this for our explicit curve
from Example~\ref{ex:GM3curve}(i). 

\begin{theorem}
  \label{thm:hgtbd}
  Let 
  $x\in\IC\ssm\{0,-1,-2\}$ with
  \begin{equation}
    \label{eq:2mult}
    x^{\alpha_1} (1+x)^{\alpha_2} (2+x)^{\alpha_3} =
    x^{\beta_1} (1+x)^{\beta_2} (2+x)^{\beta_3} = 1.
  \end{equation}
  where
  $(\alpha_1,\alpha_2,\alpha_3),(\beta_1,\beta_2,\beta_3)\in\IZ^3$
  are linearly independent. Then $x\in\IQbar$ and $H(x)\le 4$. 
\end{theorem}
\begin{proof}
  Here it is convenient to work with the logarithmic height $h(\cdot) =
  \log H(\cdot)$ of an algebraic number.

  We translate the property listed in Remark~\ref{rmk:heightprops}(iv)
  to the logarithmic setting.
  For all $x,y\in\IQbar$ we have
  \begin{equation}
    \label{eq:hgtprop1}
    h(x+y)\le \log 2 + h(x) + h(y).
  \end{equation}
  Moreover, 
  \begin{equation}
    \label{eq:hgtprop2}
    h(x^k) = |k|
    h(x) \text{ for all $k\in\IZ$, where $x\not=0$ if $k\le 0$};
  \end{equation}
  this last property follows from Exercises~\ref{exer:heightprop}(i)
  and (ii).

  The following argument is due to Bombieri.

  Let $x$ and the $(\alpha_i)_i,(\beta_i)_i\in\IZ^3$ be as in the
  hypothesis. After rearranging
  $x^{\alpha_1}(1+x)^{\alpha_2}(2+x)^{\alpha_3}=1$ we obtain a
  polynomial in integral coefficients with $x$ as a root, so $x$ is
  algebraic. 

  
  Next, we eliminate the third coordinate. So fix $\lambda,\mu\in\IZ$
  not both zero with
  $\lambda(\alpha_1,\alpha_2,\alpha_3)+\mu(\beta_1,\beta_2,\beta_3)=(r,s,0)$.
  Then $(r,s)\in\IZ^2\ssm\{0\}$ and
  $$
  x^r (1+x)^s = 1.$$

  We rearrange, to get $x^r = (1+x)^{-s}$ and insert into the height,
  \textit{i.e.},
  $h(x^r) = h((1+x)^{-s})$. We use (\ref{eq:hgtprop2}) to find
  \begin{equation*}
    |r| h(x) = |s| h(1+x). 
  \end{equation*}

  If $s=0$, then $r\not=0$ and $h(x)=0$ and we are done. So assume
  $s\not=0$. Thus $h(1+x) = |q| h(x)$ with $q=r/s$. 
  
  By (\ref{eq:hgtprop1}) we find $h(x+1) \le \log 2 + h(x) + h(1) =
  \log 2 +h(x)$ and similarly $h(x+1)\ge h(x)-\log 2$. So
  $|h(x)-h(x+1)|\le \log 2$. We insert $h(1+x) = |q|h(x)$ to conclude
  $$
  \bigl |1 - |q|\bigr |h(x) \le \log 2.$$

  If $|q| \ge 3/2$, then $|1-|q|| = |q|-1\ge 1/2$ and therefore $h(x)\le
  2\log 2$.

  If $|q|\le 1/2$, then $|1-|q|| = 1-|q|\ge 1/2$ and therefore again
  $h(x)\le 2\log 2$.

  Finally, assume $1/2 < |q| < 3/2$. We have $x^r(1+x)^s=1$ and divide
  by $x^{r+s}$, so $(1+x^{-1})^s = x^{-(r+s)}$. We do a change of
  coordinates $t=x^{-1}$ and
  find $$1=t^{-(r+s)}(1+t)^s=t^{r'}(1+t)^{s'}$$
  wir $r'=-(r+s)$ and $s'=s$. Note $q'= r'/s'=-1-q$.
  So $|q'|< 1/2$ or $|q'|>3/2$ and the argument above yields $h(t)\le
  \log 4$.
  But $h(t)=h(x^{-1})=h(x)$ and we are done. 
\end{proof}

This is not yet enough to deduce finiteness of $x$.
But as the height is bounded, then the degree of
$[\IQ(x):\IQ]$ must be large by Northcott's Theorem if we have many
solutions to (\ref{eq:2mult}).

\begin{corollary}
  Let $d\ge 1$, there are at most finitely many
  $x\in\IQbar\ssm\{0,-1,-2\}$ with (\ref{eq:2mult}) with
  $[\IQ(x):\IQ]\le d$. 
\end{corollary}

An important tool in order to bound $[\IQ(\alpha):\IQ]$ from below is
the following result towards Lehmer's Conjecture.

\begin{theorem}[Dobrowolski]
  Let $\epsilon > 0$, there exist $c(\epsilon)>0$ with the following
  property.
  If $x\in\IQbar$ satisfies $H(x) > 1$, then
  \begin{equation*}
    \log H(x) \ge \frac{c(\epsilon)}{[\IQ(x):\IQ]^{1+\epsilon}}. 
  \end{equation*}
\end{theorem}



We sketch that, under a simplifying assumption, we have finiteness of the set of
all $x$ from Theorem~\ref{thm:hgtbd}. 

Let $x$ be as in (\ref{eq:2mult}). We already know that $H(x)\le 4$;
this will be an important tool. 

Then $x,1+x,2+x$ generate a
subgroup of $\IQbar^\times$ of rank $1$. So there exists $t\in
\IQbar^\times, a,b,c\in\IZ,$ and
$(\zeta_1,\zeta_2,\zeta_3)\in\mu_\infty^3$  with 
$$
(x,1+x,2+x) = (\zeta_1 t^a, \zeta_2 t^b,\zeta_3 t^c).
$$

For \emph{simplicity}\footnote{This assumption is not so harmless.
  Doing the general case requires a more sophisticated version of
  Dobrowolski's Theorem.} we assume that
$\zeta_1=\zeta_2=\zeta_3=1$, so
$$
(x,1+x,2+x) = (t^a, t^b,t^c).
$$
We also assume for simplicity that $|a|\ge |b|$ and $|a|\ge
|c|$.\footnote{This is a harmless assumption.}

By Minkowski's Lattice Point Theorem we can find
$(\alpha,\beta,\gamma)\in\IZ^3\ssm\{0\}$ of norm $\ll
|a|^{1/2}$
with $a\alpha+b\beta+c\gamma=0$. This implies
\begin{equation}
  \label{eq:simplifiedmult}
  x^\alpha(1+x)^\beta(2+x)^\gamma =1.  
\end{equation}

From $x=t^a$ and the height properties\footnote{Including
  submultiplicativity in Remark~\ref{rmk:heightprops}(iv).}  we get
$\log H(x) \ge \log H(t^a) = |a| \log H(t)$. But $H(x)\le  4$ by
Theorem~\ref{thm:hgtbd}. So
\begin{equation*}
  \log H(t)\le \frac{\log 4}{|a|}.
\end{equation*}

If $H(t)=1$, then $t$ is a root of unity by Kronecker's Theorem.
But this is ruled out by our simplfying assumption that $x,1+x,2+x$
generate a rank $1$ subgroup. 
Dobrowolski's Theorem, with $\epsilon=1/3$,  yields $[\IQ(t):\IQ] \gg
|a|^{3/4}$.

We can rewrite (\ref{eq:simplifiedmult}) as a polynomial equation
in $x$ of degree $\ll |(\alpha,\beta,\gamma)|_\infty\ll |a|^{1/2}$. 
So $[\IQ(x):\IQ] \ll |a|^{1/2}$.
But as $t$ is in the group generated by $x,1+x,2+x$ we have
$\IQ(t)\subset\IQ(x)$ and actually equality holds since $x=t^a$. Therefore,
\begin{equation*}
|a|^{3/4}\ll  [\IQ(t):\IQ] = [\IQ(x):\IQ]  \ll |a|^{1/2}.
\end{equation*}
This means that $|a|$ is uniformly bounded from above (independent of
$x$). The same holds for $|b|$ and $|c|$. But then
$(\alpha,\beta,\gamma)$ is in a finite set (it has norm
$\ll|a|^{1/2}$).
So by (\ref{eq:simplifiedmult}) $x$ is in a finite set.



\section{Relative Manin--Mumford}


One striking feature of abelian varieties is that they deform
algebraically in families. This is already apparent in the case of
elliptic curves.

\begin{example}
  \label{ex:relmm}
  Let $\lambda\in \IC\ssm\{0,1\}$. The (long) Weierstrass equation
  \begin{equation*}
    y^2 = x(x-1)(x-\lambda)
  \end{equation*}
  determines an elliptic curve. We may consider
  $Y(2) = \IP^1\ssm\{0,1,\infty\}$ as the base of a family of elliptic
  curves. This family is called the \emph{Legendre family of elliptic
  curves}. More precisely, it is given by
  \begin{equation*}
    \cE=\{([x:y:z], [\lambda:1]) \in \IP^2 \times B : y^2z =
    x(x-z)(x-\lambda z)\}
  \end{equation*}
  and the structure morphism $\cE\rightarrow Y(2)$ induced by
  projecting.\footnote{The base curve $Y(2)$ is the modular curve of full level-$2$
  structure.} 
  We denote the fiber above $\lambda\in Y(2)(\IC)$  by $\cE_\lambda
  \subset\IP^2$.
  Of course, $\cE$ is a quasi-projective surface defined over $\IQ$.
  
  We define
  \begin{equation*}
    \cE_{\mathrm{tors}} = \bigcup_{\lambda\in Y(2)(\IC)}
    \cE_{\lambda,\mathrm{tors}}. 
  \end{equation*}

  Observe that $\cE_{\mathrm{tors}}$ is not a group in any sensible
  way. It is not possible to add two points in different fibers of
  $\cE\rightarrow Y(2)$.
  
  Suppose we are given a curve $C\subset \mathcal{E}$. Does a
  Manin--Mumford type statement hold for $C\cap
  \mathcal{E}_{\mathrm{tors}}$? In other words, is this intersection
  finite ``most of the time''?

  The answer is \textbf{no}! But why?

  Here are some very simple  heuristics. For $N\in\IZ$,  multiplication-by-$N$ is an
  morphism
  $[N]\colon \cE\rightarrow\cE$ over $Y(2)$.
  Let us denote $\ker [N] = \{P \in \cE(\IC): [N](P)=0\}$. 
  We can also write
  $\cE_{\mathrm{tors}}$ as a infinite union
  $\bigcup_{N\in\IN}\ker[N]$.
  For $N\in\IN$ the kernel  $\ker[N]$ is a (possibly reducible) curve
  in $\cE$. For example,
  \begin{equation*}
    \ker [1] = \{[0:1:0]\}\times Y(2)
  \end{equation*}
  and
  \begin{equation*}
    \ker[2] = \bigl(\{[0:0:1],[1:0:1],[0:1:0]\}\times Y(2)\bigr)\cup
    \{([\lambda:0:1],\lambda) : \lambda \in Y(2)(\IC)\}. 
  \end{equation*}

  Heuristically we would expect that the two curves $C$ and $\ker[N]$
  inside the ambient space $\cE$ have non-empty intersection. As we
  vary over infinitely many $N$ we would heuristically expect
  infinitely many points in $\bigcup_{N\in\IN}C(\IC)\cap \ker[N]$.

  Of course, there are two difficulties making this  argument precise.
  First, we would have to verify that $\ker[N]\cap C$ is non-empty, at
  least for infinitely many $N$. Second, we would need to rule out
  excessive double counting when taking the union over all $N$. % Below,
  % we
  % will give a different argument that shows $C(\IC)\cap\cE_{\mathrm{tors}}$
  % is infinite.

  The heuristic argument becomes invalid if we consider a curve in the
  fibered power $\cE\times_{Y(2)} \cE =\cE^2$ instead. Indeed, here $\ker[N]$
  are still one-dimensional. But the ambient variety $\cE^2$ is now
  three-dimensional. So it is unlikely that a curve $C$ and $\ker[N]$
  should intersect.
\end{example}

\begin{definition}  Let $m\in\IN$.
  \begin{enumerate}
  \item [(i)] The $m$-fold fibered power of $\cE\rightarrow Y(2)$
    is denoted by $\cE^m = \cE\times_{Y(2)}\cdots\times_{Y(2)}\cE$.
    Let $\pi\colon\cE^m\rightarrow Y(2)$ be the natural projection.
  \item[(ii)] For $a=(a_1,\ldots,a_m)\in\IZ^m$ we have a morphism
    $\varphi_a \colon \cE^m\rightarrow \cE$ over $Y(2)$
    determined  by $P\mapsto
    [a_1](P_1)+\cdots +[a_m](P_m)$. We also defined $H_a =
    \{(P_1,\ldots,P_m) \cE^m(\IC) : \varphi_a(P_1,\ldots,P_m)=0\}$. 
  \item[(iii)]  We write $\cE^m_{\mathrm{tors}} = \bigcup_{\lambda\in Y(2)(\IC)}
    \cE^m_{\mathrm{tors}}$. 
  \end{enumerate}
\end{definition}

Then $\cE^m$ is a family of abelian varieties. Each fiber
$\cE^m_\lambda$ is the $m$-th power of an elliptic curve.

The following Conjecture was stated by Pink~\cite{Pink} and later by
Masser--Zannier~\cite{MZ:AJM10}, but in the more general setting of
abelian schemes.

\begin{conjecture}[Relative Manin--Mumford in $\cE^m$]
  \label{conj:relmmEm}
  Let $V$ be an irreducible subvariety of $\cE^m$ with $\pi(V)=Y(2)$.
  Suppose that  
  \begin{equation*}
    C\not\subset H_a
  \end{equation*}
  for all $a\in\IZ^{m}\ssm\{0\}$.
  Then $V\cap \cE^m_{\mathrm{tors}}$ is not Zariski dense.
\end{conjecture}



This conjecture is part of a much more general conjecture posed by
Pink~\cite{Pink} and studied in special cases by
Bombieri--Masser--Zannier~\cite{BMZ,BMZgeometric} and
Zilber~\cite{Zilber}. We refer to Zannier's volume~\cite{ZannierBook}
for more background. 

Conjecture~\ref{conj:relmmEm} was proved by Masser and
Zannier~\cite{MZ:AJM10} for $\dim V = 1$. We explicitly mention the
case $m=2$ and $\dim V=1$ here.

\begin{theorem}
  \label{thm:relmm2}
  Let $V$ be an irreducible curve in $\cE^2$ defined over $\IQ$ with
  $\pi(V)=Y(2)$. Suppose that $V\not\subset H_{(a,b)}$ for all
  $(a,b)\in\IZ^2\ssm\{0\}$. Then $V\cap \cE^2_{\mathrm{tors}}$ is
  finite.
\end{theorem}

\begin{remark}
  \begin{enumerate}
  \item [(i)]In the theorem and the conjecture we imposed the condition
    $\pi(V)=Y(2)$ for the following reason. The morphism $\pi$ is proper.
    So we  have either $\pi(V)=Y(2)$ or $\pi(V)$ is a point. In the second
    case, $V$ is a subvariety of the constant abelian variety above the
    singleton $\pi(V)$. The intersection of $V$ with the torsion happens
    in a single abelian variety and we  are reduced to the (known!)
    Manin--Mumford  Conjecture. 
  \item[(ii)] Why do we impose $V\not\subset H_a$ resp. $C\not\subset
    H_{(a,b)}$? In the case of curves suppose  $C\subset H_{(a,b)}$ with
    $(a,b)\not=0$.
    Then $C$ is contained, up-to torsion, in
    a family of abelian subvarieties of $\cE^2$. So the additional
    coordinate that was introduced in rule out the heuristic in the
    end of example~\ref{ex:relmm} was a red herring. And indeed, in this case
    an updated heuristic  suggests that  $C\cap
    \cE^2_{\mathrm{tors}}$ should be infinite. 
  \end{enumerate} 
\end{remark}


\begin{exercise} 
  Formulate a conjecture akin to Conjecture \ref{conj:relmmEm} including
  a necessary and  sufficient condition for when $V(\IC)\cap
  \mathcal{E}^m_{\mathrm{tors}}$ is Zariski dense in $V$. Hint: look at
  the Manin--Mumford Conjecture. 
\end{exercise}

\section{A Uniformization of the Legendre Family}
\label{sec:unilegendre}
Our goal is find a suitable uniformization of $\cE(\IC)$ and then
to adapt the Pila--Zannier strategy to prove Theorem~\ref{thm:relmm2}.

The analog for $Y(2)$ of Klein's modular $j$-function is the modular
$\lambda$-function. This is a holomorphic map $\lambda\colon
\IH\rightarrow\IC\ssm\{0,1\}=Y(2)(\IC)$ that satisfies
\begin{equation*}
  \lambda\left(\frac{a\tau+b}{c\tau +d}\right) = \lambda(\tau)
\end{equation*}
for all $\tau\in\IN$ and all
\begin{equation*}
  \mattt{a}{b}{c}{d}  \in \Gamma(2) = \{\gamma \in
  \mathrm{SL}_{2}(\IZ) : \gamma\equiv I \mod 2\}.
\end{equation*}

In particular, $\lambda(\tau+2)=\lambda(\tau)$ and so there is a
development in $q = e^{\pi i \tau}$. The first few terms are
\begin{equation*}
  \lambda(\tau) = 16q - 128q^2 + 704q^3 +\cdots. 
\end{equation*}
The function is connected to Klein's $j$-function via
\begin{equation*}
  j(\tau) = 2^8
  \frac{(\lambda(\tau)^2-\lambda(\tau)+1)^3}{\lambda(\tau)^2(\lambda(\tau)-1)^2}. 
\end{equation*}
And $\lambda(\tau)$ is the $j$-invariant of the elliptic curve
determined by $y^2 = x(x-1)(x-\lambda(\tau))$.

Let $\cF_{\mathrm{pure}}$ be a fundamental domain for the action of $\Gamma(2)$ on
$\IH$. As in the case of the action of the full modular group, we can
take $\cF_{\mathrm{pure}}$ to be real semi-algebraic.

For definiability of $\lambda$ restricted to $\cF_{\mathrm{pure}}$  in
$\IRanexp$ we refer to work of Peterzil--Starchenko~\cite{PS:Duke13}.
However, for it turns out that we can work on some compact box in
$\IH$ and ultimately $\IRan$ can do the job. 

The function $\lambda$ uniformizes the base $Y(2)$. 
Using theta functions we can uniformize the fibers. In other words,
and after doubling coordinates,
there is a holomorphic map $u$ with
\begin{equation*}
  \begin{tikzcd}[column sep=small, row sep=small] 
    & \IC^2\times\IH  \arrow{dd}{} \arrow{r}{u} & \cE^2(\IC)\arrow{dd}{}\\
    &  &\\
    &    \IH \arrow{r}{\lambda}  & Y(2)(\IC).
  \end{tikzcd}
\end{equation*}

Each fiber $\cE_{\lambda(\tau)}$ of $\cE\rightarrow Y(2)$ is
uniformized by $\IC\rightarrow \cE_{\lambda(\tau)}(\IC)$ with kernel
$\IZ+\tau\IZ$.
We also obtain a relative fundamental domain
\begin{equation*}
  \cF_{\mathrm{mixed}} = \{(z_1,z_2,\tau) \in
  \IC^2\times \cF_{\mathrm{pure}} \text{ with }z_i=a_i+b_i\tau\text{
    where }a_1,b_1,a_2,b_2\in [0,1]  \}. 
\end{equation*}
So $\cF_{\mathrm{mixed}}$
is a real semi-algebraic set in $\IR^4\times \IH$ where, as
usual, we identify $\IC\times \IH$ with $\IR^4\times\IR\times(0,\infty)$.

In our original uniformization we do a real change of coordinates as
follows. 
For $(a_1,b_1,a_2,b_2,\tau)\in [0,1]^4\times \cF_{\mathrm{mixed}}$ we define
\begin{equation*}
   u_{\mathrm{Betti}}(a_1,b_1,a_2,b_2,\tau) = u(a_1+b_1\tau,a_2+b_2\tau,\tau)
\end{equation*}
and get a map $u_{\mathrm{Betti}}\colon \IR^4\times \cF_{\mathrm{pure}}\rightarrow\cE(\IC^2)$
and a fundamental domain 
\begin{equation*}
  \cF = [0,1]^4\times \cF_{\mathrm{pure}}. 
\end{equation*}

Let $V$ be an irreducible curve in $\cE^2$. Then
\begin{equation*}
  X = u_{\mathrm{Betti}}|_{\cF}^{-1}(V(\IC)) \subset \cF
\end{equation*}
is definable in $\IRanexp$.


Suppose $P\in V(\IC)\cap\cE_{\mathrm{tors}}^2$ and let
$z=(a_1,b_1,a_2,b_2,\tau)\in \cF$ with $u(z)=P$.
Now it will become clear why we did the real change of coordinates:
Since $P$ is torsion in $\cE^2_{\lambda(\tau)}$, any preimage under
the
uniformization $\IC\rightarrow \IC/(\IZ+\tau\IZ)$ lies in
$\IQ+\tau\IQ$. This means that $(a_1,b_1,a_2,b_2)\in \IQ^4$. Moreover,
$H(a_1,b_1,a_2,b_2)\le N$ where $N$ is the order of $P$. 

Suppose $V$ is defined over $\IQ$. 
We would like to apply the Pila--Wilkie Theorem. However, we must deal
with two issues, one old, with a twist, and a new one.

\begin{enumerate}
\item [(Old)] We need a lower bound for $[\IQ(P):\IQ]$ that grows like
  some small but fixed power of $N$. The added difficulty here when
  compare to the absolute Manin--Mumford Conjecture is that the
  elliptic curve $\cE_{\lambda(\tau)}$ is varying. So we cannot allude
  to Masser's Theorem~\ref{thm:masser} for a fixed abelian variety.
  We will handle this issue in
  Section~\ref{sec:lbdegrel}. 
  
\item[(New)] The point $z\in \IR^4\times\IH$ produced form $P$ is
  most likely \emph{not} a rational point. Indeed, only the first four
  coordinates $(a_1,b_1,a_2,b_2)$ are rational, but there is no
  reason to believe that $\tau$ is algebraic.\footnote{Indeed,
    expect if $\cE_{\lambda(\tau)}$ has complex multiplication, a
    theorem of Schneider implies that $\tau$ is transcendental.}
  So  $z$ is merely \textit{semi-rational}. We will present a
  semi-rational version of the Pila--Wilkie Theorem in
  Section~\ref{sec:semirational}. 
\end{enumerate}

\section{A Lower Bound for the Degree}
\label{sec:lbdegrel}

Next we need to make Masser's result, Theorem~\ref{thm:masser},
quantitative in terms of the elliptic curve.

\begin{theorem}[David]
  There exists $c>0$ with the following property. 
  Let $E=\cE_\lambda$ with $\lambda\in\IQbar\ssm\{0,1\}$. If $P\in
  E(K)$ has finite order $N$ where $K$ is a number field, then
  \begin{equation}
    \label{eq:davidKQlb}
    [K:\IQ] \ge c \frac{N^{1/2}}{\max\{1,\log H(\lambda)\}}.
  \end{equation}    
\end{theorem}
\begin{proof}
  This follows from Th\'eor\`eme 1.2~\cite{DavidPetiteHauteur}.
  Indeed, one can bound
  the height of the $j$-invariant of $E$ from above polynomially in
  terms of $H$. 
\end{proof}

We need the following consequence of a theorem of Silverman. Roughly,
it tells us that we can safely ignore the height that appears in the
denominator of (\ref{eq:davidKQlw}). Recall that $\pi\colon
\cE^2\rightarrow Y(2)$ is a family of squares of Legendre elliptic
curves.

\begin{theorem}
  Let $V\subset \cE^2$ be a curve defined over $\IQbar$. There exists 
  $c(V)>0$ such that if $P\in V\cap\mathcal{E}^2_{\mathrm{tors}}$  has
  order $>c(V)$, then $\pi(P)\in \IQbar\ssm\{0,1\}$ and $H(\pi(P))\le c(V)$. 
\end{theorem}
\begin{proof}
  The intersection $C\cap \ker[N]$ has dimension $0$ except if $C$ is
  an irreducible component of $\ker[N]$. In this case, all points on
  $C$ have some fixed order. So we may assume $C\cap \ker[N]$ finite
  for all $N\in\IN$. In particular, any point in this intersection is
  algebraic. We can then apply Silverman's Theorem \cite{Silverman}. 
\end{proof}

Observe that Silverman's Theorem only tells us that the height of
$\pi(P)$ is bounded from above. This does not imply that the set of
$\pi(P)$ is finite. The point is that  $\pi(P)$ could have
\textit{degree} over $\IQ$,
just as in the sequence $2^{1/d}$ whose members have height
$H(2^{1/d})=2^{1/d}\le 2$. 


\begin{corollary}
  \label{cor:davidsilverman}
  Let $V\subset \cE^2$ be a curve defined over $\IQ$. There exists 
  $c(V)>0$ such that if $P\in V(\IC)\cap\mathcal{E}^2_{\mathrm{tors}}$  has
  order $N>c(V)$, then $P$ is algebraic and 
  $[\IQ(P):\IQ] \ge c(V) N^{1/2}$. 
\end{corollary}

\begin{exercise}
  Let $B>0$ and  let $\epsilon > 0$. Show that there exists
  $\delta =\delta(B,\epsilon)\in [0,1)$ with the following property. If
  $\alpha\in\IC\ssm\{0,1\}$ is algebraic with Galois conjugates
  $\alpha_1,\ldots,\alpha_d\in\IC$ (so $d=[\IQ(\alpha):\IQ]$) and
  $H(\alpha)\le B$, then 
  \begin{equation*}
    \#\{ i : |\alpha_i|> \epsilon \text{ and
    }|\alpha_i-1|>\epsilon\text{ and }|\alpha_i|<\epsilon^{-1}\} \le
    \delta d.
  \end{equation*}
  So a positive proportion of all Galois conjugates of $\alpha$ are
  not too close to one of the three cusps $0,1,\infty$ of $Y(2)$. This
  also us to work in the o-minimal structure $\IRan$ and use simpler
  definability states when uniformizing $Y(2)$ and ultimately $\cE$. 
\end{exercise}

\section{Semi-rational Pila--Wilkie}
\label{sec:semirational}

Our approach to the Relative Manin--Mumford Conjecture for a curve in
$\cE$ has led to a set $X\subset \IR^m\times\IR^n$
definable in $\IRanexp$
(where $m=4$ and $n=2$). We have points in this set where the
first $m$ coordinates are rational, but the final $n$ most likely not.

Write $\pi_1\colon\IR^m\times\IR^n\rightarrow\IR^m$ for the projection
to the first $m$ coordinates and
$\pi_2\colon\IR^m\times\IR^n\rightarrow\IR^n$ for the projection to
the last $n$ coordinates. 

\begin{theorem}[Corollary 7.2~\cite{HabeggerPilaENS}]
  \label{thm:semirationalpw}
  Let $\epsilon>0$, there exists $c(X,\epsilon)$ with the following.
  Let $T\ge 1$ and let $\Sigma\subset X$ be a subset such that
  $$(x,y)\in
  \Sigma\Rightarrow x\in \IQ^m \text{ and }H(x)\le T.$$
  If
  $$\#\pi_2(\Sigma)>c(X,\epsilon) T^\epsilon$$
  then there exists a continuous and definable function $\beta\colon
  [0,1]\rightarrow X$ with the following properties:
  \begin{enumerate}
  \item [(i)] $\pi_1\circ\beta$ is real semi-algebraic,
  \item[(ii)] $\pi_2\circ\beta$ is non-constant,
  \item[(iii)] $\pi_2(\beta(0))\in \pi(\Sigma)$, and
  \item[(iv)] $\beta|_{(0,1)}$ is real analytic. 
  \end{enumerate}
\end{theorem}

\section{Putting Everything Together}

The strategy is now familiar.
Let $V\subset\cE^2$ be an irreducible curve defined over $\IQ$ with
$\pi(V)=Y(2)$.

We uniformize $u_{\mathrm{Betti}}\colon \IR^4\times\IH \rightarrow
\cE^{2}(\IC)$ as in Section~\ref{sec:unilegendre}.


We assume that $V(\IC)$ contains infinitely many torsion points.
If the torsion order is uniformly bounded, then $V$ is an irreducible
component of $\ker[N]$ for some $N\in\IN$.
In this case, $V\subset H_{(N,0)}$ and we are done. 

So let $P\in V(\IC)$ be of sufficiently large order $N$. Then $u(z) = P$
for some $z=(x,\tau)\in [0,1]^4\times\cF_{\mathrm{pure}}$
where $x\in\IQ^4$ satisfies $H(x)\le N$.

It follows from Corollary~\ref{cor:davidsilverman} that $\pi(P)$ and thus $P$ are
algebraic and
$$[\IQ(P):\IQ]\gg N^{1/2}.$$

Letting the Galois group $\mathrm{Gal}(\IQbar/\IQ)$ act produces
roughly $N^{1/2}$ new points in $V(\IQbar)\cap
\mathcal{E}_{\mathrm{tors}}$. They also come from points
$(x_\sigma,\tau_\sigma)$ where $x_\sigma$ is still rational of height
at most $N$.

Now the semi-rational Pila--Wilkie Theorem,
Theorem~\ref{thm:semirationalpw}, applies with $T=N$ and $\epsilon
=1/4$, say. Indeed, $\pi|_V\colon V\rightarrow Y(2)$ has finite
fibers of uniformly bounded cardinality.
So the number of distinct $\tau_\sigma$ is at least
proportional to the number of Galois conjugates of $P$.


We find a definable curve  in $X$ whose
first four components are parametrized by a real semi-algebraic
function, the final two
coordinates are parametrized by a  non-constant definable map. 

There is a suitable Ax--Lindemann--Weierstrass Theorem here. But in low
dimension an \textit{ad hoc} approach as in the work of
Masser--Zannier, Section 4~\cite{MZ:AJM10} suffices. They directly
exploit monodromy properties of the family $\cE\rightarrow Y(2)$. The
conclusion is that $V\subset H_{(a,b)}$ for some
$(a,b)\in\IZ^2\ssm\{0\}$. This is precisely the conclusion of Theorem~\ref{thm:relmm2}.


%%% Local Variables:
%%% TeX-master: "main"
%%% End:
