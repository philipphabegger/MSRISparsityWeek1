\chapter{Unlikely Intersections: Relative Manin--Mumford}

One striking feature of abelian varieties is that they deform
algebraically in families. This is already apparent in the case of
elliptic curves.

\begin{example}
  \label{ex:relmm}
  Let $\lambda\in \IC\ssm\{0,1\}$. The (long) Weierstrass equation
  \begin{equation*}
    y^2 = x(x-1)(x-\lambda)
  \end{equation*}
  determines an elliptic curve. We may consider
  $Y(2) = \IP^1\ssm\{0,1,\infty\}$ as the base of a family of elliptic
  curves. This family is called the \emph{Legendre family of elliptic
  curves}. More precisely, it is given by
  \begin{equation*}
    \cE=\{([x:y:z], [\lambda:1]) \in \IP^2 \times B : y^2z =
    x(x-z)(x-\lambda z)\}
  \end{equation*}
  and the structure morphism $\cE\rightarrow Y(2)$ induced by
  projecting.\footnote{The base curve $Y(2)$ is the modular curve of full level-$2$
  structure.} 
  We denote the fiber above $\lambda\in Y(2)(\IC)$  by $\cE_\lambda
  \subset\IP^2$.
  Of course, $\cE$ is a quasi-projective surface defined over $\IQ$.
  
  We define
  \begin{equation*}
    \cE_{\mathrm{tors}} = \bigcup_{\lambda\in Y(2)(\IC)}
    \cE_{\lambda,\mathrm{tors}}. 
  \end{equation*}

  Observe that $\cE_{\mathrm{tors}}$ is not a group in any sensible
  way. It is not possible to add two points in different fibers of
  $\cE\rightarrow Y(2)$.
  
  Suppose we are given a curve $C\subset \mathcal{E}$. Does a
  Manin--Mumford type statement hold for $C\cap
  \mathcal{E}_{\mathrm{tors}}$? In other words, is this intersection
  finite ``most of the time''?

  The answer is \textbf{no}! But why?

  Here are some very simple  heuristics. For $N\in\IZ$,  multiplication-by-$N$ is an
  morphism
  $[N]\colon \cE\rightarrow\cE$ over $Y(2)$.
  Let us denote $\ker [N] = \{P \in \cE(\IC): [N](P)=0\}$. 
  We can also write
  $\cE_{\mathrm{tors}}$ as a infinite union
  $\bigcup_{N\in\IN}\ker[N]$.
  For $N\in\IN$ the kernel  $\ker[N]$ is a (possibly reducible) curve
  in $\cE$. For example,
  \begin{equation*}
    \ker [1] = \{[0:1:0]\}\times Y(2)
  \end{equation*}
  and
  \begin{equation*}
    \ker[2] = \bigl(\{[0:0:1],[1:0:1],[0:1:0]\}\times Y(2)\bigr)\cup
    \{([\lambda:0:1],\lambda) : \lambda \in Y(2)(\IC)\}. 
  \end{equation*}

  Heuristically we would expect that the two curves $C$ and $\ker[N]$
  inside the ambient space $\cE$ have non-empty intersection. As we
  vary over infinitely many $N$ we would heuristically expect
  infinitely many points in $\bigcup_{N\in\IN}C(\IC)\cap \ker[N]$.

  Of course, there are two difficulties making this  argument precise.
  First, we would have to verify that $\ker[N]\cap C$ is non-empty, at
  least for infinitely many $N$. Second, we would need to rule out
  excessive double counting when taking the union over all $N$. % Below,
  % we
  % will give a different argument that shows $C(\IC)\cap\cE_{\mathrm{tors}}$
  % is infinite.

  The heuristic argument becomes invalid if we consider a curve in the
  fibered power $\cE\times_{Y(2)} \cE =\cE^2$ instead. Indeed, here $\ker[N]$
  are still one-dimensional. But the ambient variety $\cE^2$ is now
  three-dimensional. So it is unlikely that a curve $C$ and $\ker[N]$
  should intersect.
\end{example}

\begin{definition}  Let $m\in\IN$.
  \begin{enumerate}
  \item [(i)] The $m$-fold fibered power of $\cE\rightarrow Y(2)$
    is denoted by $\cE^m = \cE\times_{Y(2)}\cdots\times_{Y(2)}\cE$.
    Let $\pi\colon\cE^m\rightarrow Y(2)$ be the natural projection.
  \item[(ii)] For $a=(a_1,\ldots,a_m)\in\IZ^m$ we have a morphism
    $\varphi_a \colon \cE^m\rightarrow \cE$ over $Y(2)$
    determined  by $P\mapsto
    [a_1](P_1)+\cdots +[a_m](P_m)$. We also defined $H_a =
    \{(P_1,\ldots,P_m) \cE^m(\IC) : \varphi_a(P_1,\ldots,P_m)=0\}$. 
  \item[(iii)]  We write $\cE^m_{\mathrm{tors}} = \bigcup_{\lambda\in Y(2)(\IC)}
    \cE^m_{\mathrm{tors}}$. 
  \end{enumerate}
\end{definition}

Then $\cE^m$ is a family of abelian varieties. Each fiber
$\cE^m_\lambda$ is the $m$-th power of an elliptic curve.

The following Conjecture was stated by Pink~\cite{Pink} and later by
Masser--Zannier~\cite{MZ:AJM10}, but in the more general setting of
abelian schemes.

\begin{conjecture}[Relative Manin--Mumford in $\cE^m$]
  \label{conj:relmmEm}
  Let $V$ be an irreducible subvariety of $\cE^m$ with $\pi(V)=Y(2)$.
  Suppose that  
  \begin{equation*}
    C\not\subset H_a
  \end{equation*}
  for all $a\in\IZ^{m}\ssm\{0\}$.
  Then $V\cap \cE^m_{\mathrm{tors}}$ is not Zariski dense.
\end{conjecture}



This conjecture is part of a much more general conjecture posed by
Pink~\cite{Pink} and studied in special cases by
Bombieri--Masser--Zannier~\cite{BMZ,BMZgeometric} and
Zilber~\cite{Zilber}. We refer to Zannier's volume~\cite{ZannierBook}
for more background. 

Conjecture~\ref{conj:relmmEm} was proved by Masser and
Zannier~\cite{MZ:AJM10} for $\dim V = 1$. We explicitly mention the
case $m=2$ and $\dim V=1$ here.

\begin{theorem}
  \label{thm:relmm2}
  Let $V$ be an irreducible curve in $\cE^2$ defined over $\IQ$ with
  $\pi(V)=Y(2)$. Suppose that $V\not\subset H_{(a,b)}$ for all
  $(a,b)\in\IZ^2\ssm\{0\}$. Then $V\cap \cE^2_{\mathrm{tors}}$ is
  finite.
\end{theorem}

\begin{remark}
  \begin{enumerate}
  \item [(i)]In the theorem and the conjecture we imposed the condition
    $\pi(V)=Y(2)$ for the following reason. The morphism $\pi$ is proper.
    So we  have either $\pi(V)=Y(2)$ or $\pi(V)$ is a point. In the second
    case, $V$ is a subvariety of the constant abelian variety above the
    singleton $\pi(V)$. The intersection of $V$ with the torsion happens
    in a single abelian variety and we  are reduced to the (known!)
    Manin--Mumford  Conjecture. 
  \item[(ii)] Why do we impose $V\not\subset H_a$ resp. $C\not\subset
    H_{(a,b)}$? In the case of curves suppose  $C\subset H_{(a,b)}$ with
    $(a,b)\not=0$.
    Then $C$ is contained, up-to torsion, in
    a family of abelian subvarieties of $\cE^2$. So the additional
    coordinate that was introduced in rule out the heuristic in the
    end of example~\ref{ex:relmm} was a red herring. And indeed, in this case
    an updated heuristic  suggests that  $C\cap
    \cE^2_{\mathrm{tors}}$ should be infinite. 
  \end{enumerate} 
\end{remark}


\begin{exercise} 
  Formulate a conjecture akin to Conjecture \ref{conj:relmmEm} including a
  a necessary and  sufficient condition from when $V(\IC)\cap
  \mathcal{E}^m_{\mathrm{tors}}$ is Zariski dense in $V$. Hint: look at
  the Manin--Mumford Conjecture. 
\end{exercise}

\section{A Uniformization of the Legendre Family}
\label{sec:unilegendre}
Our goal is find a suitable uniformization of $\cE(\IC)$ and then
to adapt the Pila--Zannier strategy to prove Theorem~\ref{thm:relmm2}.

The analog for $Y(2)$ of Klein's modular $j$-function is the modular
$\lambda$-function. This is a holomorphic map $\lambda\colon
\IH\rightarrow\IC\ssm\{0,1\}=Y(2)(\IC)$ that satisfies
\begin{equation*}
  \lambda\left(\frac{a\tau+b}{c\tau +d}\right) = \lambda(\tau)
\end{equation*}
for all $\tau\in\IN$ and all
\begin{equation*}
  \mattt{a}{b}{c}{d}  \in \Gamma(2) = \{\gamma \in
  \mathrm{SL}_{2}(\IZ) : \gamma\equiv I \mod 2\}.
\end{equation*}

In particular, $\lambda(\tau+2)=\lambda(\tau)$ and so there is a
development in $q = e^{\pi i \tau}$. The first few terms are
\begin{equation*}
  \lambda(\tau) = 16q - 128q^2 + 704q^3 +\cdots. 
\end{equation*}
The function is connected to Klein's $j$-function via
\begin{equation*}
  j(\tau) = 2^8
  \frac{(\lambda(\tau)^2-\lambda(\tau)+1)^3}{\lambda(\tau)^2(\lambda(\tau)-1)^2}. 
\end{equation*}
And $\lambda(\tau)$ is the $j$-invariant of the elliptic curve
determined by $y^2 = x(x-1)(x-\lambda(\tau))$.

Let $\cF_{\mathrm{pure}}$ be a fundamental domain for the action of $\Gamma(2)$ on
$\IH$. As in the case of the action of the full modular group, we can
take $\cF_{\mathrm{pure}}$ to be real semi-algebraic.

For definiability of $\lambda$ restricted to $\cF_{\mathrm{pure}}$  in
$\IRanexp$ we refer to work of Peterzil--Starchenko~\cite{PS:Duke13}.
However, for it turns out that we can work on some compact box in
$\IH$ and ultimately $\IRan$ can do the job. 

The function $\lambda$ uniformizes the base $Y(2)$. 
Using theta functions we can uniformize the fibers. In other words,
and after doubling coordinates,
there is a holomorphic map $u$ with
\begin{equation*}
  \begin{tikzcd}[column sep=small, row sep=small] 
    & \IC^2\times\IH  \arrow{dd}{} \arrow{r}{u} & \cE^2(\IC)\arrow{dd}{}\\
    &  &\\
    &    \IH \arrow{r}{\lambda}  & Y(2)(\IC).
  \end{tikzcd}
\end{equation*}

Each fiber $\cE_{\lambda(\tau)}$ of $\cE\rightarrow Y(2)$ is
uniformized by $\IC\rightarrow \cE_{\lambda(\tau)}(\IC)$ with kernel
$\IZ+\tau\IZ$.
We also obtain a relative fundamental domain
\begin{equation*}
  \cF_{\mathrm{mixed}} = \{(z_1,z_2,\tau) \in
  \IC^2\times \cF_{\mathrm{pure}} \text{ with }z_i=a_i+b_i\tau\text{
    where }a_1,b_1,a_2,b_2\in [0,1]  \}. 
\end{equation*}
So $\cF_{\mathrm{mixed}}$
is a real semi-algebraic set in $\IR^4\times \IH$ where, as
usual, we identify $\IC\times \IH$ with $\IR^4\times\IR\times(0,\infty)$.

In our original uniformization we do a real change of coordinates as
follows. 
For $(a_1,b_1,a_2,b_2,\tau)\in [0,1]^4\times \cF_{\mathrm{mixed}}$ we define
\begin{equation*}
   u_{\mathrm{Betti}}(a_1,b_1,a_2,b_2,\tau) = u(a_1+b_1\tau,a_2+b_2\tau,\tau)
\end{equation*}
and get a map $u_{\mathrm{Betti}}\colon \IR^4\times \cF_{\mathrm{pure}}\rightarrow\cE(\IC^2)$
and a fundamental domain 
\begin{equation*}
  \cF = [0,1]^4\times \cF_{\mathrm{pure}}. 
\end{equation*}

Let $V$ be an irreducible curve in $\cE^2$. Then
\begin{equation*}
  X = u_{\mathrm{Betti}}|_{\cF}^{-1}(V(\IC)) \subset \cF
\end{equation*}
is definable in $\IRanexp$.


Suppose $P\in V(\IC)\cap\cE_{\mathrm{tors}}^2$ and let
$z=(a_1,b_1,a_2,b_2,\tau)\in \cF$ with $u(z)=P$.
Now it will become clear why we did the real change of coordinates:
Since $P$ is torsion in $\cE^2_{\lambda(\tau)}$, any preimage under
the
uniformization $\IC\rightarrow \IC/(\IZ+\tau\IZ)$ lies in
$\IQ+\tau\IQ$. This means that $(a_1,b_1,a_2,b_2)\in \IQ^4$. Moreover,
$H(a_1,b_1,a_2,b_2)\le N$ where $N$ is the order of $P$. 

Suppose $V$ is defined over $\IQ$. 
We would like to apply the Pila--Wilkie Theorem. However, we must deal
with two issues, one old, with a twist, and a new one.

\begin{enumerate}
\item [(Old)] We need a lower bound for $[\IQ(P):\IQ]$ that grows like
  some small but fixed power of $N$. The added difficulty here when
  compare to the absolute Manin--Mumford Conjecture is that the
  elliptic curve $\cE_{\lambda(\tau)}$ is varying. So we cannot allude
  to Masser's Theorem~\ref{thm:masser} for a fixed abelian variety.
  We will handle this issue in
  Section~\ref{sec:lbdegrel}. 
  
\item[(New)] The point $z\in \IR^4\times\IH$ produced form $P$ is
  most likely \emph{not} a rational point. Indeed, only the first four
  coordinates $(a_1,b_1,a_2,b_2)$ are rational, but there is no
  reason to believe that $\tau$ is algebraic.\footnote{Indeed,
    expect if $\cE_{\lambda(\tau)}$ has complex multiplication, a
    theorem of Schneider implies that $\tau$ is transcendental.}
  So  $z$ is merely \textit{semi-rational}. We will present a
  semi-rational version of the Pila--Wilkie Theorem in
  Section~\ref{sec:semirational}. 
\end{enumerate}

\section{A Lower Bound for the Degree}
\label{sec:lbdegrel}

Next we need to make Masser's result, Theorem~\ref{thm:masser},
quantitative in terms of the elliptic curve.

\begin{theorem}[David]
  There exists $c>0$ with the following property. 
  Let $E=\cE_\lambda$ with $\lambda\in\IQbar\ssm\{0,1\}$. If $P\in
  E(K)$ has finite order $N$ where $K$ is a number field, then
  \begin{equation}
    \label{eq:davidKQlb}
    [K:\IQ] \ge c \frac{N^{1/2}}{\max\{1,\log H(\lambda)\}}.
  \end{equation}    
\end{theorem}
\begin{proof}
  This follows from Th\'eor\`eme 1.2~\cite{DavidPetiteHauteur}.
  Indeed, one can bound
  the height of the $j$-invariant of $E$ from above polynomially in
  terms of $H$. 
\end{proof}

We need the following consequence of a theorem of Silverman. Roughly,
it tells us that we can safely ignore the height that appears in the
denominator of (\ref{eq:davidKQlw}). Recall that $\pi\colon
\cE^2\rightarrow Y(2)$ is a family of squares of Legendre elliptic
curves.

\begin{theorem}
  Let $V\subset \cE^2$ be a curve defined over $\IQbar$. There exists 
  $c(V)>0$ such that if $P\in V\cap\mathcal{E}^2_{\mathrm{tors}}$  has
  order $>c(V)$, then $\pi(P)\in \IQbar\ssm\{0,1\}$ and $H(\pi(P))\le c(V)$. 
\end{theorem}
\begin{proof}
  The intersection $C\cap \ker[N]$ has dimension $0$ except if $C$ is
  an irreducible component of $\ker[N]$. In this case, all points on
  $C$ have some fixed order. So we may assume $C\cap \ker[N]$ finite
  for all $N\in\IN$. In particular, any point in this intersection is
  algebraic. We can then apply Silverman's Theorem \cite{Silverman}. 
\end{proof}

Observe that Silverman's Theorem only tells us that the height of
$\pi(P)$ is bounded from above. This does not imply that the set of
$\pi(P)$ is finite. The point is that  $\pi(P)$ could have
\textit{degree} over $\IQ$,
just as in the sequence $2^{1/d}$ whose members have height
$H(2^{1/d})=2^{1/d}\le 2$. 


\begin{corollary}
  \label{cor:davidsilverman}
  Let $V\subset \cE^2$ be a curve defined over $\IQ$. There exists 
  $c(V)>0$ such that if $P\in V(\IC)\cap\mathcal{E}^2_{\mathrm{tors}}$  has
  order $N>c(V)$, then $P$ is algebraic and 
  $[\IQ(P):\IQ] \ge c(V) N^{1/2}$. 
\end{corollary}

\begin{exercise}
  Let $B>0$ and  let $\epsilon > 0$. Show that there exists
  $\delta =\delta(B,\epsilon)>0$ with the following property. If
  $\alpha\in\IC\ssm\{0,1\}$ is algebraic with Galois conjugates
  $\alpha_1,\ldots,\alpha_d\in\IC$ (so $d=[\IQ(\alpha):\IQ]$), then 
  \begin{equation*}
    \#\{ i : |\alpha_i|> \epsilon \text{ and
    }|\alpha_i-1|>\epsilon\text{ and }|\alpha_i|<\epsilon^{-1}\} \le
    \delta d.
  \end{equation*}
  So a positive proportion of all Galois conjugates of $\alpha$ are
  not too close to one of the three cusps $0,1,\infty$ of $Y(2)$. This
  also us to work in the o-minimal structure $\IRan$ and use simpler
  definability states when uniformizing $Y(2)$ and ultimately $\cE$. 
\end{exercise}

\section{Semi-rational Pila--Wilkie}
\label{sec:semirational}

Our approach to the Relative Manin--Mumford Conjecture for a curve in
$\cE$ has led to a set $X\subset \IR^m\times\IR^n$
definable in $\IRanexp$
(where $m=4$ and $n=2$). We have points in this set where the
first $m$ coordinates are rational, but the final $n$ most likely not.

Write $\pi_1\colon\IR^m\times\IR^n\rightarrow\IR^m$ for the projection
to the first $m$ coordinates and
$\pi_2\colon\IR^m\times\IR^n\rightarrow\IR^n$ for the projection to
the last $n$ coordinates. 

\begin{theorem}[Corollary 7.2~\cite{HabeggerPilaENS}]
  \label{thm:semirationalpw}
  Let $\epsilon>0$, there exists $c(X,\epsilon)$ with the following.
  Let $T\ge 1$ and let $\Sigma\subset X$ be a subset such that
  $$(x,y)\in
  \Sigma\Rightarrow x\in \IQ^m \text{ and }H(x)\le T.$$
  If
  $$\#\pi_2(\Sigma)>c(X,\epsilon) T^\epsilon$$
  then there exists a continuous and definable function $\beta\colon
  [0,1]\rightarrow X$ with the following properties:
  \begin{enumerate}
  \item [(i)] $\pi_1\circ\beta$ is real semi-algebraic,
  \item[(ii)] $\pi_2\circ\beta$ is non-constant,
  \item[(iii)] $\pi_2(\beta(0))\in \pi(\Sigma)$, and
  \item[(iv)] $\beta|_{(0,1)}$ is real analytic. 
  \end{enumerate}
\end{theorem}

\section{Putting Everything Together}

The strategy is now familiar.
Let $V\subset\cE^2$ be an irreducible curve defined over $\IQ$ with
$\pi(V)=Y(2)$.

We uniformize $u_{\mathrm{Betti}}\colon \IR^4\times\IH \rightarrow
\cE^{2}(\IC)$ as in Section~\ref{sec:unilegendre}.


We assume that $V(\IC)$ contains infinitely many torsion points.
If the torsion order is uniformly bounded, then $V$ is an irreducible
component of $\ker[N]$ for some $N\in\IN$.
In this case, $V\subset H_{(N,0)}$ and we are done. 

So let $P\in V(\IC)$ be of sufficiently large order $N$. Then $u(z) = P$
for some $z=(x,\tau)\in [0,1]^4\times\cF_{\mathrm{pure}}$
where $x\in\IQ^4$ satisfies $H(x)\le N$.

It follows from Corollary~\ref{cor:davidsilverman} that $\pi(P)$ and thus $P$ are
algebraic and
$$[\IQ(P):\IQ]\gg N^{1/2}.$$

Letting the Galois group $\mathrm{Gal}(\IQbar/\IQ)$ act produces
roughly $N^{1/2}$ new points in $V(\IQbar)\cap
\mathcal{E}_{\mathrm{tors}}$. They also come from points
$(x_\sigma,\tau_\sigma)$ where $x_\sigma$ is still rational of height
at most $N$.

Now the semi-rational Pila--Wilkie Theorem,
Theorem~\ref{thm:semirationalpw}, applies with $T=N$ and $\epsilon
=1/4$, say. Indeed, $\pi|_V\colon V\rightarrow Y(2)$ has finite
fibers of uniformly bounded cardinality.
So the number of distinct $\tau_\sigma$ is at least
proportional to the number of Galois conjugates of $P$.


We find a definable curve  in $X$ whose
first four components are parametrized by a real semi-algebraic
function, the final two
coordinates are parametrized by a  non-constant definable map. 

There is a suitable Ax--Lindemann--Weierstrass Theorem here. But in low
dimension an \textit{ad hoc} approach as in the work of
Masser--Zannier, Section 4~\cite{MZ:AJM10} suffices. They directly
exploit monodromy properties of the family $\cE\rightarrow Y(2)$. The
conclusion is that $V\subset H_{(a,b)}$ for some
$(a,b)\in\IZ^2\ssm\{0\}$. This is precisely the conclusion of Theorem~\ref{thm:relmm2}.


%%% Local Variables:
%%% TeX-master: "main"
%%% End:
